\documentclass[a3paper, twocolumn, 30pt]{jsarticle}

\usepackage[top = 10truemm, bottom = 20truemm, left = 20truemm, right = 20truemm]{geometry} % PDFの余白を調整しています. 
% 数式
\usepackage{amsmath,amsfonts}
\usepackage{amsthm}
\usepackage{bm}
\usepackage{mathtools}
\usepackage{amssymb}

% 表
\usepackage[utf8]{inputenc}
\usepackage{diagbox} % 斜線付きセルを作成するために必要
\usepackage{booktabs} % 表の罫線を美しくするために必要
\usepackage{hhline} % 水平罫線を制御するために必要

% 画像
\usepackage[dvipdfmx]{graphicx}
\usepackage{ascmac}
\usepackage{physics}
\usepackage{float} % 追加

% 図
\usepackage[dvipdfmx]{graphicx}
\usepackage{tikz} %図を描く
\usetikzlibrary{positioning, intersections, calc, arrows.meta,math} %tikzのlibrary

\pagestyle{empty} % ページ番号を消しています. 


\title{永久機関は存在するか?} % 「タイトル」の部分を実際のポスターのタイトルに書き替えてください. 
\author{理学部物理学科3年 大上由人} % 「名前」の部分を自分の名前に書き替えてください. 
\date{} % 日付を削除

\begin{document}

\maketitle % 上で設定したタイトル・名前・日付を出力しています. 

\setlength{\columnseprule}{0.1pt} % PDFの真ん中に2段組を分かり易くするための縦の線分を出力しています.「0.1pt」は線の太さです. 

% 以下に本文を書いてください. 
\section{観察1}
\begin{itembox}[l]{\textbf{観察1}}
    どうやら自然界には、可能な操作と不可能な操作があるようだ。
\end{itembox}
\textbf{例}
\begin{itemize}
    \item 断熱容器に対して、仕事をすることで気体の温度を上げることはできるが、逆に気体の温度を下げることはできない。
\end{itemize}

\section{断熱到達可能性}
\begin{itembox}[l]{\textbf{定義1}}
    ある状態から別の状態に、熱のやりとりをせずに到達できるとき、状態間は断熱到達可能であるという。このとき、記号$\rightarrow$を用いて、
    \begin{equation}
        X \rightarrow Y
    \end{equation}
    と表す。また、両向きに到達可能なとき、
    \begin{equation}
        X \leftrightarrow Y
    \end{equation}
    と表す。
\end{itembox}

やりたいことは、$\rightarrow$がもっているルールを考察し、そのルールを用いて、定量的に状態間の断熱到達可能性を評価することである。

\begin{itembox}[l]{\textbf{公理}}
    $\rightarrow$は以下のルールを満たす。
    \begin{itemize}
        \item A1: $ X \leftrightarrow X$
        \item A2: $ X \rightarrow Y \rightarrow Z \Rightarrow X \rightarrow Z$
        \item A3: $ X \rightarrow X'$かつ$Y \rightarrow Y' \Rightarrow (X,Y) \rightarrow (X',Y')$
        \item A4: $ X \rightarrow Y \Rightarrow tX \rightarrow tY$
        \item A5: $ X \leftrightarrow (tX,(1-t)X)$
        \item A6: $ (X,\epsilon_1 Z) \rightarrow (Y,\epsilon_2 Z) \Rightarrow X \rightarrow Y$
        \end{itemize}
\end{itembox}

それぞれのルールがどのようなものかをを見ていく。\\

\section{エントロピー}
以上のルールをもとに、状態の断熱到達可能性を定量的に評価したい。そこで用いられる量がエントロピーである。\\
\begin{itembox}[l]{\textbf{定義:エントロピー係数/エントロピー}}
    ある状態$X_0,X_1$が$X_0 \rightarrow X_1$であるとする。このとき、状態$X$のエントロピー係数$\lambda_X$は、
    \begin{equation}
        \lambda_X = \text{sup} \left\{ ((1-\lambda)X_0 + \lambda X_1) \rightarrow X \right\}
    \end{equation}
    で定義される。また、状態$X$のエントロピー$S_X$は、
    \begin{equation}
        S_X = \lambda_X s^*(\text{単位エントロピー})
    \end{equation}
    で定義される。
\end{itembox}
このとき、以下の定理が成り立つことが知られている。
\begin{itembox}[l]{\textbf{定理:エントロピー原理}}
    以下の二つは同値である。
    \begin{itemize}
        \item $\rightarrow$がA1-A6を満たす。
        \item エントロピー$S$について、
        \begin{equation}
            X \rightarrow Y \Leftrightarrow S_X \geq S_Y
        \end{equation}
        が成り立ち、かつ、このようなエントロピーは、原点と目盛りの選び方を除いて一意である。
    \end{itemize}
\end{itembox}
以上の定理から、「状態は、エントロピーが増える向きにしか到達できない」ということがわかる。\\

\section{永久機関は存在するか?}
以上の議論をもとに、永久機関が存在するかを考える。
\begin{itembox}[l]{\textbf{定義:第一種永久機関/第二種永久機関}}
    \begin{itemize}
        \item 第一種永久機関: 外部から何も受け取ることなく、仕事をする機械
        \item 第二種永久機関: 熱効率が100\%の機械
    \end{itemize}
\end{itembox}

\subsection{第一種永久機関}
第一永久機関は自明に存在しない。というのも、エネルギー保存則から、仕事をする機械は、何かしらのエネルギーを受け取らなければならないからである。\\

\subsection{第二種永久機関}
第二永久機関が存在しないことを示す。

\end{document}
