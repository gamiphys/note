\documentclass[a4paper,11pt]{jsarticle}

% 数式
\usepackage{amsmath,amsfonts}
\usepackage{amsthm}
\usepackage{bm}
\usepackage{mathtools}
\usepackage{amssymb}

% 表
\usepackage[utf8]{inputenc}
\usepackage{diagbox} % 斜線付きセルを作成するために必要
\usepackage{booktabs} % 表の罫線を美しくするために必要
\usepackage{hhline} % 水平罫線を制御するために必要

% 画像
\usepackage[dvipdfmx]{graphicx}
\usepackage{ascmac}
\usepackage{physics}
\usepackage{float} % 追加

% 図
\usepackage[dvipdfmx]{graphicx}
\usepackage{tikz} %図を描く
\usetikzlibrary{positioning, intersections, calc, arrows.meta,math} %tikzのlibrary

% ハイパーリンク
\usepackage[dvipdfm,
  colorlinks=false,
  bookmarks=true,
  bookmarksnumbered=false,
  pdfborder={0 0 0},
  bookmarkstype=toc]{hyperref}

% 式番号を章ごとにリセット
\numberwithin{equation}{section}

\begin{document}

\title{多様体}
\author{大上由人}
\date{\today}
\maketitle

\section{写像の微分}
$M,N$を多様体、$m,n$次元$C^r$級多様体とし、$f: M \to N$を$C^r$級写像とする。
点$p \in M$を通るような$M$上の$C^r$級曲線
\begin{align}
    c: (-\varepsilon, \varepsilon) \to M \quad (c(0) = p)
\end{align}
を考える。この曲線を$f$でうつすと、$f(p)$を通る$N$上の$C^r$級曲線
\begin{align}
    f \circ c: (-\varepsilon, \varepsilon) \to N \quad ((f \circ c)(0) = f(p))
\end{align}
が得られる。ここでは、$t=0$での曲線$c$の速度ベクトルと、$t=0$での曲線$f \circ c$の速度ベクトルの関係を調べる。

$T_pM$の任意の元$\vb{v}$をとる。このとき、$\left.\dv{c}{t}\right|_{t=0} = \vb{v}$となるような、$p$を通る$C^r$級曲線
\begin{align}
    c: (-\varepsilon, \varepsilon) \to M \quad (c(0) = p)
\end{align}
が存在する。この曲線を写像$f: M \to N$でうつすと、$q = f(p)$を通る$C^r$級曲線
\begin{align}
    f \circ c: (-\varepsilon, \varepsilon) \to N \quad ((f \circ c)(0) = q)
\end{align}
が得られる。$t=0$におけるこの曲線の速度ベクトルは、
\begin{align}
    \vb{w} = \left.\dv{t} (f \circ c)\right|_{t=0}
\end{align}
である。このようにして、$T_pM$の元$\vb{v}$に対して$T_qN$の元$\vb{w}$が対応する。また、この対応は曲線の取り方によらないことが示せる。
これにより、$T_pM$の元$\vb{v}$に対して$T_qN$の元$\vb{w}$が対応する写像として、微分が定義される。

\begin{itembox}[l]{\textbf{Def.写像の微分}}
  上の対応で定まる写像
  \begin{align}
    (df)_p: T_pM \to T_{f(p)}N
  \end{align}
  を、$f: M \to N$の$p$における微分という。
\end{itembox}
この写像に"微分"という名前がついていることを納得するために、以下の例を考えてみる。
\textbf{ex.}\\
$M = \mathbb{R}$, $N = \mathbb{R}$, $f(x) = x^2$とする。このとき、$p = 1$における$f$の微分$(df)_1$は、
\begin{align}
  (df)_1: T_1\mathbb{R} \to T_1\mathbb{R}
\end{align}
である。$T_1\mathbb{R}$の元は、$1$における接ベクトルである。$1$における接ベクトルは、$1$を通る曲線の速度ベクトルである。$1$を通る曲線は、$c(t) = 1 + t$である。この曲線を$f$でうつすと、
\begin{align}
  f \circ c(t) = f(1 + t) = (1 + t)^2
\end{align}
となる。$t=0$におけるこの曲線の速度ベクトルは、
\begin{align}
  \left.\dv{t} (f \circ c)\right|_{t=0} = 2
\end{align}
である。したがって、$(df)_1$は、
\begin{align}
  (df)_1(2) = 2
\end{align}

写像の微分を成分表示する。
hogehoge

\begin{align}
  \begin{pmatrix}
    w_1 \\
    w_2 \\
    \vdots \\
    w_n
  \end{pmatrix}
  =
  \begin{pmatrix}
    \pdv{f_1}{x_1}(p) & \pdv{f_1}{x_2}(p) & \cdots & \pdv{f_1}{x_m}(p) \\
    \pdv{f_2}{x_1}(p) & \pdv{f_2}{x_2}(p) & \cdots & \pdv{f_2}{x_m}(p) \\
    \vdots & \vdots & \ddots & \vdots \\
    \pdv{f_n}{x_1}(p) & \pdv{f_n}{x_2}(p) & \cdots & \pdv{f_n}{x_m}(p)
  \end{pmatrix}
  \begin{pmatrix}
    v_1 \\
    v_2 \\
    \vdots \\
    v_m
  \end{pmatrix}
\end{align}



\end{document}