\documentclass[a4paper,11pt]{jsarticle}

% 数式
\usepackage{amsmath,amsfonts}
\usepackage{amsthm}
\usepackage{bm}
\usepackage{mathtools}
\usepackage{amssymb}

% 表
\usepackage[utf8]{inputenc}
\usepackage{diagbox} % 斜線付きセルを作成するために必要
\usepackage{booktabs} % 表の罫線を美しくするために必要
\usepackage{hhline} % 水平罫線を制御するために必要

% 画像
\usepackage[dvipdfmx]{graphicx}
\usepackage{ascmac}
\usepackage{physics}
\usepackage{float} % 追加

% 図
\usepackage[dvipdfmx]{graphicx}
\usepackage{tikz} %図を描く
\usetikzlibrary{positioning, intersections, calc, arrows.meta,math} %tikzのlibrary

% ハイパーリンク
\usepackage[dvipdfm,
  colorlinks=false,
  bookmarks=true,
  bookmarksnumbered=false,
  pdfborder={0 0 0},
  bookmarkstype=toc]{hyperref}

% 式番号を章ごとにリセット
\numberwithin{equation}{section}

\begin{document}

\title{多様体}
\author{大上由人}
\date{\today}
\maketitle

\section{多様体の定義}
\begin{itembox}[l]{\textbf{Def.多様体}}
  $m$次元可微分多様体とは、次の条件を満たす位相空間$M$のことである。
  \begin{enumerate}
    \item $M$はハウスドルフかつパラコンパクトである。
    \item 座標近傍と呼ばれる開集合と同相写像の組$(U_i, \varphi_i)$と、その集合でアトラスと呼ばれる集合族$\{(U_i, \varphi_i)\}$が存在し、次の条件を満たす。
    \begin{enumerate}
      \item $\bigcup_i U_i = M$
      \item $\varphi_i: U_i \to \mathbb{R}^m$は同相写像である。
      \item $\varphi_i \circ \varphi_j^{-1}: \varphi_j(U_i \cap U_j) \to \varphi_i(U_i \cap U_j)$は$C^\infty$級写像である。
    \end{enumerate}
  \end{enumerate}
\end{itembox}

多様体がHaussdorffであるご利益は、以下の命題による。\\
\begin{itembox}[l]{\textbf{Prop.}}
  位相空間$M$がハウスドルフ空間であるとする。このとき、$M$の点列が極限点をもてば、その極限点はただ一つである。
\end{itembox}
すなわち、ハウスドルフ空間において、極限を定義できるようになる。多様体上での関数の収束や、微分などを考えるときに、この性質は非常に重要である。\\

\section{多様体上の関数/写像}

\begin{itembox}[l]{\textbf{Def.多様体の写像が$C^r$級である}}
  多様体$M$から多様体$N$への写像
  \begin{align}
    f: M \to N
  \end{align}
  が、1点$p \in M$において$C^r$級であるとは、$p$の任意の座標近傍$(U, \varphi)$と、$f(p)$の任意の座標近傍$(V, \psi)$が存在して、次の条件を満たすことをいう。
  \begin{enumerate}
    \item $f(U) \subset V$
    \item $\psi \circ f \circ \varphi^{-1}: \varphi(U) \to \psi(V)$が$C^r$級である
  \end{enumerate}
  また、$f: M \to N$が$C^r$級であるとは、$f$が$M$の任意の点$p$において$C^r$級であることをいう。
\end{itembox}
要するに、一旦座標近傍に引き戻して、
$\mathbb{R}^m$上の関数として考えることで、$C^r$級性を定義している。\\
とくに、多様体上の\textbf{関数}とは、$N$が$\mathbb{R}$であるときの写像のことをいう。逆に、$M$が一次元空間$\mathbb{R}$であるとき、
\begin{align}
  c: \mathbb{R} \to N
\end{align}
を、$N$上の\textbf{曲線}という。

\section{接ベクトル空間}
\subsection{方向微分}
多様体$M$上で、点$p$を通るようななめらかな曲線$c: (-\varepsilon, \varepsilon) \to M$を考える。$p$の周りで座標近傍$(U, \varphi)$をとると、曲線の座標表示は、
\begin{align}
  c(t) = (x^1(t), x^2(t), \ldots, x^m(t))
\end{align}
である。このとき、$t=0$における曲線の速度ベクトルは、
\begin{align}
  \dv{t} c(t) = \qty(\dv{x^1}{t}, \dv{x^2}{t}, \ldots, \dv{x^m}{t})
\end{align}
である。しかし、この速度ベクトルの表示は、局所座標の取り方に依存してしまう。そこで、速度ベクトルを一般化することを考える。\\
準備として、$p$の開近傍$U$で定義された$C^r$級関数$f: U \to \mathbb{R}$を考える。このとき、$c$と$f$の合成関数
\begin{align}
  f \circ c: (-\varepsilon, \varepsilon) \to \mathbb{R}
\end{align}
を作ることができる。この、関数$f$に対してこの微分係数を対応させる対応
\begin{align}
  f \mapsto \left.\dv{t} (f \circ c)\right|_{t=0} 
\end{align}
を、$c$における$f$の方向微分といい、
\begin{align}
  \vb{v}_c = \left.\dv{t} (f \circ c)\right|_{t=0}
\end{align}
で表す。このとき、方向微分は以下の性質を持つ。\\
\begin{itembox}[l]{\textbf{方向微分の性質}}
  \begin{enumerate}
    \item $f,g$が点$p$の開近傍$U$で定義された$C^r$級関数で、しかも、$p$のある十分小さな開近傍上で$f=g$であるとする。このとき、
    \begin{align}
      \vb{v}_c(f) = \vb{v}_c(g)
    \end{align}
    が成り立つ。
    \item 線形性
    \begin{align}
      \vb{v}_c(f + g) = \vb{v}_c(f) + \vb{v}_c(g)
    \end{align}
    が成り立つ。
    \item Leibniz則
    \begin{align}
      \vb{v}_c(fg) = f(p) \vb{v}_c(g) + g(p) \vb{v}_c(f)
    \end{align}
    が成り立つ。
  \end{enumerate}
\end{itembox}
これらの性質を用いて、方向微分を定義する。\\

\begin{itembox}[l]{\textbf{Def.方向微分}}
  点$p$における方向微分$\vb{v}$とは、上の1,2,3の性質を満たす写像である。
\end{itembox}
このとき、方向微分すべての集合$D(p)$はベクトル空間をなす。

\subsection{接ベクトル空間}
\begin{itembox}[l]{\textbf{Def.接ベクトル空間}}
  多様体$M$の点$p$における接ベクトル空間$T_pMとは、$以下のベクトルが張る$D(p)$の部分空間のことをいう。
  \begin{align}
    \qty{\qty(\pdv{x^1})_p, \qty(\pdv{x^2})_p, \ldots, \qty(\pdv{x^m})_p}
  \end{align}  
\end{itembox}
このとき、接ベクトル空間が、局所座標の取り方に寄らないことが示される。\\

以上の準備の下、速度ベクトルを一般化する。
\begin{itembox}[l]{\textbf{Def.速度ベクトル}}
  $c: (-\varepsilon, \varepsilon) \to M$かつ$c(0) = p$である曲線の$t=0$における速度ベクトルとは、
  \begin{align}
    \vb{v}_c = \left.\dv{t} c(t)\right|_{t=0} 
  \end{align}
  で定義される接ベクトルである。
\end{itembox}



\section{写像の微分}
$M,N$を多様体、$m,n$次元$C^r$級多様体とし、$f: M \to N$を$C^r$級写像とする。
点$p \in M$を通るような$M$上の$C^r$級曲線
\begin{align}
    c: (-\varepsilon, \varepsilon) \to M \quad (c(0) = p)
\end{align}
を考える。この曲線を$f$でうつすと、$f(p)$を通る$N$上の$C^r$級曲線
\begin{align}
    f \circ c: (-\varepsilon, \varepsilon) \to N \quad ((f \circ c)(0) = f(p))
\end{align}
が得られる。ここでは、$t=0$での曲線$c$の速度ベクトルと、$t=0$での曲線$f \circ c$の速度ベクトルの関係を調べる。

$T_pM$の任意の元$\vb{v}$をとる。このとき、$\left.\dv{c}{t}\right|_{t=0} = \vb{v}$となるような、$p$を通る$C^r$級曲線
\begin{align}
    c: (-\varepsilon, \varepsilon) \to M \quad (c(0) = p)
\end{align}
が存在する。この曲線を写像$f: M \to N$でうつすと、$q = f(p)$を通る$C^r$級曲線
\begin{align}
    f \circ c: (-\varepsilon, \varepsilon) \to N \quad ((f \circ c)(0) = q)
\end{align}
が得られる。$t=0$におけるこの曲線の速度ベクトルは、
\begin{align}
    \vb{w} = \left.\dv{t} (f \circ c)\right|_{t=0}
\end{align}
である。このようにして、$T_pM$の元$\vb{v}$に対して$T_qN$の元$\vb{w}$が対応する。また、この対応は曲線の取り方によらないことが示せる。
これにより、$T_pM$の元$\vb{v}$に対して$T_qN$の元$\vb{w}$が対応する写像として、微分が定義される。

\begin{itembox}[l]{\textbf{Def.写像の微分}}
  上の対応で定まる写像
  \begin{align}
    (df)_p: T_pM \to T_{f(p)}N
  \end{align}
  を、$f: M \to N$の$p$における微分という。\footnote{以降、$f_*$と書くこともある。}
\end{itembox}
この写像に"微分"という名前がついていることを納得するために、以下の例を考えてみる。\\
\textbf{ex.}\\
$M = \mathbb{R}$, $N = \mathbb{R}$, $f(x) = x^2$とする。このとき、$p = 1$における$f$の微分$(df)_1$は、
hoge

写像の微分を成分表示する。
hogehoge

\begin{align}
  \begin{pmatrix}
    w_1 \\
    w_2 \\
    \vdots \\
    w_n
  \end{pmatrix}
  =
  \begin{pmatrix}
    \pdv{f_1}{x_1}(p) & \pdv{f_1}{x_2}(p) & \cdots & \pdv{f_1}{x_m}(p) \\
    \pdv{f_2}{x_1}(p) & \pdv{f_2}{x_2}(p) & \cdots & \pdv{f_2}{x_m}(p) \\
    \vdots & \vdots & \ddots & \vdots \\
    \pdv{f_n}{x_1}(p) & \pdv{f_n}{x_2}(p) & \cdots & \pdv{f_n}{x_m}(p)
  \end{pmatrix}
  \begin{pmatrix}
    v_1 \\
    v_2 \\
    \vdots \\
    v_m
  \end{pmatrix}
\end{align}




\end{document}