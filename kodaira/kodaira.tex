\documentclass[a4paper,11pt]{jsarticle}

% 数式
\usepackage{amsmath,amsfonts}
\usepackage{amsthm}
\usepackage{bm}
\usepackage{mathtools}
\usepackage{amssymb}

% 表
\usepackage[utf8]{inputenc}
\usepackage{diagbox} % 斜線付きセルを作成するために必要
\usepackage{booktabs} % 表の罫線を美しくするために必要
\usepackage{hhline} % 水平罫線を制御するために必要

% 画像
\usepackage[dvipdfmx]{graphicx}
\usepackage{ascmac}
\usepackage{physics}
\usepackage{float} % 追加

% 図
\usepackage[dvipdfmx]{graphicx}
\usepackage{tikz} %図を描く
\usetikzlibrary{positioning, intersections, calc, arrows.meta,math} %tikzのlibrary

% ハイパーリンク
\usepackage[dvipdfm,
  colorlinks=false,
  bookmarks=true,
  bookmarksnumbered=false,
  pdfborder={0 0 0},
  bookmarkstype=toc]{hyperref}

% 式番号を章ごとにリセット
\numberwithin{equation}{section}

\begin{document}

\title{解析}
\author{大上由人}
\date{\today}
\maketitle

小平解析ゼミの内容を可能な限り簡潔にまとめたい。

\subsection{数列の極限}
\begin{itembox}[l]{\textbf{Def.数列の極限}}
    数列$\{a_n\}$がある実数$\alpha$に収束するとは、任意の正の実数$\varepsilon$に対応して、ある自然数$n_0(\varepsilon)$が定まって
    \begin{align}
        n > n_0(\varepsilon) \Rightarrow |a_n - \alpha| < \varepsilon
    \end{align}
    が成り立つことである。また、このとき$\alpha$を数列$\{a_n\}$の極限といい、
    \begin{align}
        \lim_{n \to \infty} a_n = \alpha
    \end{align}
    と書く。
\end{itembox}

\begin{itembox}[l]{\textbf{Thm.}}
    数列$\{a_n\}$が実数$\alpha$に収束するための必要十分条件は、$\rho<\alpha<\sigma$なる実数$\rho,\sigma$が任意に与えられたとき、不等式
    \begin{align}
        \rho < a_n < \sigma \quad (n=1,2,3,\cdots)
    \end{align}
    が有限個の自然数$n$を除いて成立することである。
\end{itembox}
\textbf{Prf.}\\
$(\Rightarrow)$\\
$\{a_n\}$が$\alpha$に収束すると仮定する。$\rho<\alpha<\sigma$なる実数$\rho,\sigma$が任意に与えられたとする。$\min\{\alpha-\rho,\sigma-\alpha\} = \varepsilon$とおくと、
\begin{align}
    \rho \leq \alpha - \varepsilon \leq \sigma
\end{align}
が成り立つ。仮定により、$n > n_0(\varepsilon)$のとき、$|a_n - \alpha| < \varepsilon$である。すなわち、
\begin{align}
    \alpha - \varepsilon < a_n < \alpha + \varepsilon
\end{align}
が成り立つ。したがって、有限個の自然数$n$を除いて、
\begin{align}
    \rho \leq \alpha - \varepsilon < a_n < \alpha + \varepsilon \leq \sigma
\end{align}
が成り立つ。\\
$(\Leftarrow)$\\
正の実数$\varepsilon$が任意に与えられたとする。このとき、条件より、有限個の自然数$n$を除いて、
\begin{align}
    \alpha - \varepsilon < a_n < \alpha + \varepsilon
\end{align}
が成り立つ。\footnote{条件より、今みたいに都合よく$\rho,\sigma$を$\alpha-\varepsilon,\alpha+\varepsilon$と取れる。}したがって、この有限個の自然数のうち最大のものを$n_0(\varepsilon)$とおくと、
\begin{align}
    n > n_0(\varepsilon) \Rightarrow |a_n - \alpha| < \varepsilon
\end{align}
が成り立つ。したがって、$\{a_n\}$は$\alpha$に収束する。\qed

\begin{itembox}[l]{\textbf{Cor.極限の一意性}}
    数列$\{a_n\}$がある実数$\alpha$に収束するとき、その極限は一意的である。
\end{itembox}
\textbf{Prf.}\\
背理法により示す。$\{a_n\}$が異なる実数$\alpha$と$\beta$に収束すると仮定し、$\alpha < \beta$とする。このとき、有理数の稠密性から、$\alpha < r < \beta$なる有理数$r$が存在する。
このとき、前の定理より、有限個の自然数$n$を除いて、$a_n < r$が成り立つ。また、有限個の自然数$n$を除いて、$a_n > r$が成り立つ。これは矛盾である。したがって、$\alpha = \beta$である。\qed

\begin{itembox}[l]{\textbf{Thm.Cauthyの収束条件}}
    数列$\{a_n\}$が収束するための必要十分条件は、任意の正の実数$\varepsilon$に対応して一つの自然数$n_0(\varepsilon)$が定まって、
    \begin{align}
        m,n > n_0(\varepsilon) \Rightarrow |a_m - a_n| < \varepsilon
    \end{align}
    が成り立つことである。
\end{itembox}
\textbf{Prf.}\\
$(\Rightarrow)$\\
$\{a_n\}$が$\alpha$に収束すると仮定する。任意に与えられた正の実数$\varepsilon$に対して、$0 <a < \varepsilon$となるような有理数$a$をとる。(これが存在することは、有理数の稠密性から明らか。)
このとき、仮定より、
\begin{align}
    n > n_0(\varepsilon) \Rightarrow |a_n - \alpha| < \frac{a}{2}
\end{align}
なる自然数$n_0(\varepsilon)$が定まる。(有理数にしか割り算が定まっていないのでこのように抑えている。)このとき、$m,n > n_0(\varepsilon)$とすると、
\begin{align}
    |a_m - a_n| \leq |a_m - \alpha| + |\alpha - a_n| < \frac{a}{2} + \frac{a}{2} = a < \varepsilon
\end{align}
が成り立つ。\\
$(\Leftarrow)$\\
hoge(実数の連続性からわかる。)\qed



\end{document}