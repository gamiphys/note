\documentclass[a4paper,11pt]{jsarticle}

% 数式
\usepackage{amsmath,amsfonts}
\usepackage{amsthm}
\usepackage{bm}
\usepackage{mathtools}
\usepackage{amssymb}

% 表
\usepackage[utf8]{inputenc}
\usepackage{diagbox} % 斜線付きセルを作成するために必要
\usepackage{booktabs} % 表の罫線を美しくするために必要
\usepackage{hhline} % 水平罫線を制御するために必要

% 画像
\usepackage[dvipdfmx]{graphicx}
\usepackage{ascmac}
\usepackage{physics}
\usepackage{float} % 追加

% 図
\usepackage[dvipdfmx]{graphicx}
\usepackage{tikz} %図を描く
\usetikzlibrary{positioning, intersections, calc, arrows.meta,math} %tikzのlibrary

% ハイパーリンク
\usepackage[dvipdfm,
  colorlinks=false,
  bookmarks=true,
  bookmarksnumbered=false,
  pdfborder={0 0 0},
  bookmarkstype=toc]{hyperref}

% 式番号を章ごとにリセット
\numberwithin{equation}{section}

\begin{document}

\title{Majorization}
\author{大上由人}
\date{\today}
\maketitle

\section{古典的エントロピー及びダイバージェンス}
\subsection{古典的状態及び系}
\begin{itembox}[l]{\textbf{Def:状態を表す確率分布}}
    古典的系における状態は確率分布
    \begin{equation}
        p = (p_1, p_2, \cdots, p_d)^{\top}
    \end{equation}
    で表される。ここで、$p_i \geq 0$かつ$\sum_{i=1}^{d}p_i = 1$である。
    また、d次元の確率分布全体の集合を、$\mathcal{P}_d$と表す。\\
    また、その集合に属する一様分布を、
    \begin{equation}
        u = \left(\frac{1}{d}, \frac{1}{d}, \cdots, \frac{1}{d}\right)^{\top}
    \end{equation}
    と表す。\\
    また、異なる確率分布の積を、
    \begin{equation}
        p \otimes q \in \mathcal{P}_{dd'} \quad p \in \mathcal{P}_d, q \in \mathcal{P}_{d'}
    \end{equation}
    と表し、とくに、同じ確率分布の累乗を、
    \begin{equation}
        p^{\otimes n} \in \mathcal{P}_{d^n} \quad p \in \mathcal{P}_d
    \end{equation}
    と表す。
\end{itembox}

\begin{itembox}[l]{\textbf{Def:Supp}}
    確率分布$p ={p_i}_i \in \mathcal{P}_d$に対して、pの台を、
    \begin{equation}
        \text{spp}(p) = \{i \in [d] | p_i > 0\} \subset \{1, 2, \cdots, d\}
    \end{equation}
    と表す。また、
    \begin{equation}
        \text{rank}(p) = |\text{spp}(p)|
    \end{equation}
    を、pのランクという。とくに、$\text{rank}(p) = d$のとき、pはフルランクであるという。
\end{itembox}
要するに、確率が0でないようなインデックスの集合を台と呼び、その要素数をランクと呼ぶ。\\

\begin{itembox}[l]{\textbf{Def:確率遷移行列}}
    古典的な確率分布の時間発展は、確率遷移行列$T$を用いて以下のように表される。
    \begin{equation}
        p_i ' = \sum_{j=1} T_{ij}p_j 
    \end{equation}
\end{itembox}

\begin{itembox}[l]{\textbf{Prop:確率遷移行列の性質}}
    確率遷移行列$T$は以下の性質を持つ。
    \begin{equation}
        \sum_{i=1}^{d}T_{ij} = 1
    \end{equation}
\end{itembox}
\textbf{Prf}\\
%TODO後で書く

\begin{itembox}[l]{\textbf{Def:二重確率遷移行列}}
    確率遷移行列$T$が、
    \begin{equation}
        \sum_{j=1} T_{ij} = 1
    \end{equation}
    をみたすとき、二重確率遷移行列という。
\end{itembox}
\begin{itembox}[l]{\textbf{Prop:二重確率遷移行列の特徴づけ}}
    以下の二つは同値である。
    \begin{enumerate}
        \item Tは二重確率遷移行列である。
        \item 一様分布uはTに対して不変である。すなわち、$u=Tu$である。
    \end{enumerate}
\end{itembox}
\textbf{Prf}\\
\begin{align}
    p_i ' &= \sum_{j=1} T_{ij}u_j\\
    &= \frac{1}{d}\sum_{j=1}^{d}T_{ij}\\
    &= \frac{1}{d}\cdot d\\
    &= 1
\end{align}
であることからわかる。\\

\begin{itembox}[l]{\textbf{Def:トレース距離}}
    二つの確率分布$p, q$のトレース距離は、
    \begin{equation}
        D(p, q) = \frac{1}{2}\sum_{i=1}^{d}|p_i - q_i|
    \end{equation}
    で定義される。
\end{itembox}
\begin{itembox}[l]{\textbf{Prop:トレース距離の性質}}
    トレース距離は、$T$に対して非増加である。すなわち、
    \begin{equation}
        D(p,q) \geq D(Tp, Tq)
    \end{equation}
    が成り立つ。
\end{itembox}
\textbf{Prf}\\
後により一般の証明をするため、ここでは省略する。\\

\subsection{シャノンエントロピー及びKLダイバージェンス}
\begin{itembox}[l]{\textbf{Def:シャノンエントロピー}}
    確率分布$p \in \mathcal{P}_d$のシャノンエントロピーは、
    \begin{equation}
        S_1(p) = -\sum_{i=1}^{d}p_i\log p_i
    \end{equation}
    で定義される。
\end{itembox}
\begin{itembox}[l]{\textbf{Def:KLダイバージェンス}}
    二つの確率分布$p, q \in \mathcal{P}_d$のKLダイバージェンスは、
    \begin{equation}
        S_1(p||q) = \sum_{i=1}^{d}p_i\log\frac{p_i}{q_i}
    \end{equation}
    で定義される。ただし、$\text{supp}(p) \subset \text{supp}(q)$でないときは、$S_1(p||q) = \infty$とする。
\end{itembox}
このとき、エントロピーとKLダイバージェンスの関係がわかる。
\begin{itembox}[l]{\textbf{Prop:エントロピーとKLダイバージェンスの関係}}
    任意の$p, q \in \mathcal{P}_d$に対して、
    \begin{equation}
        S_1(p) = ln(d) - S_1(p||u)
    \end{equation}
\end{itembox}
\textbf{Prf}\\
\begin{align}
    S_1(p||u) &= \sum_{i=1}^{d}p_i\log\frac{p_i}{\frac{1}{d}}\\
    &= \sum_{i=1}^{d}p_i\log dp_i\\
    &= \sum_{i=1}^{d}p_i\log d + \sum_{i=1}^{d}p_i\log p_i\\
    &= \log d - S_1(p)
\end{align}
であることからわかる。\hfill \qedsymbol\\
これより、$S_1(p) \leq \log d$であることがわかる。\\
このとき、KLダイバージェンスのテイラー展開は以下のようになる。
\begin{align}
    S_1(p||p-\Delta p) &= \frac{1}{2}\sum_i \frac{(\Delta p_i)^2}{p_i} + O(\Delta p^3)\\
\end{align}
ここで、$\sum_i \Delta p_i = 0$を用いている。\\
%TODO後で示す。\\
\begin{itembox}[l]{\textbf{Prop:KLダイバージェンスの単調性}}
    KLダイバージェンスは、$p' = Tp$および$q' = Tq$に対して、
    \begin{equation}
        S_1(p||q) \geq S_1(p'||q')
    \end{equation}
    が成り立つ。
\end{itembox}
\textbf{Prf}\\
%後で示す。
注意されたいこととして、KLダイバージェンスの単調性の逆はいえない。すなわち、単調性を満たすが、$p' = Tp$および$q' = Tq$を満たすような$T$が存在しない場合がある。\\

次に、二重確率遷移行列について考える。このとき、KLダイバージェンスの単調性と、
\begin{equation}
    S_1(p) \leq S_1(Tp)
\end{equation}
が成り立つ。\\
すなわち、二重確率遷移行列による時間発展は、エントロピーを増加させる。\\

\begin{itembox}[l]{\textbf{Def:相互情報量}}
    二つの確率分布$p, q \in \mathcal{P}_d$の相互情報量は、
    \begin{equation}
        I_1 (p_{AB})_{A:B} = S_1 (p_A) + S_1 (p_B) - S_1 (p_{AB}) = S_1(p_{AB}||p_A \otimes p_B) \geq 0
    \end{equation}
    で定義される。
\end{itembox}
この量は、AとBの相関を表す量である。\\
\begin{itembox}[l]{\textbf{Prop:相互情報量の性質}}
    任意の$p, q \in \mathcal{P}_d$に対して、
    \begin{equation}
        I_1(p_{AB})_{A:B} =0 \Leftrightarrow p_{AB} = p_A \otimes p_B
    \end{equation}
    が成り立つ。また、KLダイバージェンスの単調性から、
    \begin{equation}
        I_1(p_{AB})_{A:B} \geq I_1(T_A \otimes T_B p_{AB})_{A:B}
    \end{equation}
    が成り立つ。ただし、$T_A \otimes T_B$は、各A,Bに独立に作用する確率遷移行列である。\\
    
\end{itembox}
\textbf{Prf}\\
%TODO後で示す。\\

\subsection{Rényiエントロピー及びダイバージェンス}
\begin{itembox}[l]{\textbf{Def:Rényiエントロピー}}
    確率分布$p \in \mathcal{P}_d$のRényiエントロピーは、$0 \leq \alpha \leq \infty$、$p \in \mathcal{P}_d$に対して、
    \begin{equation}
        S_{\alpha}(p) = \frac{1}{1-\alpha}\log(\sum_{i=1}^{d}p_i^{\alpha})
    \end{equation}
    で定義される。
\end{itembox}
\begin{itembox}[l]{\textbf{Def:Rényiダイバージェンス}}
    二つの確率分布$p, q \in \mathcal{P}_d$のRényiダイバージェンスは、$0 \leq \alpha \leq \infty$、$p \in \mathcal{P}_d$に対して、
    \begin{equation}
        S_{\alpha}(p||q) = \frac{1}{\alpha - 1}\log(\sum_{i=1}^{d}p_i^{\alpha}q_i^{1-\alpha})
    \end{equation}
    で定義される。ただし、$\text{supp}(p) \subset \text{supp}(q)$でないときは、$S_{\alpha}(p||q) = \infty$とする。
\end{itembox}
\begin{itembox}[l]{\textbf{Prop:RényiエントロピーとKLダイバージェンスの関係}}
    任意の$p, q \in \mathcal{P}_d$に対して、
    \begin{equation}
        S_{\alpha}(p) = \frac{1}{1-\alpha}\log d - S_{\alpha}(p||u)
    \end{equation}
    が成り立つ。
\end{itembox}

\begin{itembox}[l]{\textbf{Prop:Rényiダイバージェンスの非負性}}
    任意の$p, q \in \mathcal{P}_d$に対して、
    \begin{equation}
        S_{\alpha}(p||q) \geq 0
    \end{equation}
    が成り立つ。また、$0 < \alpha \leq  \infty$に対して、
    \begin{equation}
        S_{\alpha}(p||q) =0 \Leftrightarrow p = q
    \end{equation}
    であり、また、$\alpha =0$のとき、
    \begin{equation}
        S_{0}(p||q) = 0 \Leftrightarrow \text{supp}(p) \subset \text{supp}(q)
    \end{equation}
    が成り立つ。
\end{itembox}

\begin{itembox}[l]{\textbf{Prop:Rényiダイバージェンスの単調性}}
    任意の$p' = Tp, q' = Tq \in \mathcal{P}_d$に対して、
    \begin{equation}
        S_{\alpha}(p||q) \geq S_{\alpha}(p'||q')
    \end{equation}
    が成り立つ。また、二重確率遷移行列に対して、
    \begin{equation}
        S_{\alpha}(p) \leq S_{\alpha}(Tp)
    \end{equation}
    が成り立つ。
\end{itembox}

\begin{itembox}[l]{\textbf{Prop:Rényiダイバージェンスの単調性(2)}}
    $\alpha \leq \alpha' $に対して、
    \begin{equation}
        S_{\alpha}(p||q) \leq S_{\alpha'}(p||q)
    \end{equation}
    が成り立ち、また、
    \begin{equation}
        S_{\alpha}(p) \geq S_{\alpha'}(p)
    \end{equation}
    が成り立つ。
\end{itembox}

\begin{itembox}[l]{\textbf{Lem:}}
    fを下に凸な関数であるとし、$p,q,p',q' \in \mathbb{R}^d$がすべて正であるとする。
    もし、$p' = Tp, q' = Tq$であるとき、
    \begin{equation}
        \sum_{i=1}^{d} q_i'f\left(\frac{p_i'}{q_i'}\right) \leq \sum_{i=1}^{d} q_i f\left(\frac{p_i}{q_i}\right)
    \end{equation}
    が成り立つ。
\end{itembox}
\textbf{Prf}\\
Jensenの不等式より、
\begin{equation}
    \sum_{j=1}^{d}q_j'f\left(\frac{p_j'}{q_j'}\right) \leq \sum_{j=1}^{d}\sum_{i=1}^{d}\frac{T_{ji}q_i}{q_j'}f\left(\frac{p_i}{q_i}\right)=\sum_{i=1}^{d}q_i f\left(\frac{p_i}{q_i}\right)
\end{equation}
である。\hfill \qedsymbol\\
ここで、$p,q \in \mathcal{P}_d$として、$f(x)=xlogx$とすると、
\begin{equation}
    S_1 (p||q) \geq S_1 (p'||q')
\end{equation}
が成り立つ。これは、KLダイバージェンスの単調性を示している。\\

\textbf{Prf(Rényiダイバージェンスの非負性)}\\
$f_{\alpha}(x) = x^{\alpha}$とすると、$1<\alpha \leq \infty$に対して、$f_{\alpha}(x)$は下に凸な関数である。\\
Jensenの不等式より
\begin{equation}
    \sum_{i=1}^{d}q_i f(\frac{p_i}{q_i}) \geq f(\sum_{i=1}^{d}q_i \frac{p_i}{q_i}) = f(1) = 0
\end{equation}
である。%TODOここから先は後で書く。\\

\begin{itembox}[l]{\textbf{Def:f-ダイバージェンス}}
    $f(0, \infty) \to \mathbb{R}$を下に凸な関数とし、$x=1$で$f(x)$が狭義凸かつ$f(1)=0$であるとする。このとき、$p, q \in \mathcal{P}_d$に対して、
    \begin{equation}
        D_f(p||q) = \sum_{i=1}^{d}q_i f\left(\frac{p_i}{q_i}\right)
    \end{equation}
    で定義される。
\end{itembox}
\begin{itembox}[l]{\textbf{Prop:f-ダイバージェンスの非負性}}
    任意の$p, q \in \mathcal{P}_d$に対して、
    \begin{equation}
        D_f(p||q) \geq 0
    \end{equation}
    が成り立つ。また、
    \begin{equation}
        D_f(p||q) = 0 \Leftrightarrow p = q
    \end{equation}
    が成り立つ。
\end{itembox}
\textbf{Prf}\\
%TODO後で書く。\\

\begin{itembox}[l]{\textbf{Prop:f-ダイバージェンスの単調性}}
    任意の$p' = Tp, q' = Tq \in \mathcal{P}_d$に対して、
    \begin{equation}
        D_f(p||q) \geq D_f(p'||q')
    \end{equation}
    が成り立つ。
\end{itembox}
\textbf{Prf}\\
%TODO後で書く。\\


KLダイバージェンスは、$f(x) = x\log x$のときのf-ダイバージェンスである。しかし、Rényiダイバージェンスは、f-ダイバージェンスの形では表されない。\\
\subsection{フィッシャー情報量}
%TODO地の文を書く。
以下、我々は、なめらかにパラメータ化された確率分布$p(\theta)$について考える。ただし、$\theta$の取りうる領域は、$\mathbb{R}^m$の開部分集合である。\\
\begin{itembox}[l]{\textbf{Def:フィッシャー情報量}}
    $p(\theta) \in \mathcal{P}_d$がフルランクであるとし、$\theta \in \mathbb{R}^m$をパラメータとする。このとき、フィッシャー情報量は$m \times m$行列で、
    \begin{equation}
        J_{p(\theta),kl} = \sum_{i=1}^{d}p_i(\theta)\partial_k[ \log p_i(\theta)]\partial_l[ \log p_i(\theta)] = \sum_{i=1}^{d}\frac{\partial_k p_i(\theta)\partial_l p_i(\theta)}{p_i(\theta)}
    \end{equation}
    で定義される。
\end{itembox}
フィッシャー情報量は、f-ダイバージェンスの極限として得られる。%TODO後で書く。\\



\end{document}