\documentclass[a4paper,11pt]{jsarticle}

% 数式
\usepackage{amsmath,amsfonts}
\usepackage{bm}
\usepackage{mathtools}

% 表
\usepackage[utf8]{inputenc}
\usepackage{diagbox}
\usepackage{booktabs}
\usepackage{hhline}

% 画像
\usepackage[dvipdfmx]{graphicx}
\usepackage{ascmac}
\usepackage{physics}
\usepackage{float}

% 図
\usepackage[dvipdfmx]{graphicx}
\usepackage{tikz}
\usetikzlibrary{positioning, intersections, calc, arrows.meta,math}

% ハイパーリンク
\usepackage[dvipdfm,
  colorlinks=false,
  bookmarks=true,
  bookmarksnumbered=false,
  pdfborder={0 0 0},
  bookmarkstype=toc]{hyperref}

% その他
% \usepackage{circuitikz}
% \usepackage{caption}
% \usepackage{cancel}
% \usepackage{tensor}
% \usepackage{tikz}
% \usepackage{ascmac}
% \usepackage{float}
% \usepackage{hyperref}
% \usepackage{pxjahyper}
% \usepackage{tablefootnote}
% \usepackage[thicklines]{cancel}
\usepackage[version=4]{mhchem}
\usepackage{here}
\usepackage{pgfplots}

\begin{document}

\quad\\[35mm]
\centerline{\Huge{\textsf{物理学実験}}}
\quad\\[5mm]
\begin{table}[h]
  \centering
  \begin{tabular}{| c | c |}
    \hline
    \Huge\textsf{{題目}} & \Huge{\textsf{セナルモン法}} \rule[-5mm]{0mm}{15mm} \\
    \hline
  \end{tabular}
\end{table}
\quad\\[10mm]
\begin{table}[h]
  \centering
  \begin{tabular}{l l}
    \hline
    \LARGE{\textsf{氏\qquad 名}} & \LARGE{\textsf{:大上 由人}} \rule[0mm]{0mm}{6mm} \\
    \hline
    \LARGE{\textsf{学  籍  番  号}} & \LARGE{\textsf{: 02222100}} \rule[0mm]{0mm}{6mm} \\
    \LARGE{\textsf{学部学科学年}} & \LARGE{\textsf{: 理学部物理学科3年}}\\
    \hline
  \end{tabular}
\end{table}

\quad\\[10mm]
\centerline{\LARGE{\textsf{\today}}}\\[2mm]

\quad\\[10mm]
\thispagestyle{empty}
\clearpage
\addtocounter{page}{-1}
\newpage

\section{目的}


\section{原理}

\section{方法}
\subsection{フォトダイオード、偏光板、波長板の特性}
\begin{enumerate}
  \item フォトダイオードの線形領域で実験ができるよう、光強度vs信号電圧を測定し、適切なDNフィルタを選んだ。
  \item 机で反射した光を偏光板を通して見ることで、偏光板の向きを決定した。
  \item レーザー光に対して偏光板の牡蠣度を変えて光の強度がどのように変化するかを調べた。
  \item 偏光板を用いて、1/4波長板の光学軸を見つけた。
  \item 二つの偏光板の間に波長板を挿入して第一の偏光板をテーブルに垂直な偏光が通過するように配置した。
  \item 波長板や第二の偏光板を回転して、透過光の強度がどのように変化するかを調べ、理想曲線と比較した。
\end{enumerate}

\subsection{セナルモン法}
\begin{enumerate}
  \item 上の実験と同様に試料の光軸を調べ、光軸が水平になるときの回転ステージの角度を記録した。
  \item 45度試料を回転させてから固定した。
  \item Polarizerの偏光を水平にし、試料をセットした。
  \item 試料の下流側に1/4波長板の光学軸が水平になるようにセットした。
  \item 波長板から出てきた光の偏光方向を調べた。
\end{enumerate}

\section{結果}

\section{課題}

\section{考察}

\section{問題}

\begin{thebibliography}{99}
  \bibitem{ref} 参考文献
\end{thebibliography}

\end{document}