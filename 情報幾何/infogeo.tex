\documentclass[a4paper,11pt]{jsarticle}

% 数式
\usepackage{amsmath,amsfonts}
\usepackage{amsthm}
\usepackage{bm}
\usepackage{mathtools}
\usepackage{amssymb}

% 表
\usepackage[utf8]{inputenc}
\usepackage{diagbox} % 斜線付きセルを作成するために必要
\usepackage{booktabs} % 表の罫線を美しくするために必要
\usepackage{hhline} % 水平罫線を制御するために必要

% 画像
\usepackage[dvipdfmx]{graphicx}
\usepackage{ascmac}
\usepackage{physics}
\usepackage{float} % 追加

% 図
\usepackage[dvipdfmx]{graphicx}
\usepackage{tikz} %図を描く
\usetikzlibrary{positioning, intersections, calc, arrows.meta,math} %tikzのlibrary

% ハイパーリンク
\usepackage[dvipdfm,
  colorlinks=false,
  bookmarks=true,
  bookmarksnumbered=false,
  pdfborder={0 0 0},
  bookmarkstype=toc]{hyperref}

% 式番号を章ごとにリセット
\numberwithin{equation}{section}

\begin{document}

\title{情報幾何学の基礎}
\author{大上由人}
\date{\today}
\maketitle

\section{多様体のアフィン接続}
\subsection{ベクトル場の共変微分}
一般の多様体上では、共変微分を用いて平行移動を定義する。以下では初めに共変微分の定義を述べる。

\begin{itembox}[l]{\textbf{Def:共変微分}}
    Mを多様体とする。以下の4条件を満たす写像
    \begin{equation}
        \nabla : \mathfrak{X}(M) \times \mathfrak{X}(M) \to \mathfrak{X}(M)
    \end{equation}
    をM上の共変微分という。
    \begin{enumerate}
        \item 
        \begin{equation}
            \nabla_X(Y+Z) = \nabla_XY + \nabla_XZ
        \end{equation}

        \item
        \begin{equation}
            \nabla_X(fY) = (Xf)Y + f\nabla_XY
        \end{equation}

        \item
        \begin{equation}
            \nabla_{X+Y}Z = \nabla_XZ + \nabla_YZ
        \end{equation}

        \item
        \begin{equation}
            \nabla_{fX}Y = f\nabla_XY
        \end{equation}
    \end{enumerate}
\end{itembox}
\textbf{ex:Euclid空間}\\
%時間がある時に書く



\end{document}