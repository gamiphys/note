\documentclass[a4paper,11pt]{jsarticle}

% 数式
\usepackage{amsmath,amsfonts}
\usepackage{amsthm}
\usepackage{bm}
\usepackage{mathtools}
\usepackage{amssymb}

% 表
\usepackage[utf8]{inputenc}
\usepackage{diagbox} % 斜線付きセルを作成するために必要
\usepackage{booktabs} % 表の罫線を美しくするために必要
\usepackage{hhline} % 水平罫線を制御するために必要

% 画像
\usepackage[dvipdfmx]{graphicx}
\usepackage{ascmac}
\usepackage{physics}
\usepackage{float} % 追加

% 図
\usepackage[dvipdfmx]{graphicx}
\usepackage{tikz} %図を描く
\usetikzlibrary{positioning, intersections, calc, arrows.meta,math} %tikzのlibrary

% ハイパーリンク
\usepackage[dvipdfm,
  colorlinks=false,
  bookmarks=true,
  bookmarksnumbered=false,
  pdfborder={0 0 0},
  bookmarkstype=toc]{hyperref}

% 式番号を章ごとにリセット
\numberwithin{equation}{section}

\begin{document}

\title{幾何学B}
\author{大上由人}
\date{\today}
\maketitle

\section{接触空間}
\begin{itembox}[l]{\textbf{Def.接触空間}}
    $(X,A)$を位相空間対とし、$Y$を位相空間とする。また、$\phi: A \to Y$を連続写像とする。このとき、
    \begin{align}
        Y &\sqcup_{\phi} X := (X \sqcup Y) / \sim\\
        a & \sim \phi(a) \quad (\forall a \in A)
    \end{align}
    で定まる位相空間を、$X$と$Y$の接触空間という。
\end{itembox}

\begin{itembox}[l]{\textbf{Def.1点に潰した空間}}
    $(X,A)$を位相空間対とする。$A$上の点をすべて同一視する同値関係を$\sim$で定める。このとき、
    \begin{align}
        X/A \simeq \{p\}\sqcup_{\phi} X
    \end{align}
    を、$A$を一点に潰した空間という。
\end{itembox}

\begin{itembox}[l]{\textbf{Prop:}}
    \begin{align}
        X/A \simeq \{p\}\sqcup_{\phi} X
    \end{align}
    が成り立つ。ただし、$\phi: A \to \{p\}$は定値写像である。
\end{itembox}
\textbf{Prf.}\\
略\qed\\

\textbf{ex.}\\
$D^2/\partial D^2 \simeq \{p\}\sqcup_{\phi} D^2 \simeq S^2$\\
TODO: 図を挿入する。

\section{胞体複体}
$i$胞体$e^i$: $D^i \backslash \partial D^i = \text{Int}D^i$と同相なもの。\\
閉セル$\bar{e}^i$: $D^i$と同相なもの。\\
TODO: 図を挿入する。





\end{document}