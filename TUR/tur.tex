\documentclass[a4paper,11pt]{jsarticle}

% 数式
\usepackage{amsmath,amsfonts}
\usepackage{amsthm}
\usepackage{bm}
\usepackage{mathtools}
\usepackage{amssymb}

% 表
\usepackage[utf8]{inputenc}
\usepackage{diagbox} % 斜線付きセルを作成するために必要
\usepackage{booktabs} % 表の罫線を美しくするために必要
\usepackage{hhline} % 水平罫線を制御するために必要

% 画像
\usepackage[dvipdfmx]{graphicx}
\usepackage{ascmac}
\usepackage{physics}
\usepackage{float} % 追加

% 図
\usepackage[dvipdfmx]{graphicx}
\usepackage{tikz} %図を描く
\usetikzlibrary{positioning, intersections, calc, arrows.meta, math, matrix} %tikzのlibrary

% ハイパーリンク
\usepackage[dvipdfm,
  colorlinks=false,
  bookmarks=true,
  bookmarksnumbered=false,
  pdfborder={0 0 0},
  bookmarkstype=toc]{hyperref}

% 式番号を章ごとにリセット
\numberwithin{equation}{section}

\begin{document}

\title{TUR}
\author{大上由人}
\date{\today}
\maketitle

\section{TUR}
\subsection{Main Claim}
\textbf{記号}
\begin{itemize}
    \item $\hat{\mathcal{J}}_{ij} = \sum_{n} (\delta_{\omega_j \to \omega_i}(\omega^{n-1}\to \omega^n) - \delta_{\omega_i \to \omega_j}(\omega^{n-1}\to \omega^n))$\\
$\to (\text{jからiへのjumpが起こった回数})-(\text{iからjへのjumpが起こった回数})$(time-integrated empirical probability current)\\
    \item $\hat{\mathcal{J}}_{d} = \sum_{(i,j)} d_{ij} \hat{\mathcal{J}}_{ij}$
$\to$示量変数の積算カレント(cumulative thermodynamic current)
    \item $d_{ij} = X_i - X_j$(Xは任意の示量変数)
\end{itemize}

\textbf{ex.熱流}\\
このとき、積算熱流は、$X_i = E_i$とすることで、以下のように定義される:
\begin{align}
    \hat{\mathcal{J}}_{d} = \sum_{(i,j)} (E_i - E_j) \hat{\mathcal{J}}_{ij}
\end{align}

また、jump quantityとしては以下のように定義される\\
\begin{align}
    {J}_{ij}(t) &= R_{ij}p_j(t) - R_{ji}p_i(t)\\
    (\hat{J}_{d}(t))_{ij} &= d_{ij} {J}_{ij}(t)\\
    J_{d}(t) &= \ev{(\hat{J}_{d}(t)_{ij})} = \sum_{(i,j)} d_{ij} {J}_{ij}(t)
\end{align}

また、以下では、$\ev{\cdot}$で時間積分された量の平均も、jump quantityの平均も表すこととする。ただし、後者は、
\begin{align}
    \ev{X} = \sum_{i,j} X_{ij} R_{ij}p_j(t)
\end{align}
である。\\

このとき、以下の定理が成り立つことが知られている:

\begin{itembox}[l]{\textbf{Thm.熱力学的不確定性関係}}
    定常Markov過程が局所詳細釣り合い条件を満たすとき、以下の関係が成り立つことが知られている:
    \begin{align}
        \frac{\text{Var}(\hat{\mathcal{J}}_d)}{\ev{\hat{\mathcal{J}}_d^{\text{ss}}}^2} \sigma \geq 2
    \end{align}
    ただし、
    \begin{align}
        \text{Var}(\hat{\mathcal{J}}_d) &= \ev{\qty(\hat{\mathcal{J}}_d -\ev{\hat{\mathcal{J}_d}})}^2
    \end{align}
    である。\footnote{期待値は経路に対してとっている。}
\end{itembox}
すなわち、カレントの相対ゆらぎを小さくするためには、それ相応のコスト(エントロピー生成)が必要だということである。また、パワー$\ev{\hat{\mathcal{J}}_d}^2$を上げるためにも、エントロピー生成が必要であることがわかる。\\

\subsection{Cram\'er-Raoの不等式による証明}
Cram\'er-Raoの不等式を用いて、熱力学的不確定性関係を証明する。\footnote{\'eを初めてタイプした。}まず、準備として、Fisher情報量を以下のように定義する。
\begin{itembox}[l]{\textbf{Def:Fisher情報量}}
    パラメータ$\theta$についての確率分布$P_{\theta}(x)$が与えられたとき、Fisher情報量$F(\theta)$は以下のように定義される:
    \begin{align}
        F(\theta) = -\ev{\pdv[2]{\theta} \log P_{\theta}(x)}_{\theta} =\ev{\qty(\pdv{\theta} \log P_{\theta}(x))^2}_{\theta}
    \end{align}
\end{itembox}

$P_{\theta}(x)$をパラメータ$\theta$をもつ確率変数$x$の確率分布であるとする。また、引数$\theta$の関数$f(\theta)$が与えられていたとする。
ここでの目的は、$x$を$g(x)$として測定したとき、$f(\theta)$を推定することである。ここで、$g(x)$が$f(\theta)$の不偏推定量であると仮定する。すなわち、
\begin{align}
    f(\theta) = \ev{g(x)} 
\end{align}
が成り立つとする。
このとき、$g(x)$がどれほど正確な推定量なのかの限界を表すのがCram\'er-Raoの不等式である。\\
\begin{itembox}[l]{\textbf{Thm.一般化Cram\'er-Raoの不等式}}
    $g(x)$がパラメータ$f(\theta)$の不偏推定量であるとき、
    \begin{align}
        \text{Var}_{\theta}(g(x)) \geq \frac{(f'(\theta))^2}{F(\theta)}
    \end{align}
    が成り立つ。
\end{itembox}
\textbf{Prf.}\\
\begin{align}
    \text{Var}_{\theta}g(x)F(\theta) &= \qty(\int \dd{x} (g(x) - f(\theta))^2 P_{\theta}(x))\qty(\int \dd{x} \qty(\pdv{\theta} \log P_{\theta}(x))^2 P_{\theta}(x))\\
    &\geq \qty(\int \dd{x} (g(x) - f(\theta)) \pdv{\theta} (\ln P_{\theta}(x)) P_{\theta}(x))^2 \quad \because \text{Cauchy-Schwarz の不等式}\\
    &= \qty(\int \dd{x} g(x) \pdv{\theta} P_{\theta}(x) - \int \dd{x} f(\theta) \pdv{\theta} P_{\theta}(x))^2\\
    &= \qty(\pdv{\theta} \int \dd{x} g(x) P_{\theta}(x) )^2\quad \because \text{第二項の積分は規格化条件より0}\\
    &= \qty(\pdv{\theta} f(\theta))^2 \quad \because \text{不偏推定量の仮定より}\\
\end{align}
\qed

\begin{itembox}[l]{\textbf{Cor.Cram\'er-Raoの不等式}}
    $g(x)$がパラメータ$\theta$の不偏推定量であるとき、
    \begin{align}
        \text{Var}_{\theta}(g(x)) \geq \frac{1}{F(\theta)}
    \end{align}
    が成り立つ。

\end{itembox}
\textbf{Prf.}\\
$f(\theta) = \theta$とすればよい。\qed\\

また、Fisher情報量とKLダイバージェンスは以下のように関係している:\footnote{ここの計算は、Taylor展開などを用いればできる。手書きで補うかもしれない。}
\begin{align}
    D(P_{\theta + d\theta} \| P_{\theta})
    &= \int dx \, \pdv{}{\theta'} \left.\left( P_{\theta'}(x) \ln \frac{P_{\theta'}(x)}{P_{\theta}(x)} \right)\right|_{\theta' = \theta} d\theta \notag \\
    &\quad + \frac{1}{2} \int dx \, \frac{\partial^2}{\partial \theta'^2} \left.\left( P_{\theta'}(x) \ln \frac{P_{\theta'}(x)}{P_{\theta}(x)} \right)\right|_{\theta' = \theta} (d\theta)^2 + O(d\theta^3) \notag \\
    &= \frac{1}{2} \int dx \, \Bigg[ 2 \frac{1}{P_{\theta}(x)} \left( \pdv{}{\theta'} P_{\theta'}(x) \bigg|_{\theta' = \theta} \right)^2 \notag \\
    &\quad + P_{\theta}(x) \frac{\partial^2}{\partial \theta'^2} \ln P_{\theta'}(x) \bigg|_{\theta' = \theta} \Bigg] (d\theta)^2 + O(d\theta^3) \notag \\
    &= \frac{1}{2} \int dx \, P_{\theta}(x) \left( \pdv{}{\theta'} \ln P_{\theta'}(x) \bigg|_{\theta' = \theta} \right)^2 (d\theta)^2 + O(d\theta^3)\\
    &= \frac{1}{2} F(\theta) (d\theta)^2 + O(d\theta^3)
\end{align}
この表式を見ると、$\theta$を変化させたとき、KLダイバージェンスの意味での確率分布の距離がどれほど変化するかを表す量がFisher情報量であることがわかる。この文脈ででCram\'er-Raoの不等式を見ると、
$\theta$の変化による確率分布間の距離が大きければ大きいほど、推定量の分散は小さくなるということがわかる。直感的には、確率分布同士が似ているよりも、はっきり異なるほうが推定がしやすい(見分けがつきやすい)
ということである。具体例は後に見ることにする。\\

\noindent
\textbf{Prf(熱力学的不確定性関係)}\\
パラメータ$\theta$つきの遷移レートを以下のように定義する。
\begin{align}
    R_{ij}^{\theta} &= R_{ij}e^{\theta Z_{ij}} \quad (i \neq j)\\
    R_{jj}^{\theta} &= -\sum_{i (\neq j)} R_{ij}^{\theta}e^{\theta Z_{ij}}\\
\end{align}
ただし、
\begin{align}
    Z_{ij} = \frac{R_{ij}p_j^\text{ss} - R_{ji}p_i^\text{ss}}{R_{ij}p_j^\text{ss} + R_{ji}p_i^\text{ss}}
\end{align}
である。この遷移レートで特徴づけられるような経路の確率分布を$P_{\theta}(\Gamma)$とする。\\
一般化Cram\'er-Raoの不等式
\begin{align}
    \text{Var}_{\theta}(g(x)) \geq \frac{(f'(\theta))^2}{F(\theta)}
\end{align}
において、
$x = \Gamma,g(\Gamma) = \hat{\mathcal{J}}_{d}, f(\theta) = \ev{g(\Gamma)}^{\text{ss}}_{\theta} = \ev{\hat{\mathcal{J}}_{d}}^{\text{ss}}_{\theta}$としたとき
の$\theta = 0$の場合を使う。すなわち、
\begin{align}
\eval{\left( \frac{\partial \langle {\hat{\mathcal{J}}_{d}} \rangle_\theta^{\text{ss}}}{\partial \theta} \right)^2}_{\theta = 0}
\leq \mathrm{Var}_{0} (\hat{\mathcal{J}}_{d}) F(0) \label{eq:1}
\end{align}
を用いる。\\


\textbf{Fisher情報量について}\\
\begin{align}
    F(0) &= - \int_0^\tau dt 
    \left[
        \sum_{i \neq j} R_{ij} p_j(t) 
        \eval{\frac{\partial^2}{\partial \theta^2} \ln R_{ij}^\theta}_{\theta = 0}
        + \sum_j p_j(t) 
        \eval{\frac{\partial^2}{\partial \theta^2} \ln e^{R_{jj}^\theta}}_{\theta = 0}
    \right] \nonumber \\
    &= \int_0^\tau dt \sum_j p_j(t) 
    \eval{\frac{\partial^2}{\partial \theta^2} 
        \sum_{k (\neq j)} R_{kj} e^{\theta Z_{kj}}}_{\theta = 0} \\
        &\quad \because \text{第一項は微分で消える。第二項について、}R_{jj} = -\sum_{k (\neq j)} R_{kj}  \\
    &= \int_0^\tau dt \sum_{k \neq j} R_{kj} p_j(t) Z_{kj}^2 \quad \because \text{微分を実行} \label{eq:ff}\\
    &= \left\langle \int_0^\tau dt \, \hat{Z}^2 \right\rangle
\end{align}
と計算できる。\footnote{最初の等号は後の方に書いてある。}

また、
\begin{align}
    &\ev{\hat{Z}^2}^{\text{ss}}\\
    =& \frac{1}{2} \sum_{i,j} (Z_{ij}^2 p_j^\text{ss} R_{ij} + Z_{ji}^2 p_i^\text{ss} R_{ji})\\
    =& \frac{1}{2} \sum_{i,j} Z_{ij}^2(R_{ij} p_j^\text{ss} + R_{ji} p_i^\text{ss})\\
    =& \frac{1}{2} \sum_{i,j} \frac{(R_{ij} p_j^\text{ss} - R_{ji} p_i^\text{ss})^2}{R_{ij} p_j^\text{ss} + R_{ji} p_i^\text{ss}}\\
    \leq& \frac{1}{2}\qty(\frac{1}{2} \sum_{i,j} (R_{ij} p_j^\text{ss} - R_{ji} p_i^\text{ss})\ln \frac{R_{ij} p_j^\text{ss}}{R_{ji} p_i^\text{ss}}) \quad \because \text{16章で示した不等式}\\
    \leq& \frac{\dot{\sigma}}{2} 
\end{align}
が得られる。

\textbf{$f$について}\\
はじめの状態を、定常状態に取る。(定理の仮定より)このとき、$R_{ij} ^{\theta} = R_{ij}(1+\theta Z_{ij}) + O(\theta^2)$であることを用いて、
\begin{align}
    &\sum_{i}R_{ij}^{\theta} p_j^\text{ss} - R_{ji}^{\theta} p_i^\text{ss} \\
    =& \sum_{i}R_{ij} p_j^\text{ss} - R_{ji} p_i^\text{ss} + \theta (R_{ij} p_j^\text{ss} Z_{ij} - R_{ji} p_i^\text{ss} Z_{ji}) + O(\theta^2)\\
    =& \sum_{i}R_{ij} p_j^\text{ss} - R_{ji} p_i^\text{ss} + \theta Z_{ij}(R_{ij} p_j^\text{ss} + R_{ji} p_i^\text{ss}) + O(\theta^2)\\
    =& \sum_{i}R_{ij} p_j^\text{ss} - R_{ji} p_i^\text{ss} + \theta  (R_{ij} p_j^\text{ss} - R_{ji} p_i^\text{ss}) + O(\theta^2)\\
    =& O(\theta^2) \quad \because \text{定常分布}
\end{align}
となる。したがって、$\theta$系の定常状態は、$O(\theta)$までの近似で、$\theta = 0$の定常状態と一致する。このときの定常カレントは、
\begin{align}
   (\hat{J}_{ij})_\theta^{\text{ss}}:= R_{ij}^\theta p_j^\text{ss} - R_{ji}^\theta p_i^\text{ss} = O(\theta^2)
\end{align}
と書くことができる。したがって、
\begin{align}
    &\eval{\pdv{({{{J}}_{ij}})^{\text{ss}}_\theta}{\theta}}_{\theta = 0} \\
    =& \eval{\pdv{\theta} (R_{ij}^\theta p_j^\text{ss} - R_{ji}^\theta p_i^\text{ss})}_{\theta = 0} \\
    =&  R_{ij} Z_{ij} p_j^\text{ss} - R_{ji} Z_{ji} p_i^\text{ss} \\
    =&  Z_{ij} (R_{ij} p_j^\text{ss} + R_{ji} p_i^\text{ss}) \\
    =&  R_{ij}p_j^\text{ss} - R_{ji}p_i^\text{ss} \\
    =& {J}_{ij}^\text{ss}
\end{align}
となる。したがって、
\begin{align}
    \eval{\pdv{({{J}_{ij}})^{\text{ss}}_\theta}{\theta}}_{\theta = 0} = {J}_{ij}^{\text{ss}}
\end{align}
である。両辺$d_{ij}$をかけて和を取ることで、
\begin{align}
    \left.\sum_{(i,j)} \pdv{\theta} ({d_{ij}{J}_{ij}})^{\text{ss}}_\theta \right|_{\theta = 0} = \sum_{(i,j)} d_{ij} {J}_{ij}^{\text{ss}}
\end{align}
が成り立つ。すなわち、
\begin{align}
    \left.\pdv{\theta} \ev{\hat{{J}}_{d}}^{\text{ss}}_\theta\right|_{\theta = 0} = \ev{\hat{{J}}_{d}}^{\text{ss}}_{0}
\end{align}
が成り立つ。両辺時間について積分することにより、
\begin{align}
    \left.\pdv{\theta} \ev{\hat{{\mathcal{J}}_{d}}}^{\text{ss}}_\theta\right|_{\theta = 0} = \ev{\hat{{\mathcal{J}}_{d}}}^{\text{ss}}_{0}
\end{align}
が成り立つ。\footnote{jump quantityの期待値の時間積分は対応するpath quantityの期待値の積分に対応する。}


以上の関係式を用いて、式(\ref{eq:1})を評価すると、
\begin{align}
    \qty(\ev{\hat{\mathcal{J}}_{d}}^{\text{ss}}_{0})^2\leq \eval{\left( \frac{\partial \langle {\hat{\mathcal{J}}_{d}} \rangle_\theta^{\text{ss}})^2}{\partial \theta} \right)^2}_{\theta = 0}
    \leq \mathrm{Var}_{0} (\hat{\mathcal{J}}_{d}) \int_0^\tau dt \, \langle \hat{Z}^2 \rangle \leq \frac{{\sigma}}{2} \mathrm{Var}_0 (\hat{\mathcal{J}}_{d})
\end{align}
が得られる。したがって、
\begin{align}
    \frac{\mathrm{Var}(\hat{\mathcal{J}}_{d})}{\qty(\ev{\hat{\mathcal{J}}_{d}}^{\text{ss}})^2} \sigma \geq 2
\end{align}
が成り立つ。\qed\\

\textbf{補足:$F(0)$の表式}\\
経路にわたっての期待値を計算するのだが、一旦時間を離散化したほうがわかりやすいので、離散化してマルコフ連鎖として考える。このとき、経路$\Gamma$の実現確率は
\begin{align}
    P_\theta(\Gamma) = \prod_{n=1}^N T^\theta_{w^n w^{n-1}} p^0_{w^0}
\end{align}
と書かれる。このもとで、m-step目における確率分布は、
\begin{align}
    p^m_{w_m} = \sum_{w_0, \dots, w_{m-1}} \prod_{n=1}^m T^{w^n w^{n-1}} p^0_{w^0}
\end{align}
と書かれる。これを下に、Fisher情報量を計算する。
\begin{align}
    \int d\Gamma \, P(\Gamma) \frac{\partial^2}{\partial \theta^2} \ln P_\theta(\Gamma) 
    &= \sum_{w_0, \dots, w_N} \prod_{n=1}^N T^{w^n w^{n-1}} p^0_{w^0} 
    \frac{\partial^2}{\partial \theta^2} 
    \left(
        \sum_{n=1}^N \ln T^\theta_{w^n w^{n-1}} + \ln p^0_{w^0}
    \right) \\
    &= \sum_{w_0, \dots, w_N} \prod_{n=1}^N T^{w^n w^{n-1}} p^0_{w^0} 
    \frac{\partial^2}{\partial \theta^2} 
    \left(
        \sum_{n=1}^N \ln T^\theta_{w^n w^{n-1}}
    \right) \\
\end{align}
一番最後の$\Sigma$について考える。和の$k$番目の項について、
\begin{align}
    (\text{第k項}) &= \sum_{w_0, \dots, w_N} \prod_{n=1}^N T^{w^n w^{n-1}} p^0_{w^0} \frac{\partial^2}{\partial \theta^2} \ln T^\theta_{w^k w^{k-1}} \\
    &(w^{0}\text{から}w^{k-2}\text{まで計算して、}) \nonumber \\
    &= \sum_{w_{k-1}, w_k,\cdots, w_N} \prod_{n=k-1}^N T^{w^n w^{n-1}} p^{k-1}_{w^{k-1}}\frac{\partial^2}{\partial \theta^2} \ln T^\theta_{w^k w^{k-1}} \\
    &(w^{k+1}\text{から}w^{N}\text{まで計算すると、規格化条件より}) \nonumber \\
    &= \sum_{w_{k-1}, w_k} T^{w^k w^{k-1}} p^{k-1}_{w^{k-1}}\frac{\partial^2}{\partial \theta^2} \ln T^\theta_{w^k w^{k-1}} 
\end{align}
となる。したがって、
\begin{align}
    F(\theta) = -\sum_{n=1}^N \sum_{w_{n-1}, w_n} T^{w^n w^{n-1}} p^{n-1}_{w^{n-1}} \pdv[2]{\theta} \ln T^\theta_{w^n w^{n-1}}
\end{align}
となる。これを連続極限に持っていくと、$w^n = w^{n-1}$の場合と$w^n \neq w^{n-1}$の場合に分けて、
\begin{align}
    F(\theta) = -\int \dd{t} \qty[\sum_{i \neq j} R_{ij} p_j(t) \pdv[2]{\theta} \ln R_{ij}^\theta + \sum_j p_j(t) \pdv[2]{\theta} \ln e^{R_{jj}^\theta}]
\end{align}
となる。ただし、第二項については、$T_{ii} = 1-\sum_{j(\neq i)} R_{ij}\Delta t =1 + R_{ii} \Delta t$であることを用いて、
\begin{align}
    (1+R_{ii}\Delta t )\ln (1+R_{ii}\Delta t) &= (1+R_{ii}\Delta t )\ln \exp(R_{ii}\Delta t) + O(\Delta t^2)\\
    &= (1+R_{ii}\Delta t )R_{ii} \Delta t + O(\Delta t^2)\\
    &= R_{ii} \Delta t + O(\Delta t^2)\\
    &= \ln e^{R_{ii}\Delta t} + O(\Delta t^2)
\end{align}
となることを用いている。

\section{TUR-Type Inequalities}
\subsection{TURの一般化}
\subsubsection{緩和過程}
先程は、初期状態も含めて常に系の状態が定常状態であると仮定した。ここでは、遷移レートが固定されていて(すなわち、外部操作がない状況で)系の状態が平衡状態とは限らないときに成り立つTURを考える。
すなわち、緩和過程におけるTURである。\\

\begin{itembox}[l]{\textbf{Thm.緩和過程におけるTUR}}
    連続時間$(0 \leq t \leq \tau)$におけるMarkov過程について考える。ただし、遷移レートは固定されているとする。また、初期状態が定常状態であることは要求しない。このとき、
    \begin{align}
        \frac{\text{Var}(\hat{\mathcal{J}}_d)}{(\tau J_{d}(\tau))^2} \sigma \geq 2
    \end{align}
    が成り立つ。ただし、$J_{d}(\tau)$は、時刻$t=\tau$における平均のカレントである。
\end{itembox}
\textbf{Prf.}\\
\begin{align}
    Z_{ij} &= \frac{R_{ij}p_j(t) - R_{ji}p_i(t)}{R_{ij}p_j(t) + R_{ji}p_i(t)}\\
\end{align}
とおく。(前の定理と同じような形だが、定常状態でない。)初期状態を、$\theta$とは独立に$\vb{p}(0)$ととる。このとき、前の定理と同様に、
\begin{align}
    \int_0^\tau dt \langle \hat{Z}(t)^2 \rangle \leq \frac{\sigma}{2}
\end{align}
\begin{align}
    Z_{ij}(t) R_{ij} \vb{p}_j(t) - Z_{ji}(t) R_{ji} \vb{p}_i(t) 
    &= R_{ij} \vb{p}_j(t) - R_{ji} \vb{p}_i(t)
\end{align}
\begin{align}
    R^\theta(t) \vb{p}(t) &= (1 + \theta) R \vb{p}(t) + O(\theta^2)
\end{align}
となる。3つ目の式を観察すると、$R^{\theta}$によって時間発展する確率分布は、
\begin{align}
    \vb{p}'(t) = \vb{p}((1+\theta)t) + O(\theta^2)
\end{align}
となるのではないかと考えられる。すなわち、タイムスケールを$(1+\theta)$倍にしたものになるということである。これを確かめるには、$\vb{p}'(t)$が$R^{\theta}$によって時間発展するかを確かめればよい。実際、
\begin{align}
    \frac{d}{dt} \vb{p}'(t) 
    &= (1 + \theta) \eval{\frac{d}{dt'} \vb{p}(t')}_{t' = (1 + \theta)t} \\
    &= (1 + \theta) \eval{\frac{d}{dt'} \vb{p}(t')}_{t' = t} 
    + \eval{\frac{d}{dt''} \left( \frac{d}{dt'} \vb{p}(t') \right)}_{t' = t''} \cdot \theta t + O(\theta^2) \\
    & \because t'\text{について展開して、}(1+\theta)\cdot\theta t = \theta t + O(\theta^2) \nonumber \\
    &= R^\theta(t) \vb{p}(t) + \eval{\frac{d}{dt'} \left( R \vb{p}(t') \right)}_{t} \cdot \theta t + O(\theta^2) \quad \because \text{master eq.} \\
    &= R^\theta(t) \vb{p}((1 + \theta)t) + O(\theta^2) \quad \because R-R^\theta = O(\theta)\\
    &= R^\theta(t) \vb{p}'(t) + O(\theta^2)
\end{align}
したがって、$\vb{p}'(t)$は、$R^\theta$によって時間発展することがわかり、$R^\theta$によって時間発展する確率分布は、もとの系のタイムスケールを$(1+\theta)$倍にしたものであることがわかる。\\
これを踏まえると、
\begin{align}
    \langle \hat{\mathcal{J}_d} \rangle_\theta - \langle \hat{\mathcal{J}_d} \rangle_0 
    &= \int_0^{(1+\theta)\tau} dt J_d(t) - \int_0^\tau dt J_d(t) \nonumber =\theta \tau J_d(\tau) + O(\theta^2).
\end{align}
である。以上のことと、一般化Cram\'er-Raoの不等式
\begin{align}
    \text{Var}_{\theta}(g(x))F(\theta) \geq (f'(\theta))^2
\end{align}
で$\theta = 0$の場合を考えると、
\begin{align}
    (\tau J_d(\tau))^2 \leq \text{Var}_0(\mathcal{J}_d) \cdot \frac{\sigma}{2}
\end{align}
が得られる。したがって、
\begin{align}
    \frac{\text{Var}(\mathcal{J}_d)}{(\tau J_d(\tau))^2} \sigma \geq 2
\end{align}
が成り立つ。\qed

\textbf{別証明と一般化}\\
このあと操作パラメータにより遷移レートが時間変化する場合への一般化を考えるが、白石本の流れで進めてもこの一般化がうまくいかなかったので、上の定理の一部を別証明に差し替え、それをそのまま利用する。具体的には、$\langle \hat{\mathcal{J}_d} \rangle_\theta$の部分を別のものに差し替える。\\

\begin{align}
    \langle \hat{\mathcal{J}_d} \rangle_\theta &=\int_0^\tau \dd{s} \sum_{i \neq j} R_{ij}^{\theta}d_{ij}p_j^{\theta}(s)\\
\end{align}
について考える。$\theta$確率分布は、
\begin{align}
    p_{i}^{\theta}(t) = p_{i}(t) + \theta \phi_{i}(t) + O(\theta^2)
\end{align}
と展開することができる。このとき、$\theta$確率分布に対するマスター方程式は、
\begin{align}
    \pdv{p_{i}^{\theta}(t)}{t} = \sum_{j} R_{ij}^{\theta}p_{j}^{\theta}(t) 
\end{align}
となる。これより、
\begin{align}
    \pdv{p_{i}}{t} + \theta \pdv{\phi_{i}}{t} &=\sum_{j(\neq i)} R_{ij}(1 + \theta Z_{ij})p_{j}^{\theta} - \sum_{j(\neq i)} R_{ji}(1 + \theta Z_{ji})p_{i}^{\theta} \\
    &= \sum_{j(\neq i)} (R_{ij}(1 + \theta Z_{ij})(p_{j} + \theta \phi_{j}) - R_{ji}(1 + \theta Z_{ji})(p_{i} + \theta \phi_{i}))\\
    &= \sum_{j(\neq i)} ((R_{ij}p_{j} - R_{ji}p_{i}) + \theta(R_{ij}\phi_{j}-R_{ji}\phi_{i})+ \theta(R_{ij}Z_{ij}p_{j} - R_{ji}Z_{ji}p_{i}))\\
    &= \sum_{j} R_{ij}p_{j} + \theta \sum_{j} R_{ij}\phi_{j} + \theta \sum_{j(\neq i)} Z_{ij}(R_{ij}p_{j} + R_{ji}p_{i})\\
    &= \sum_{j} R_{ij}p_{j} + \theta \sum_{j} R_{ij}\phi_{j} + \theta \sum_{j(\neq i)} (R_{ij}p_{j} - R_{ji}p_{i})\\
    &= \sum_{j} R_{ij}p_{j} + \theta \sum_{j} R_{ij}\phi_{j} + \theta \sum_{j} (R_{ij}p_{j})
\end{align}
となる。したがって、
\begin{align}
    \pdv{p_{i}}{t} &= \sum{j} R_{ij}p_{j}\\
    \pdv{\phi_{i}}{t} &= \dv{p_{i}}{t} + \sum_{j} R_{ij}\phi_{j}
\end{align}
となる。このとき、$\phi$は、
\begin{align}
    \phi_{i}(t) = t\pdv{p_{i}}{t} 
\end{align}
と書くことができる。これをもとに、$f(\theta)$を微分することを考える。
\begin{align}
    \pdv{f(\theta)}{\theta} &= \int_0^\tau \dd{s} \sum_{i \neq j} \pdv{R_{ij}^{\theta}}{\theta}d_{ij}p_j^{\theta}(s) + \int_0^\tau \dd{s} \sum_{i \neq j} R_{ij}^{\theta}d_{ij}\pdv{p_j^{\theta}(s)}{\theta}\\
    &= \int_0^\tau \dd{s} \sum_{i \neq j}Z_{ij} R_{ij}e^{\theta Z_{ij}}d_{ij}p_j^{\theta}(s) + \int_0^\tau \dd{s} \sum_{i \neq j} R_{ij}^{\theta}d_{ij}\phi_j(s)\\
    &\underset{\theta \to 0}{\longrightarrow} \int_0^\tau \dd{s} \sum_{i \neq j} Z_{ij}R_{ij}d_{ij}p_j(s) + \int_0^\tau \dd{s} \sum_{i \neq j} R_{ij}d_{ij}\phi_j(s)
\end{align}
となる。第一項について、
\begin{align}
    \int_0^\tau \dd{s} \sum_{i \neq j} Z_{ij}R_{ij}d_{ij}p_j(s) 
    &= \frac{1}{2} \int_0^\tau \dd{s} \sum_{i \neq j} (Z_{ij}R_{ij}d_{ij}p_j(s) + Z_{ji}R_{ji}d_{ji}p_i(s))\\
    &= \frac{1}{2} \int_0^\tau \dd{s} \sum_{i \neq j} d_{ij}Z{ij}(R_{ij}p_j(s) + R_{ji}p_i(s))\\
    &= \frac{1}{2} \int_0^\tau \dd{s} \sum_{i \neq j} d_{ij}(R_{ij}p_j(s) - R_{ji}p_i(s))\\
    &= \int_{0}^{\tau} \dd{s} \sum_{i \neq j} R_{ij}d_{ij}p_j(s)\\
    &= \langle \hat{\mathcal{J}_d} \rangle_0
\end{align}
となる。また、第二項について、
\begin{align}
    \int_0^\tau \dd{s} \sum_{i \neq j} R_{ij}d_{ij}\phi_j(s) 
    &= \int_0^\tau \dd{s} \sum_{i \neq j} R_{ij}d_{ij}s\pdv{p_j(s)}{s} \\
    &= \left[\sum_{i \neq j}sR_{ij}d_{ij}p_j(s)\right]_0^\tau - \int_{0}^{\tau} \dd{s} \sum_{i \neq j} R_{ij}d_{ij}p_j(s) - \int_{0}^{\tau} \dd{s} \sum_{i \neq j} \pdv{R_{ij}}{s}d_{ij}p_j(s)\\
    &= \tau J(\tau)- \langle \hat{\mathcal{J}_d} \rangle_0 - \int_{0}^{\tau} \dd{s} \sum_{i \neq j} \pdv{R_{ij}}{s}d_{ij}p_j(s)\\
\end{align}
となる。最後の項については、
\begin{align}
    s\pdv{s}R_{ij}(vs) &= \pdv{v}(vs)\pdv{R_{ij}}{s}\\
    &= v\pdv{v}(vs)\pdv{R_{ij}}{(vs)}\\
    &= v\pdv{v}R_{ij}
\end{align}
を用いて、
\begin{align}
    \int_{0}^{\tau} \dd{s} \sum_{i \neq j} \pdv{R_{ij}}{s}d_{ij}p_j(s) 
    &= \int_{0}^{\tau} \dd{s} \sum_{i \neq j} v\pdv{v}R_{ij}d_{ij}p_j(s)\\
    &= v\pdv{v}\langle \hat{\mathcal{J}_d} \rangle_0
\end{align}
となる。以上より、
\begin{align}
    \pdv{\langle \hat{\mathcal{J}_d} \rangle_\theta}{\theta} \underset{\theta \to 0}{\longrightarrow}  \tau J(\tau)- v\pdv{v}\langle \hat{\mathcal{J}_d} \rangle_0
\end{align}
となる。残りは前の証明と同様。\\


\subsubsection{Markov chain}



\textbf{記号}
\begin{itemize}
    \item $\mathcal{J}_d(\Gamma) = \sum_{n=0}^{N-1} d_{\omega_{n+1}\omega_n}$:積算カレント
    \item $J_d^n = \sum_{i,j} d_{ij} T_{ij} p_{j}^{n-1}$:カレント
    \item $\sigma = \sum_{n=1}^{N} \sum_{i,j} T_{ij} p_{j}^{n-1} \ln \frac{T_{ij} p_{j}^{n-1}}{T_{ji} p_{i}^{n-1}}$エントロピー生成
\end{itemize}

\begin{itembox}[l]{\textbf{Thm.Markov連鎖についてのTUR}}
    離散時間離散状態のMarkov連鎖について考える。step数は$N$とし、遷移行列$T$は固定されているとする。また、初期分布は任意に取る。このとき、
    任意のカレント$J_{d}$について、
    \begin{align}
        \frac{\text{Var}(\hat{J}_{d})}{(NJ_{d}^{N-2})^2} \frac{\tilde{\sigma}}{a} \geq 2    
    \end{align}
    が成り立つ。ただし、modifed entropy production$\tilde{\sigma}$は、
    \begin{align}
        \tilde{\sigma} = \sigma + \sum_{n=1}^{N} D(p^n \| p^{n-1})
    \end{align}
    で定義され、
    \begin{align}
        a = \min_{i} T_{ii}
    \end{align}
    である。
\end{itembox}
\textbf{Prf.}\\
$\theta-$遷移行列を以下のように定義する。
\begin{align}
    T_{ij}^{n,\theta} &= T_{ij}e^{\theta Z_{ij}^n} \quad (i \neq j)\\
    T_{ii}^{n,\theta} &= 1 - \sum_{i(\neq j)} T_{ij}^{n,\theta}
\end{align}
また、
\begin{align}
    K_{ij}^n &= T_{ij}p_j^{n-1} 
\end{align}
\begin{align}
    Z_{ij}^n &= \frac{K_{ij}^n - K_{ji}^n}{K_{ij}^n + K_{ji}^n} \quad (i \neq j)\\
    Z_{jj}^n &=\left.\pdv{\theta}\ln{T_{jj}^{n,\theta}}\right|_{\theta = 0} = -\frac{1}{T_{jj}}\sum_{i(\neq j)}T_{ij} Z_{ij}^n
\end{align}

Cram\'er-Raoの不等式を用いる。すなわち、
\begin{align}
    \eval{\left( \frac{\partial \langle {\hat{\mathcal{J}}_{d}} \rangle_\theta^{\text{ss}}}{\partial \theta} \right)^2}_{\theta = 0}
    \leq \mathrm{Var}_{0} (\hat{\mathcal{J}}_{d}) F(0) \label{eq:2}
    \end{align}
    を用いる。

\textbf{Fisher情報量について}\\
\begin{align}
    F(0) &= -\sum_{n=1}^{N} \sum_{i,j} T_{ij} p_j^{n-1} \eval{\frac{\partial^2}{\partial \theta^2} \ln T_{ij}^{n,\theta}}_{\theta=0} \\
    &= -\sum_{n=1}^{N} \sum_{j} T_{jj} p_j^{n-1} \eval{\frac{\partial^2}{\partial \theta^2} \ln T_{jj}^{n,\theta}}_{\theta=0} \qty(\because \pdv[2]{\theta}\ln{T_{ij}^{n,\theta}} = \pdv[2]{\theta} \ln T_{ij} + \pdv[2]{\theta} (\theta Z_{ij}^n) = 0)\\
    &= -\sum_{n=1}^{N} \sum_{j} p_j^{n-1} \left[
        \eval{\frac{\partial^2}{\partial \theta^2} T_{jj}^{n,\theta}}_{\theta=0}
        - T_{jj} p_j^{n-1} \left( \eval{\frac{1}{T_{jj}} \frac{\partial}{\partial \theta} T_{jj}^{n,\theta}}_{\theta=0} \right)^2
    \right]\\
    &\qty(\because \pdv[2]{\theta} \ln T_{jj}^{n,\theta} = \frac{1}{T_{jj}} \pdv[2]{\theta} T_{jj}^{n,\theta} - \qty(\frac{1}{T_{jj}} \pdv{\theta} T_{jj}^{n,\theta})^2)\\
\end{align}
となることを用いている。ここで、第一項について、
\begin{align}
    \pdv[2]{\theta}(1-\sum_{i(\neq j)}T_{ij}^\theta) &= -\pdv[2]{\theta}\sum_{i(\neq j)}T_{ij}e^{\theta Z_{ij}} = -\sum_{i(\neq j)}(Z_{ij})^2 T_{ij} 
\end{align}
となる。また、第二項について、
\begin{align}
    \frac{1}{T_{jj}^n} \pdv{\theta} T_{jj}^{n,\theta} &= \frac{1}{T_{jj}} \pdv{\theta} (1-\sum_{i(\neq j)}T_{ij}^{n,\theta}) = -\frac{1}{T_{jj}} \sum_{i(\neq j)} Z_{ij} T_{ij}
\end{align}
となるから、
\begin{align}
    T_{jj}\qty(\frac{1}{T_{jj}} \pdv{\theta} T_{jj}^{n,\theta})^2 = \frac{1}{T_{jj}} (\sum_{i(\neq j)} Z_{ij} T_{ij})^2 =T_{jj}(Z_{jj}^n)^2
\end{align}
となる。したがって、    
\begin{align}
    F(0) &= \sum_{n=1}^{N} \sum_{j} (p_j^{n-1} \sum_{i(\neq j)}(Z_{ij}^n)^2 T_{ij} + p_j^{n-1} T_{jj}^n (Z_{jj}^n)^2)\\
    &= \sum_{n=1}^{N} \ev{\hat{Z}^2}_n
\end{align}
となる。最後の期待値は、$n$ステップ目におけるjump quantityの意味での期待値である。\\
ここで求めた$F(0)$をエントロピー生成で抑えることを考える。Markov jumpのときとは異なり、以下で見るようなmodified entropy productionにより抑えられる。\\
\textbf{非対角成分}\footnote{$i \neq j$なので、5行目は不等号}
\begin{align}
    \sum_{n=1}^{N} \sum_{i\neq j} \ev{(Z_{ij}^n)^2} &= \frac{1}{2}\sum_{n=1}^{N} \sum_{i\neq j} \frac{(K_{ij}^n - K_{ji}^n)^2}{(K_{ij}^n + K_{ji}^n)^2} \\
    &\leq \frac{1}{2}\sum_{n=1}^{N} \sum_{i\neq j} K_{ij}^n \ln \frac{K_{ij}^n}{K_{ji}^n}\\
    &= \frac{1}{2}\sum_{n=1}^{N} \sum_{i\neq j} T_{ij} p_j^{n-1} \ln \frac{T_{ij} p_j^{n-1} p_i^{n}}{T_{ji} p_i^{n-1} p_j^{n}}\\
    &= \frac{1}{2}\sum_{n=1}^{N} \sum_{i\neq j} \qty(T_{ij} p_j^{n-1} \ln \frac{T_{ij} p_j^{n-1}}{T_{ji} p_i^{n}} + T_{ij} p_j^{n-1} \ln \frac{p_i^{n}}{p_i^{n-1}})\\
    &\leq \frac{1}{2}(\sigma + \sum_{n=1}^{N} D(p^n \| p^{n-1}))\\
    &= \frac{1}{2}\tilde{\sigma}
\end{align}

\textbf{対角成分}
\begin{align}
    \sum_{n=1}^{N} \sum_{i} \ev{(Z_{ii}^n)^2} &= \sum_{n=1}^{N} \sum_{i} p_{i}^{n-1} \frac{(\sum_{j(\neq i)}T_{ji}Z_{ji}^n)^2}{T_{ii}^n} \\
    &= \sum_{n=1}^{N} \sum_{i} p_{i}^{n-1} \frac{(\sum_{j(\neq i)}T_{ji}Z_{ji}^n)^2}{\sum_{j(\neq i)}T_{ji}}\frac{1-T_{ii}}{T_{ii}}\\
    &\leq \sum_{n=1}^{N} \sum_{i} p_i^{n-1} \sum_{j (\neq i)} T_{ji} \left(Z_{ji}^n\right)^2 \frac{1 - T_{ii}}{T_{ii}} \quad \because \text{Cauchy-Schwarz}\\
    &\leq \frac{1 - \min{_i} T_{ii}}{\min_{i} T_{ii}} \sum_{n=1}^{N} \sum_{i} p_i^{n-1} \sum_{j (\neq i)} T_{ji} \left(Z_{ji}^n\right)^2 \\
    &\leq \frac{1 - a}{a} \frac{1}{2}\tilde{\sigma}
\end{align}
したがって、対角成分と非対角成分を合わせて、
\begin{align}
    F(0) = \sum_{n=1}^{N} \ev{\hat{Z}^2}_n \leq \frac{1}{a}\frac{\tilde{\sigma}}{2}
\end{align}
となる。\\
\noindent
\textbf{fについて}\\
後の計算のための準備をする。まず、$i \neq j$で
\begin{align}
    Z_{ij}^n K_{ij}^n - Z_{ji}^n K_{ji}^n &= Z_{ij}^n(K_{ij}^n + K_{ji}^n)\\
    &= K_{ij}^n - K_{ji}^n \quad (i \neq j)
\end{align}
が成り立つ。これは、$i=j$の場合も自明に成り立つ。したがって、
\begin{align}
    Z_{ij}^n K_{ij}^n - Z_{ji}^n K_{ji}^n = K_{ij}^n - K_{ji}^n
\end{align}
となる。それゆえ、$\theta-$遷移行列は以下の関係を満たす。
\begin{align}
    \left[\left(T^{n,\theta} - T\right)p^{n-1}\right]_i 
    &= \sum_{j} \theta Z_{ij}^n T_{ij} p_j^{n-1} \\
    &= \sum_{j} \theta Z_{ij}^n K_{ij}^n \\
    &= \sum_{j (\neq i)} \theta \left(Z_{ij}^n K_{ij}^n - Z_{ji}^n K_{ji}^n\right) \\
    &= \sum_{j (\neq i)} \theta \left(K_{ij}^n - K_{ji}^n\right) \\
    &= \theta \sum_{j (\neq i)} (T_{ij}p_j^{n-1} - T_{ji}p_i^{n-1})\\
    &= \theta \sum_{j (\neq i)} T_{ij}p_j^{n-1} - \theta(1-T_{ii})p_i^{n-1}\\
    &= \theta \left(p_i^n - p_i^{n-1}\right) 
\end{align}
この関係を用いることで、$\theta$系における$m$ステップ目の確率分布を計算できる。\footnote{二つ目の等号について、$T^{n,\theta} = T^n + \theta \phi^n + O(\theta^2)$の形を踏まえて1次まで展開する。ただし、積のどこの一次をとるかで和がとられる。}
\begin{align}
    p^{\theta, m} 
    &= \prod_{n=1}^{m} T^{n,\theta} p^0 \\
    &= (T)^m p^0 + \sum_{n=1}^{m} (T)^{m-n} (T^{n,\theta} - T) (T)^{n-1} p^0 + O(\theta^2) \\
    &= p^m + \sum_{n=1}^{m} (T)^{m-n} (T^{n,\theta} - T) p^{n-1} + O(\theta^2) \\
    &= p^m + \theta \sum_{n=1}^{m} (T)^{m-n} (p^n - p^{n-1}) + O(\theta^2) \\
    &= p^m + \theta m (p^m - p^{m-1}) + O(\theta^2) \label{eq:stepm}
\end{align}
以上の準備のもと、$\expval{\hat{\mathcal{J}}_d}^\theta$を計算する。
\begin{align}
    \expval{\hat{\mathcal{J}}_d}^\theta 
    &= \sum_{n=1}^{N} \sum_{i,j} d_{ij} T_{ij}^{n,\theta} p_j^{\theta,n-1} \\
    &= \sum_{n=1}^{N} \sum_{i,j} d_{ij} T_{ij}^{n,\theta} \left[ p_j^{n-1} + \theta (n-1)(p_j^{n-1} - p_j^{n-2}) \right] + O(\theta^2) \quad \because \eqref{eq:stepm}\\
     \intertext{第一項について$T^{n,\theta}$を展開して、}
    &= \sum_{n=1}^{N} \sum_{i,j} d_{ij} T_{ij}^n p_j^{n-1} + \theta \sum_{n=1}^{N} \sum_{i,j} d_{ij} T_{ij}^n Z_{ij}^n p_j^{n-1} \\
    &\quad + \theta \sum_{n=1}^{N} \sum_{i,j} (n-1) d_{ij} T_{ij} (p_j^{n-1} - p_j^{n-2}) + O(\theta^2)\\
    &= \expval{\hat{\mathcal{J}}_d} + \theta \sum_{n=1}^{N} \sum_{i,j} d_{ij} T_{ij}^n Z_{ij}^n p_j^{n-1} + \theta \sum_{n=1}^{N} \sum_{i,j} (n-1) d_{ij} T_{ij} (p_j^{n-1} - p_j^{n-2}) + O(\theta^2)
\end{align}
となる。第二項について、
\begin{align}
    \sum_{i,j} d_{ij} T_{ij} Z_{ij}^n p_j^{n-1} 
    &= \sum_{(i,j)} d_{ij} \left(T_{ij} Z_{ij}^n p_j^{n-1} - T_{ji} Z_{ji}^n p_i^{n-1}\right) \\
    &= \sum_{(i,j)} d_{ij} \left(T_{ij} p_j^{n-1} + T_{ji} p_i^{n-1}\right) Z_{ij}^n \\
    &= \sum_{(i,j)} d_{ij} \left(T_{ij} p_j^{n-1} - T_{ji} p_i^{n-1}\right) \\
    &= \sum_{i,j} d_{ij} T_{ij} p_j^{n-1}
\end{align}
となることを用いて、
\begin{align}
    \expval{\hat{\mathcal{J}}_d}^\theta 
    &= \expval{\hat{\mathcal{J}}_d} + \theta \sum_{n=1}^{N} \sum_{i,j} d_{ij} T_{ij} p_j^{n-1} \nonumber \\
    &\quad + \theta \sum_{n=1}^{N} \sum_{i,j} (n-1) d_{ij} T_{ij} \left(p_j^{n-1} - p_j^{n-2}\right) + o(\theta^2) \\
    &= \expval{\hat{\mathcal{J}}_d} + \theta \sum_{n=1}^{N} \sum_{i,j} \left[ n d_{ij} T_{ij} p_j^{n-1} - (n-1) d_{ij} T_{ij} p_j^{n-2} \right] + O(\theta^2) \\
    &= \expval{\hat{\mathcal{J}}_d} + \theta N \mathcal{J}_d^{N-1} + O(\theta^2)
\end{align}
最後の等号は、和をとるときに$n=N$のみ生き残って他は相殺することからわかる。\\

以上の結果を用いて、\eqref{eq:2}を利用すると、
\begin{align}
    (N \mathcal{J}_d^{N-1})^2 \leq \text{Var}_0(\hat{\mathcal{J}}_d) F(0) \leq \text{Var}_0(\hat{\mathcal{J}}_d) \frac{1}{a} \frac{\tilde{\sigma}}{2}
\end{align}
となる。最右辺と再左辺を比較することで、
\begin{align}
    \frac{\text{Var}(\hat{\mathcal{J}}_d)}{(N \mathcal{J}_d^{N-1})^2} \frac{\tilde{\sigma}}{a} \geq 2
\end{align}
が成り立つことがわかる。\qed

\subsection{KUR}

\begin{itembox}[l]{\textbf{Def.Activity}}
    状態の組(i,j)に対するactivityは、
    \begin{align}
        \hat{A}_{ij} := -\frac{1}{\tau} \sum_{n} \left( 
    \delta_{w_j \to w_i}(w^{n-1} \to w^n) + \delta_{w_i \to w_j}(w^{n-1} \to w^n) 
    \right)
    \end{align}
    で定義される。また、系のactivityは、
    \begin{align}
        \hat{A} := \sum_{(i,j)} d_{ij} \hat{A}_{ij}
    \end{align}
    で定義される。
\end{itembox}

\begin{align}
    \expval{\hat{1}} = \mathcal{A} := \int_{0}^{\tau} dt \, A(t) := \int dt \sum_{i \neq j} R_{ij} p_j(t).
\end{align}
TODOhoge(記号まわり)

\begin{itembox}[l]{\textbf{Thm.Kinetic uncertainty relation}}
    jump quantity$X$に対して、$\hat{\mathcal{X}}$のゆらぎは以下のように抑えられる。
    \begin{align}
        \frac{\operatorname{Var} \hat{\mathcal{X}}}{\left(\tau X(\tau)\right)^2} \mathcal{A} \geq 1
    \end{align}
\end{itembox}
\textbf{Prf.}\\
$\theta-$遷移レートを以下のように定義する。
\begin{align}
    R_{ij}^{\theta} &= R_{ij}e^{\theta } \quad (i \neq j)\\
    R_{jj}^{\theta} &= -\sum_{i (\neq j)} R_{ij}^{\theta}e^{\theta }\\
\end{align}
Cram\'er-Raoの不等式を用いる。すなわち、
\begin{align}
    \eval{\left( \frac{\partial \langle {\hat{\mathcal{X}}_{d}} \rangle_\theta}{\partial \theta} \right)^2}_{\theta = 0}
    \leq \mathrm{Var}_{0} (\hat{\mathcal{X}}) F(0) \label{eq:2}
    \end{align}
    を用いる。

\textbf{Fisher情報量について}\\
\begin{align}
    F(0) &= \int_{0}^{\tau} dt \sum_{i \neq j} R_{ij} p_j(t) = \mathcal{A} \quad \because (\ref{eq:ff})\text{で}Z=1\text{とした。}
\end{align}

\textbf{$\ev{\hat{\mathcal{X}}}_{\theta}$について}\\
$\theta$系の時間発展は、
\begin{align}
    \pdv{p_i^{\theta}(t)}{t} &= \sum_{j} R_{ij}^{\theta}p_j^{\theta} (t)\\
    &= e^{\theta} \sum_{j} R_{ij}p_j^{\theta} (t)
\end{align}
となる。この式から、
\begin{align}
    p_i^{\theta}(t) = p_i(e^{\theta}t)
\end{align}
となっていることがわかる。(タイムスケールの変換)。これを用いると、
\begin{align}
    \pdv{\ev{\hat{\mathcal{X}}}_{\theta}}{\theta} &= \pdv{\theta}\int_{0}^{\tau} \dd{t} \sum_{i \neq j} R_{ij}^{\theta} X_{ij}p_j^{\theta}(t)\\
    &= \pdv{\theta}\int_{0}^{\tau} \dd{t} e^{\theta} \sum_{i \neq j} R_{ij} X_{ij}p_j(e^{\theta}t)\\
    &= \pdv{\theta}\int_{0}^{e^{\theta}\tau} \dd{t} \sum_{i \neq j} R_{ij} X_{ij}p_j(t)\\
    &\underset{\theta \to 0}{\longrightarrow} \tau \ev{\hat{X}(\tau)}
\end{align}
となる。\\
以上の結果とCram\'er-Raoの不等式を用いると、
\begin{align}
    (\tau \ev{\hat{X}(\tau)})^2 \leq \text{Var}_0(\hat{\mathcal{X}}) \mathcal{A}
\end{align}
となる。これより、
\begin{align}
    \frac{\text{Var}(\hat{\mathcal{X}})}{\qty(\tau \ev{\hat{X}(\tau)})^2} \mathcal{A} \geq 1
\end{align}
が成り立つことがわかる。\qed

\section{Cram\'er-Raoの不等式}
統計学の文脈でのCram\'er-Raoの不等式について述べる。\\
$P_{\theta}(x)$をパラメータ$\theta$をもつ確率変数$x$の確率分布であるとする。また、引数$\theta$の関数$f(\theta)$が与えられていたとする。
ここでの目的は、$x$を$g(x)$として測定したとき、$f(\theta)$を推定することである。ここで、$g(x)$が$f(\theta)$の不偏推定量であると仮定する。すなわち、
\begin{align}
    f(\theta) = \ev{g(x)} 
\end{align}
が成り立つとする。

このとき、Cram\'er-Raoの不等式は以下のように表される:
\begin{align}
    \text{Var}_{\theta}(g(x)) \geq \frac{(f')^2}{F(\theta)}
\end{align}
ただし、$F(\theta)$はFisher情報量であり、以下のように定義される:
\begin{align}
    F(\theta) = -\ev{\pdv[2]{\theta} \log P_{\theta}(x)} =\ev{\qty(\pdv{\theta} \log P_{\theta}(x))^2}
\end{align}
Cram\'er-Raoの不等式の左辺は、どれほどこの推定が正確かを表している。
右辺のフィッシャー情報量は、$\theta$に対する鋭敏性を表している。もし、$F(\theta)$が大きいとき、$\theta$の変化に対して、確率分布$P_{\theta}(x)$が大きく変化することになる。
また、$F(\theta)$が小さい場合は、$\theta$の変化に対して、確率分布$P_{\theta}(x)$があまり変化しないことを意味しており、これは、測定値から$\theta$を測定するのが難しいということを意味する。\\


例えば、誤差が毎回でるような秤を用いて物の重さを測定することを考える。このとき、各測定値が$x$であり、真の重さが$\theta$に対応する。仮に、測定値が以下のような$\theta$まわりのガウス分布に従うとする:
\begin{align}
    P_{\theta}(x) = \frac{1}{\sqrt{2\pi \sigma^2}}\exp(-\frac{(x-\theta)^2}{2\sigma^2})
\end{align}
このとき、Fisher情報量は、
\begin{align}
    F(\theta) &= \ev{\qty(\frac{(x-\theta)}{\sigma^2})^2} = \frac{1}{\sigma^2}
\end{align}
となる。このとき、$\sigma$が大きいと、$F(\theta)$が小さくなり、$\theta$の推定が難しくなることがわかる。

Cram\'er-Raoの不等式は、正確な推定の限界を示すものである。すなわち、不偏推定量を変えることで推定の精度を向上させたいときに、どこまで精度を向上させることができるかを示すものである。

\textbf{ex.}\\

\begin{align}
    \sum_{x'} W_{xx'} \frac{W_{xx'}P(x')-W_{x'x}P(x)}{W_{xx'}P(x')+W_{x'x}P(x)} P(x') = \sum_{x'} W_{xx'}  P(x') 
\end{align}


\end{document}