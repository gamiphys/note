\documentclass[a4paper,11pt]{jsarticle}

% 数式
\usepackage{amsmath,amsfonts}
\usepackage{bm}
\usepackage{mathtools}

% 表
\usepackage[utf8]{inputenc}
\usepackage{diagbox} % 斜線付きセルを作成するために必要
\usepackage{booktabs} % 表の罫線を美しくするために必要
\usepackage{hhline} % 水平罫線を制御するために必要

% 画像
\usepackage[dvipdfmx]{graphicx}
\usepackage{ascmac}
\usepackage{physics}
\usepackage{float} % 追加

% 図
\usepackage[dvipdfmx]{graphicx}
\usepackage{tikz} %図を描く
\usetikzlibrary{positioning, intersections, calc, arrows.meta,math} %tikzのlibrary

% ハイパーリンク
\usepackage[dvipdfm,
  colorlinks=false,
  bookmarks=true,
  bookmarksnumbered=false,
  pdfborder={0 0 0},
  bookmarkstype=toc]{hyperref}

\begin{document}

\title{Lieb-Yngvasonノート}
\author{大上由人}
\date{\today}
\maketitle

\section{状態と状態空間}
\begin{itembox}[l]{\textbf{Def:状態と状態空間}}
状態は、状態空間$\Gamma$の点であり、$X,Y,Z$などと表す。
\end{itembox}

\begin{itembox}[l]{\textbf{Def:状態空間の合成}}
    状態空間$\Gamma_1$と$\Gamma_2$の合成は、状態空間の直積$\Gamma_1 \times \Gamma_2$である。\\
\end{itembox}
また、物理的直観から明らかなように、複数の状態空間を合成するとき、その合成の順序は問題にならない。すなわち、
\begin{equation}
    (\Gamma_1 \times \Gamma_2) \times \Gamma_3 = \Gamma_1 \times (\Gamma_2 \times \Gamma_3)
\end{equation}
である。

\begin{itembox}[l]{\textbf{Def:スケーリングコピー}}
    $t>0$に対して、状態空間$\Gamma(t)$の構成要素を、$tX$とする。\\
    このとき、$\Gamma(t)$は、$\Gamma$のスケーリングコピーである。また、状態空間が$\mathbb{R}^n$の部分集合であるときは、$tX$は、ベクトルのスカラー倍としての表現となる。
    このとき、$\Gamma(t)=t\Gamma$と書く。
\end{itembox}
このとき、スケーリングコピーに関する以下の性質は、明らかである。
\begin{itemize}
    \item $\Gamma(1)=\Gamma$
    \item $1X=X$
    \item $(\Gamma(t))(s)=\Gamma(ts)$
    \item $s(tX)=(st)X$
    \item $(\Gamma_1 \times \Gamma_2)(t)=\Gamma_1(t) \times \Gamma_2(t)$
    \item $t(X,Y)=(tX,tY)$
\end{itemize}

\begin{itembox}[l]{\textbf{Def:多重スケーリングコピー}}
    状態空間が$\Gamma_1,\Gamma_2,\cdots,\Gamma_n$の直積であるとき、$t_1,t_2,\cdots,t_n>0$に対して、
    \begin{equation}
        \Gamma_1(t_1) \times \Gamma_2(t_2) \times \cdots \times \Gamma_n(t_n)
    \end{equation}
    なる直積を形成できる。とくに、$\Gamma_1=\Gamma_2=\cdots=\Gamma_n=\Gamma$のとき、
    \begin{equation}
        \Gamma(t_1,t_2,\cdots,t_n)=\Gamma(t_1) \times \Gamma(t_2) \times \cdots \times \Gamma(t_n)
    \end{equation}
    と書く。これを多重スケーリングコピーという。
\end{itembox}

\textbf{ex:状態空間}\\
(a)水素が1molのとき、状態空間$\Gamma_a$は、エネルギーと体積で表され、$\mathbb{{R}}^2$の部分集合である。\\
(b)水素が0.5molのとき、状態空間$\Gamma_b$は、エネルギーと体積で表され、$\mathbb{{R}}^2$の部分集合であり、$\Gamma_b=\frac{1}{2}\Gamma_a=\{(\frac{1}{2}U,\frac{1}{2}V)|U,V \in \Gamma_a\}$である。\\
(c)水素が1mol、酸素が0.5mol(混合されていない)のとき、状態空間$\Gamma_c=\Gamma_a \times (酸素0.5molの状態空間)$であり、複合系である。\\
(d)水1molのとき、状態空間$\Gamma_d$
(e)水素1molと酸素0.5mol(混合)のとき、$\Gamma_e\neq \Gamma_d$であり、$\Gamma_e\neq \Gamma_c$である。\\

\section{順序関係}
\end{document}
