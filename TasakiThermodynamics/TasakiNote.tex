\documentclass[a4paper,11pt]{jsarticle}

% 数式
\usepackage{amsmath,amsfonts}
\usepackage{bm}
\usepackage{mathtools}

% 表
\usepackage[utf8]{inputenc}
\usepackage{diagbox} % 斜線付きセルを作成するために必要
\usepackage{booktabs} % 表の罫線を美しくするために必要
\usepackage{hhline} % 水平罫線を制御するために必要

% 画像
\usepackage[dvipdfmx]{graphicx}
\usepackage{ascmac}
\usepackage{physics}
\usepackage{float} % 追加

% 図
\usepackage[dvipdfmx]{graphicx}
\usepackage{tikz} %図を描く
\usetikzlibrary{positioning, intersections, calc, arrows.meta,math} %tikzのlibrary

\begin{document}

\title{田崎熱力学ノート}
\author{大上由人}
\date{\today}
\maketitle

\section{前提など}
\subsection{はじめに}
まず、田崎熱における熱力学への立場を明確にする。\\
\begin{itembox}[l]{\textbf{前提}}
    熱力学系は、その外にマクロな力学的な世界が存在する。

\end{itembox}
以上の過程の意味するところは、我々は、熱力学系を詳細に\footnote[1]{例えば、粒子一つ一つの運動状態を知ることができる程度のこと。}知ることはできないが、外から仕事を加えるなどして、系の状態を操作することができ、その操作によって、系の状態を操作することができるということである。\\

\section{要請}
\subsection{熱力学系に関する要請}
いくらかの要請を課す。\\
\begin{itembox}[l]{\textbf{要請2.1:平衡状態}}
    ある環境に熱力学系を置き、示量変数を固定したまま十分長い時間が経過した後、系は平衡状態に達する。また、同じ環境に置いた系の平衡状態は、示量変数の組の値だけで完全に決定される。
\end{itembox}
\begin{itembox}[l]{\textbf{要請2.2:環境と温度}}
    各々の環境を特徴づける温度という実数の量が存在する。環境に置いた熱力学系の平衡状態を左右するのは環境の温度のみであり、環境の温度以外の詳細によらない。
\end{itembox}
以上の二つを組み合わせると、以下のことがわかる。\\
\begin{itembox}[l]{\textbf{結果2.3:平衡状態の記述}}
    熱力学的な系の平衡状態は、環境の温度と示量変数の組$(T,X)$だけで完全に指定できる。
\end{itembox}
すなわち、環境を$T$で、熱力学系を$X$で表すことで、平衡状態を区別しているのである。

また、以下、熱力学系の状態は、系の形状や、重力による効果を無視して考える。\\

\subsection{断熱系}
断熱壁に囲まれた系について、以下の要請を行う。
\begin{itembox}[l]{\textbf{要請2.4:断熱系の平衡状態}}
    熱力学的な系を断熱壁で囲み、示量変数の組$X$を固定したまま十分長い時間が経過すると、系はある平衡状態(T,X)に達する。ただし、このときの平衡状態の温度Tは系の初めの状況によって決まる。
\end{itembox}
ここで注意したいこととしては、断熱壁に囲まれることによって達成される平衡状態$(T,X)$が、環境のもと平衡状態に達した$(T,X)$と同じであることも要請している点である。\\

\section{等温操作}
\subsection{等温操作}

\begin{itembox}[l]{\textbf{def:等温操作}}
    等温操作とは、温度一定の環境下で、ある平衡状態を、別の平衡状態に移すことである。すなわち、
    \begin{equation}
        (T,X_1) \xrightarrow{i} (T,X_2)
    \end{equation}
    という操作を等温操作という。
\end{itembox}
ここで注意すべきことは、この操作の途中においては、(系が環境のもとにあるならば)どんな操作を行っても良いということである。例えば、途中で断熱壁で囲ってしまうことをしても良い。というのも、操作の途中においては、系は平衡状態とは限らず、系の温度が定義されないからである。\\

\begin{itembox}[l]{\textbf{def:等温準静操作}}
    等温操作のうち、系がいつも平衡状態にあるような操作を等温準静操作という。これを、
    \begin{equation}
        (T,X_1) \xrightarrow{iq} (T,X_2)
    \end{equation}
    と表す。
\end{itembox}
例えば、平衡状態にある系に、壁をそっと差し込む操作は準静的である。\\
しかし、逆に、平衡状態にある系からそっと壁を引き抜く行為は一般に準静的でない。というのも、壁の両側でそれぞれ平衡状態にあるとしても、両者の平衡状態が一致するとは限らず、壁を引き抜いたときに流れが生じてしまうからである。\\
ただし、すでに両側の平衡状態が釣り合っている場合、壁を引き抜く過程は準静的である。これは、系に対する力学的操作および系の状態の両方を、「平衡状態にある系から壁をゆっくり引き抜く」操作の真逆になっていることに対応する。\footnote{ここは嘘をついている可能性がある。}\\

上の例を見てみると、等温準静操作においては、示量変数の組の動きを完全に逆向きにたどることで、逆向きの等温準静操作を行うことができることがわかる。このとき、系の示量変数の組の動きは、ちょうど時間反転対称性を持つ。\\
しかし、準静的でない場合については、このような時間反転対称性は成り立たない。というのも、系の示量変数がそもそも定まるかもよくわからないし\footnote{ここ自信ない}、定義できたとしても、非平衡状態なのだから、逆向きの操作を行っても、元の状態に戻るとは限らない。

\subsection{kelvinの原理}
\begin{itembox}[l]{\textbf{kelvinの原理}}
    任意の温度における任意の等温サイクルについて、系は外部に正の仕事を行うことなく、そのサイクルを完了することはできない。すなわち、
    \begin{equation}
        W_{\text{cyc}} \leq 0
    \end{equation}
    が成り立つ。
\end{itembox}
所謂、第二永久機関は存在しないということである。\\

\begin{itembox}[l]{\textbf{prop:等温準静サイクルの仕事}}
    等温準静サイクルにおいて、系が外部に行う仕事は、
    \begin{equation}
        W_{\text{cyc}} = 0
    \end{equation}
    である。
\end{itembox}
\textbf{Prf}\\
kelvinの原理より、
\begin{equation}
    W_{\text{cyc}} \leq 0
\end{equation}
また、逆過程も等温準静であるから、
\begin{equation}
    -W_{\text{cyc}} \leq 0
\end{equation}
以上より、
\begin{equation}
    W_{\text{cyc}} = 0
\end{equation}
が成り立つ。\\




\end{document}