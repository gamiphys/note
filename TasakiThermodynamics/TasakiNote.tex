\documentclass[a4paper,11pt]{jsarticle}

% 数式
\usepackage{amsmath,amsfonts}
\usepackage{bm}
\usepackage{mathtools}

% 表
\usepackage[utf8]{inputenc}
\usepackage{diagbox} % 斜線付きセルを作成するために必要
\usepackage{booktabs} % 表の罫線を美しくするために必要
\usepackage{hhline} % 水平罫線を制御するために必要

% 画像
\usepackage[dvipdfmx]{graphicx}
\usepackage{ascmac}
\usepackage{physics}
\usepackage{float} % 追加

% 図
\usepackage[dvipdfmx]{graphicx}
\usepackage{tikz} %図を描く
\usetikzlibrary{positioning, intersections, calc, arrows.meta,math} %tikzのlibrary

\begin{document}

\title{田崎熱力学ノート}
\author{大上由人}
\date{\today}
\maketitle

\section{前提など}
\subsection{はじめに}
まず、田崎熱における熱力学への立場を明確にする。\\
\begin{itembox}[l]{\textbf{前提}}
    熱力学系は、その外にマクロな力学的な世界が存在する。

\end{itembox}
以上の過程の意味するところは、我々は、熱力学系を詳細に\footnote[1]{例えば、粒子一つ一つの運動状態を知ることができる程度のこと。}知ることはできないが、外から仕事を加えるなどして、系の状態を操作することができ、その操作によって、系の状態を操作することができるということである。\\

\subsection{hoge}



\end{document}