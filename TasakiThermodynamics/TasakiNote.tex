\documentclass[a4paper,11pt]{jsarticle}

% 数式
\usepackage{amsmath,amsfonts}
\usepackage{bm}
\usepackage{mathtools}

% 表
\usepackage[utf8]{inputenc}
\usepackage{diagbox} % 斜線付きセルを作成するために必要
\usepackage{booktabs} % 表の罫線を美しくするために必要
\usepackage{hhline} % 水平罫線を制御するために必要

% 画像
\usepackage[dvipdfmx]{graphicx}
\usepackage{ascmac}
\usepackage{physics}
\usepackage{float} % 追加

% 図
\usepackage[dvipdfmx]{graphicx}
\usepackage{tikz} %図を描く
\usetikzlibrary{positioning, intersections, calc, arrows.meta,math} %tikzのlibrary

% プリアンブルに追加する
\usepackage[dvipdfm,
  colorlinks=false,
  bookmarks=true,
  bookmarksnumbered=false,
  pdfborder={0 0 0},
  bookmarkstype=toc]{hyperref}


\begin{document}

\title{田崎熱力学ノート}
\author{大上由人}
\date{\today}
\maketitle

\tableofcontents

\newpage

\section{前提など}
\subsection{はじめに}
まず、田崎熱における熱力学への立場を明確にする。\\
\begin{itembox}[l]{\textbf{前提}}
    熱力学系は、その外にマクロな力学的な世界が存在する。

\end{itembox}
以上の過程の意味するところは、我々は、熱力学系を詳細に\footnote[1]{例えば、粒子一つ一つの運動状態を知ることができる程度のこと。}知ることはできないが、外から仕事を加えるなどして、系の状態を操作することができ、その操作によって、系の状態を操作することができるということである。\\

\section{要請}
\subsection{熱力学系に関する要請}
いくらかの要請を課す。\\
\begin{itembox}[l]{\textbf{要請2.1:平衡状態}}
    ある環境に熱力学系を置き、示量変数を固定したまま十分長い時間が経過した後、系は平衡状態に達する。また、同じ環境に置いた系の平衡状態は、示量変数の組の値だけで完全に決定される。
\end{itembox}
\begin{itembox}[l]{\textbf{要請2.2:環境と温度}}
    各々の環境を特徴づける温度という実数の量が存在する。環境に置いた熱力学系の平衡状態を左右するのは環境の温度のみであり、環境の温度以外の詳細によらない。
\end{itembox}
以上の二つを組み合わせると、以下のことがわかる。\\
\begin{itembox}[l]{\textbf{結果2.3:平衡状態の記述}}
    熱力学的な系の平衡状態は、環境の温度と示量変数の組$(T,X)$だけで完全に指定できる。
\end{itembox}
すなわち、環境を$T$で、熱力学系を$X$で表すことで、平衡状態を区別しているのである。\\

また、以下、熱力学系の状態は、系の形状や、重力による効果を無視して考える。\\

\subsection{断熱系}
断熱壁に囲まれた系について、以下の要請を行う。
\begin{itembox}[l]{\textbf{要請2.4:断熱系の平衡状態}}
    熱力学的な系を断熱壁で囲み、示量変数の組$X$を固定したまま十分長い時間が経過すると、系はある平衡状態(T,X)に達する。ただし、このときの平衡状態の温度Tは系の初めの状況によって決まる。
\end{itembox}
ここで注意したいこととしては、断熱壁に囲まれることによって達成される平衡状態$(T,X)$が、環境のもと平衡状態に達した$(T,X)$と同じであることも要請している点である。\\

\section{等温操作}
\subsection{等温操作}

\begin{itembox}[l]{\textbf{def:等温操作}}
    等温操作とは、温度一定の環境下で、ある平衡状態を、別の平衡状態に移すことである。すなわち、
    \begin{equation}
        (T,X_1) \xrightarrow{i} (T,X_2)
    \end{equation}
    という操作を等温操作という。
\end{itembox}
ここで注意すべきことは、この操作の途中においては、(系が環境のもとにあるならば)どんな操作を行っても良いということである。例えば、途中で断熱壁で囲ってしまうことをしても良い。というのも、操作の途中においては、系は平衡状態とは限らず、系の温度が定義されないからである。\\

\begin{itembox}[l]{\textbf{def:等温準静操作}}
    等温操作のうち、系がいつも平衡状態にあるような操作を等温準静操作という。これを、
    \begin{equation}
        (T,X_1) \xrightarrow{iq} (T,X_2)
    \end{equation}
    と表す。
\end{itembox}
例えば、平衡状態にある系に、壁をそっと差し込む操作は準静的である。\\
しかし、逆に、平衡状態にある系からそっと壁を引き抜く行為は一般に準静的でない。というのも、壁の両側でそれぞれ平衡状態にあるとしても、両者の平衡状態が一致するとは限らず、壁を引き抜いたときに流れが生じてしまうからである。\\
ただし、すでに両側の平衡状態が釣り合っている場合、壁を引き抜く過程は準静的である。これは、系に対する力学的操作および系の状態の両方を、「平衡状態にある系から壁をゆっくり引き抜く」操作の真逆になっていることに対応する。\footnote{ここは嘘をついている可能性がある。}\\

上の例を見てみると、等温準静操作においては、示量変数の組の動きを完全に逆向きにたどることで、逆向きの等温準静操作を行うことができることがわかる。このとき、系の示量変数の組の動きは、ちょうど時間反転対称性を持つ。\\
しかし、準静的でない場合については、このような時間反転対称性は成り立たない。というのも、系の示量変数がそもそも定まるかもよくわからないし\footnote{ここ自信ない}、定義できたとしても、非平衡状態なのだから、逆向きの操作を行っても、元の状態に戻るとは限らない。


\subsection{Kelvinの原理}
\begin{itembox}[l]{\textbf{Kelvinの原理}}
    任意の温度における任意の等温サイクルについて、系は外部に正の仕事を行うことなく、そのサイクルを完了することはできない。すなわち、
    \begin{equation}
        W_{\text{cyc}} \leq 0
    \end{equation}
    が成り立つ。
\end{itembox}

\begin{itembox}[l]{\textbf{prop:等温準静サイクルの仕事}}
    等温準静サイクルにおいて、系が外部に行う仕事は、
    \begin{equation}
        W_{\text{cyc}} = 0
    \end{equation}
    である。
\end{itembox}
\textbf{Prf}\\
Kelvinの原理より、
\begin{equation}
    W_{\text{cyc}} \leq 0
\end{equation}
また、逆過程も等温準静であるから、
\begin{equation}
    -W_{\text{cyc}} \leq 0
\end{equation}
以上より、
\begin{equation}
    W_{\text{cyc}} = 0
\end{equation}
が成り立つ。\\

\subsection{最大仕事}
\begin{itembox}[l]{\textbf{def:最大仕事}}
    示量変数の組$X$で記述される系において、等温操作
    \begin{equation}
        (T,X_1) \xrightarrow{i} (T,X_2)
    \end{equation}
    を考える。操作の間に系が外界にする仕事の最大値を$W_{\text{max}(T;X_1\rightarrow X_2)}$と書き、最大仕事という。
\end{itembox}

\begin{itembox}[l]{\textbf{prop:最大仕事の原理}}
    最大仕事$W_{\text{max}(T;X_1\rightarrow X_2)}$は、等温準静操作において系が外界にする仕事に等しい。
\end{itembox}
\textbf{Prf}\\
ある等温準静操作を、$(T,X_1) \xrightarrow{iq} (T,X_2)$とする。このとき、系が外界にする仕事を$W$とする。以下、この$W$が$W_{\text{max}(T;X_1\rightarrow X_2)}$に等しいことを示す。\\
準静的とは限らない任意の等温操作$(T,X_1) \xrightarrow{i} (T,X_2)$を考える。このとき、系が外界にする仕事を$W'$とする。このとき、
\begin{equation}
    (T,X_1) \xrightarrow{i} (T,X_2) \xrightarrow{iq} (T,X_1)
\end{equation}
という操作を考えると、系が外界にする仕事は、
\begin{equation}
    W_{\text{cyc}} = W'-W
\end{equation}
である。\\
ここで、Kelvinの原理より、
\begin{equation}
    W_{\text{cyc}} \leq 0
\end{equation}
であるから、
\begin{equation}
    W' \leq W
\end{equation}
が成り立つ。今、準静的とは限らない等温操作の選び方は任意であったから、最大仕事$W_{\text{max}}(T;X_1\rightarrow X_2)$は、等温準静操作において系が外界にする仕事に等しい。\\

最大仕事の原理により、準静的操作において系が外界にする仕事を測定することで、最大仕事を求めることができる。\\

\begin{itembox}[l]{\textbf{prop:最大仕事の性質}}
    最大仕事の性質は以下の通りである。
    \begin{enumerate}
        \item 等温準静操作を逆向きに行うときに系が外界にする仕事は$W_{\text{max}}(T;X_2\rightarrow X_1)=-W_{\text{max}}(T;X_1\rightarrow X_2)$である。
        \item 最大仕事に結合率が成り立つ。すなわち、$W_{\text{max}}(T;X_1\rightarrow X_2)+W_{\text{max}}(T;X_2\rightarrow X_3) \geq W_{\text{max}}(T;X_1\rightarrow X_3)$である。
        \item 最大仕事には相加性がある。すなわち、$W_{\text{max}}(T;X_1\rightarrow X_2)+W_{\text{max}}(T;Y_1\rightarrow Y_2) = W_{\text{max}}(T;\{X_1,Y_1\}\rightarrow \{X_2,Y_2\})$である。
        \item 最大仕事には示量性が成り立つ。
    \end{enumerate}
\end{itembox}
\textbf{Prf}\\
あとで書く。\\

\subsection{Helmholtz自由エネルギー}
\begin{itembox}[l]{\textbf{def:基準点}}
    各々の$T$に対して、示量変数の組の適当な値$X_0$を固定し、温度$T$での基準点と呼ぶ。このとき、系のスケールを$\lambda$倍すると、基準点も$\lambda$倍され、$\lambda X_0$となる。
\end{itembox}
例えば、系を$\lambda$倍して$N$を$\lambda N$にすると、基準点は、$X_0(T) =(v(TN),N)$から$\lambda X_0(T) = (\lambda v(TN),\lambda N)$になる。\\

\begin{itembox}[l]{\textbf{def:Helmholtz自由エネルギー}}
    任意の温度$T$と、$X_0(T)$から、何らかの操作で到達できる任意の$X$に対して、ヘルムホルツ自由エネルギーを、
    \begin{equation}
        F(T,X) = W_{\text{max}}(T;X_0\rightarrow X)
    \end{equation}
    と定義する。

\end{itembox}
ここで、最大仕事は、状態の始点と終点および温度を定めることで一意に決まるため、ヘルムホルツ自由エネルギーは、温度と示量変数の組$(T,X)$だけで決まる。(すなわち、ヘルムホルツ自由エネルギーは状態量である。)\\

最大仕事の性質より、ヘルムホルツ自由エネルギーには以下の性質がある。\\
\begin{itembox}[l]{\textbf{prop:Helmholtz自由エネルギーの性質}}
    ヘルムホルツ自由エネルギーの性質は以下の通りである。
    \begin{enumerate}
        \item ヘルムホルツ自由エネルギーは、相加性および示量性を持つ。
        \item $F(T,X_1)-F(T,X_2)=W_{\text{max}}(T;X_1\rightarrow X_2)$である。
    \end{enumerate}
\end{itembox}
とくに、二つ目の性質から、熱力学的な系が等温操作により別の状態に移る時、系が外界に為す仕事の最大値は、二つの状態のヘルムホルツ自由エネルギーの差に等しいことがわかる。\\

\textbf{ヘルムホルツ自由エネルギーの気持ち}\\
上で見たように、ヘルムホルツ自由エネルギーの差が、系が外界にする仕事となっている。これと対応する力学的な現象としては、「始状態において粒子が静止しているところに、仕事を加え、終状態においても粒子を静止させる」といった状況に対応する。式として書くと、
\begin{equation}
    W_{in} = V(x_2)-V(x_1)
\end{equation}
といった状況である。ここで、$V(x)$はポテンシャルエネルギーである。\\
これの仕事の向きを反転させると、
\begin{equation}
    W_{out} = V(x_1)-V(x_2)
\end{equation}
となる。これとヘルムホルツ自由エネルギーの性質2を比較すると、
\begin{equation}
    W_{\text{max}}(T;x_1\rightarrow x_2) = F(T,x_1)-F(T,x_2)
\end{equation}
となる。\\
こうしてみると、力学における「ポテンシャルエネルギーによる外への仕事」と、熱力学における自由エネルギー変化は、非常によく似ていることがわかる。\footnote{熱力学特有の「温度」を固定した状況を考えているから力学っぽいのかなーなどと思いながら書いている。}\\
また、ヘルムホルツ自由エネルギーが、「自由に使えるエネルギー」と呼ばれる理由も、まさに外に(自由に)取り出すことのできるエネルギーが、ヘルムホルツ自由エネルギー(の差)であることに由来する。\\

\subsection{Helmholtz自由エネルギーと圧力}
体積をわずかに変化させる等温操作に注目して、圧力を導入する。\\
平衡状態$(T,V,N)$に対して、等温準静操作を施し、体積を$\Delta V$だけ変化させる。すなわち、
\begin{equation}
    (T,V,N) \xrightarrow{iq} (T,V+\Delta V,N)
\end{equation}
を考える。
この操作の間に系が外界に行う仕事は、最大仕事の原理により、
\begin{equation}
    W_{\text{max}}(T;V\rightarrow V+\Delta V) = F(T,V)-F(T,V+\Delta V)
\end{equation}
である。\\
また、この仕事は、力学的な仕事としても書くことができ、
\begin{equation}
    W_{\text{max}}(T;V\rightarrow V+\Delta V) = P\delta V
\end{equation}
と書くことができる。ここで、$P$は系の圧力である。\\
したがって、
\begin{equation}
    P = -\pdv{F}{V}
\end{equation}
となる。ただし、ここで、$P(T,V,N)$が、任意の状態において正の値をとり、$T,V,N$に関して連続的であることを仮定している。\\
ここで、圧力が示強性を持つことを示す。\\
今、ヘルムホルツ自由エネルギーと圧力との関係をもう一度書くと、
\begin{equation}
    P(T,V,N) = -\pdv{F}{V}
\end{equation}
であった。ここで、系のスケールを$\lambda$倍すると、
\begin{align}
    P(T,\lambda V,\lambda N) &= -\pdv{F(T,\lambda V,\lambda N)}{\lambda V}\\
    &= -\frac{\lambda \partial F(T,V,N)}{\lambda \partial V}\\
    &= -\pdv{F(T,V,N)}{V}\\
    &= P(T,V,N)
\end{align}
となる。したがって、圧力は示強性を持つ。\\

また、$N$と$T$が適当な値で固定されているとき、圧力とヘルムホルツ自由エネルギーの関係を積分することで、ヘルムホルツ自由エネルギーが求まる。すなわち、
\begin{equation}
    F(T,V,N) = -\int_{v(T)N}^{V}P(T,V',N)dV'
\end{equation}
である。\\

\section{断熱操作とエネルギー}
\subsection{断熱操作}
\begin{itembox}[l]{\textbf{def:断熱操作}}
    熱力学的な系が平衡状態$(T,X)$にある。この系を断熱壁で囲み、示量変数の組を$X$から$X'$に変化させる。操作の後、断熱壁で囲ったまま十分長い時間放置すると、系はある平衡状態$(T',X')$に達する。この操作を断熱操作という。
    式で書くと、
    \begin{equation}
        (T,X) \xrightarrow{a} (T',X')
    \end{equation}
    となる。
\end{itembox}
操作の過程においては、断熱壁で系をかこっている限りは、どのような操作を行ってもよい。\\

\begin{itembox}[l]{\textbf{prop:断熱準静操作}}
    断熱操作のうち、系がいつも平衡状態にあるような操作を断熱準静操作という。これを、
    \begin{equation}
        (T,X) \xrightarrow{aq} (T',X')
    \end{equation}
    と表す。
\end{itembox}
等温準静操作のときと同様に、断熱準静操作においては、逆向きの操作を行うことで、逆向きの断熱準静操作を行うことができる。\\

\subsection{断熱操作に関する要請}
\begin{itembox}[l]{\textbf{要請4.1:温度を上げる断熱操作の存在}}
    $(T,X)$を任意の平衡状態とする。$T'>T$を満たす任意の温度$T'$に対して、示量変数の組を変化させずに温度を上げる操作、すなわち、
    \begin{equation}
        (T,X) \xrightarrow{a} (T',X)
    \end{equation}
    が存在する。
\end{itembox}
この要請は非常に現実的な要請である。例えば、手のひらをこすり合わせると、摩擦によって温度が上がるといった経験事実が存在する。\\
この要請と、断熱準静操作の性質から、以下のことがわかる。\\
\begin{itembox}[l]{\textbf{結果4.2:断熱操作の存在}}
    示量変数の組$X$から$X'$へ何らかの操作で移ることが可能であるとすると、$T,T'$を任意の温度としたとき、二つの断熱操作
    \begin{align}
        (T,X) \xrightarrow{a} (T',X')\\
        (T',X') \xrightarrow{a} (T,X)
    \end{align}
    の少なくとも一つが存在する。
\end{itembox}
\textbf{Prf}\\
まず、$(T,X)$に断熱準静操作を行い、
\begin{equation}
    (T,X) \xrightarrow{aq} (T'',X')
\end{equation}
とする。ここで、$T'\geq T''$とすると、要請4.1より、
\begin{equation}
    (T'',X') \xrightarrow{a} (T',X')
\end{equation}
という操作が存在する。したがって、
\begin{equation}
    (T,X) \xrightarrow{aq} (T'',X') \xrightarrow{a} (T',X')
\end{equation}
という操作が存在する。\\
また、$T''\geq T'$のときは、$(T',X') $からスタートすることで、
\begin{equation}
    (T',X') \xrightarrow{a} (T'',X') \xrightarrow{aq} (T,X)
\end{equation}
という操作が存在する。\\
以上より、結果4.2が示された。\\

\subsection{熱力学におけるエネルギー保存則と断熱仕事}
等温過程においては、系が外部にする仕事は操作の仕方に依存したが、断熱操作においては系の初めと終わりの状態により一意に決まることが知られている。以下我々はこのことを要請として用いる。\\
\begin{itembox}[l]{\textbf{要請4.3:熱力学におけるエネルギー保存則}}
    任意の断熱操作の間に熱力学的な系が外界に行う仕事は、はじめの平衡状態と最終的な平衡状態だけで決まり、操作方法や過程には依存しない。
\end{itembox}
以上の要請を踏まえると、ある断熱操作$(T,X) \xrightarrow{a} (T',X')$において、系が外界に行う仕事は、$(T,X)$と$(T',X')$だけで決まる。この時の仕事を断絶仕事という。\\
\begin{itembox}[l]{\textbf{def:断熱仕事}}
    熱力学的な系が平衡状態$(T,X)$にある。この系を断熱壁で囲み、示量変数の組を$X$から$X'$に変化させる。操作の後、断熱壁で囲ったまま十分長い時間放置すると、系はある平衡状態$(T',X')$に達する。この操作において系が外界に行う仕事を断熱仕事という。\\
    この断熱仕事を、$W_{\text{ad}}((T,X)\rightarrow (T',X'))$と書く。
\end{itembox}
断熱仕事は、最大仕事と同様の性質を示す。\\
\begin{itembox}[l]{\textbf{prop:断熱仕事の性質}}
    断熱仕事の性質は以下の通りである。
    \begin{enumerate}
        \item 断熱操作を逆向きに行うときに系が外界に行う仕事は、$W_{\text{ad}}((T',X')\rightarrow (T,X)) = -W_{\text{ad}}((T,X)\rightarrow (T',X'))$である。
        \item 断熱仕事に結合率が成り立つ。すなわち、$W_{\text{ad}}((T,X)\rightarrow (T',X'))+W_{\text{ad}}((T',X')\rightarrow (T'',X'')) \geq W_{\text{ad}}((T,X)\rightarrow (T'',X''))$である。
        \item 断熱仕事には相加性がある。すなわち、$W_{\text{ad}}((T,X)\rightarrow (T',X'))+W_{\text{ad}}((T',X')\rightarrow (T'',X'')) = W_{\text{ad}}((T,X)\rightarrow (T'',X''))$である。
        \item 断熱仕事には示量性が成り立つ。
    \end{enumerate} 
\end{itembox}
\textbf{Prf}\\
あとで書く。\\

\subsection{断熱仕事と内部エネルギー}
最大仕事から、ヘルムホルツ自由エネルギーを導入することができたように、断熱仕事から内部エネルギーを導入することができる。\footnote{ただし、最大仕事のときは、準静操作を用いて定義していたのに対し、今回は準静的とは限らない断熱操作を用いて定義するので、結果4.2の場合分けが必要になってしまう。}\\
\begin{itembox}[l]{\textbf{def:基準点}}
    基準の温度$T^*$と示量変数の組$X^*$を適当うに定め、基準点と呼ぶ。このとき、系のスケールを$\lambda$倍すると、基準点も$\lambda$倍され、$\lambda X^*$となる。
\end{itembox}
結果4.2より、任意の温度$T$に対して、$(T,X)\xrightarrow{a} (T^*,X^*)$か$(T^*,X^*)\xrightarrow{a} (T,X)$のどちらかの断熱操作が存在する。このことを用いて内部エネルギーを定める。\\
\begin{itembox}[l]{\textbf{def:内部エネルギー}}
上で述べた操作のうち、一つ目の操作が可能なとき、
\begin{equation}
    U(T,X) = W_{\text{ad}}((T,X)\rightarrow (T^*,X^*))
\end{equation}
と定義する。もし、二つ目の操作が可能なときは、
\begin{equation}
    U(T,X) = -W_{\text{ad}}((T^*,X^*)\rightarrow (T,X))
\end{equation}
と定義する。\\
また、二つの操作がともに可能なときは、二つの定義は一致する。
\end{itembox}
ただし、ここで、温度や示量変数を微小変化させるのに必要な仕事は小さいはずなので、内部エネルギーは$T$や$X$について連続であることを要請する。\\



\end{document}