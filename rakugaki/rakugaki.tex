\documentclass[a4paper,11pt]{jsarticle}

% 数式
\usepackage{amsmath,amsfonts}
\usepackage{amsthm}
\usepackage{bm}
\usepackage{mathtools}
\usepackage{amssymb}

% 表
\usepackage[utf8]{inputenc}
\usepackage{diagbox} % 斜線付きセルを作成するために必要
\usepackage{booktabs} % 表の罫線を美しくするために必要
\usepackage{hhline} % 水平罫線を制御するために必要

% 画像
\usepackage[dvipdfmx]{graphicx}
\usepackage{ascmac}
\usepackage{physics}
\usepackage{float} % 追加

% 図
\usepackage[dvipdfmx]{graphicx}
\usepackage{tikz} %図を描く
\usetikzlibrary{positioning, intersections, calc, arrows.meta,math} %tikzのlibrary

% ハイパーリンク
\usepackage[dvipdfm,
  colorlinks=false,
  bookmarks=true,
  bookmarksnumbered=false,
  pdfborder={0 0 0},
  bookmarkstype=toc]{hyperref}

% 式番号を章ごとにリセット
\numberwithin{equation}{section}

\begin{document}

\title{rakugaki}
\author{大上由人}
\date{\today}
\maketitle

\begin{itembox}[l]{\textbf{Lem:}}
  $ A \underset{\text{cofibration}}{\subset} X$であるならば、$X/A \underset{\text{cofibration}}{\supset}A/A$
\end{itembox}

\begin{itembox}[l]{\textbf{Thm:等価空間のオイラー数}}
  $\phi \neq A \underset{\text{cofibration}}{\subset}X$かつ$\mathcal{X}(A), \mathcal{X}(X), \mathcal{X}(X/A)$のどれか二つがwell-defined

\end{itembox}
\textbf{Prf}\\
$X \cup CA =M_{\iota}/A\times \{0\}$\\
$X(X \cup CA) = X(D) + X(E) -X(D \cap E)$\\
右辺第一項は可縮である、第二項は$E\simeq X$であり、$D\cap E \equiv A \times (\frac{1}{3},\frac{2}{3}) \simeq S$\\
$(X \cup CA)\sqcup_{p} {*} \cong X/A$\\
左辺$\simeq X \cup CA$\\
よって、求めるものを得る。\\

$Top_0$起点つき位相空間の圏(射は連続写像であって、基点を基点に移すもの)\\
$ X \in Obj(Top_0)$について、
\begin{equation}
  \tilde{X}(X) = X(X) -1(X(\{*\}))
\end{equation}
と定義する。\\

\begin{itembox}[l]{\textbf{Cor}}
  $A,X \in Obj(Top_0)$かつ$A \subset X$について(つまり、$X$の基点が$A$の基点に移る)、
  \begin{equation}
    \tilde{X}(X/A) = \tilde{X}(X) - \tilde{X}(A)
  \end{equation}
  ただし、3項のうちどれか二つがwell-definedである。
\end{itembox}
$A,B \in Obj(Top_0)$について、
\begin{equation}
  A \vee B := A \sqcup B / \{A^* \sim B^*\}
\end{equation}
\begin{equation}
  A \wedge B := A \times B / \{A \times \{*_B\} \cup \{*_A\} \times B\}
\end{equation}
と定義する。\\

\begin{itembox}[l]{\textbf{Thm:$\tilde{\mathcal{X}}$の性質}}
  すべて基点つき空間で考える。$A,B \in X \text{s.t.} X = A \cup B$かつ、
  \begin{equation}
    A \underset{\text{cofibration}}{\supset}A \cap B 
  \end{equation}
  かつ
  \begin{equation}
    A \cap B \underset{\text{cofibration}}{\subset} B 
  \end{equation}
  であるとする。このとき、$\mathcal{X}(A), \mathcal{X}(B), \mathcal{X}(X), \mathcal{X}(A \cap B)$のうちどれか三つがwell-definedであるならヴぁ
  \begin{equation}
    \tilde{X}(X) = \tilde{X}(A) + \tilde{X}(B) - \tilde{X}(A \cap B)
  \end{equation}
  が成り立つ。
  

\end{itembox}
\textbf{Prf}\\
仮定より、
\begin{equation}
  A \underset{\text{cofibration}}{\subset} X \underset{\text{cofibration}}{\supset} A
\end{equation}
\begin{equation}
  X/A \cong B/A \cap B \quad X/B \cong A/A \cap B
\end{equation}
先のことから、
\begin{equation}
  \tilde{\mathcal{X}}(X/A) = \tilde{\mathcal{X}}(X) - \tilde{\mathcal{X}}(A)
\end{equation}
\begin{equation}
  \tilde{\mathcal{X}}(X/A) = \tilde{\mathcal{X}}(B/A \cap B) = \tilde{\mathcal{X}}(B) - \tilde{\mathcal{X}}(A \cap B)
\end{equation}
\begin{equation}
  \tilde{\mathcal{X}}(X/B) = \tilde{\mathcal{X}}(X) - \tilde{\mathcal{X}}(B)
\end{equation}
\begin{equation}
  \tilde{\mathcal{X}}(A/A \cap B) = \tilde{\mathcal{X}}(A) - \tilde{\mathcal{X}}(A \cap B)
\end{equation}
よって、
\begin{equation}
  \tilde{\mathcal{X}}(X) - \tilde{\mathcal{X}}(A) = \tilde{\mathcal{X}}(B) - \tilde{\mathcal{X}}(A \cap B) \label{eq:1}
\end{equation}
\begin{equation}
  \tilde{\mathcal{X}}(X) - \tilde{\mathcal{X}}(B) = \tilde{\mathcal{X}}(A) - \tilde{\mathcal{X}}(A \cap B) \label{eq:2}
\end{equation}
であるから、(\ref{eq:1})-(\ref{eq:2})を辺々足して2で割ることで、求めるものを得る。\\

\begin{itembox}[l]{\textbf{Def:退化}}
  $A \in Obj(Top_0)$について、$A$の基点$\{*_A\} \in A $が非退化であるとは、($A$がwell-pointedであるとは)、
  $\{*_A\} \in A$がcofibrationであることをいう。
\end{itembox}

\begin{itembox}[l]{\textbf{Prop:}}
  $A,B$がwell-pointedであるならば、$A \wedge B, A \vee B$もwell-pointedである。
\end{itembox}

\begin{itembox}[l]{\textbf{Cor}}
  $A,B \in Obj(Top_0)$がwell-pointedであるかつ、$\tilde{\mathcal{X}}(A), \tilde{\mathcal{X}}(B)$がwell-definedであるならば、
  \begin{align}
    \tilde{\mathcal{X}}(A \wedge B) &= \tilde{\mathcal{X}}(A) + \tilde{\mathcal{X}}(B) -1\\
    \tilde{\mathcal{X}}(A \vee B) &= \tilde{\mathcal{X}}(A) + \tilde{\mathcal{X}}(B) -1
  \end{align}
  が成り立つ。
\end{itembox}
\textbf{Prf}\\
$A \subset A \vee B \supset B$と思う。
仮定より、$A \cap B \in A \vee B = A \cup B$であり、
\begin{equation}
  A \underset{\text{cofibration}}{\subset} A \cap B(=\{*_A\} = \{*_B\}) \underset{\text{cofibration}}{\subset} B
\end{equation}
\begin{equation}
  A \underset{\text{cofibration}}{\subset} A \cup B(\equiv A \vee B) \underset{\text{cofibration}}{\supset} B
\end{equation}
前の定理より、
\begin{equation}
  \tilde{\mathcal{X}}(A \vee B) = \tilde{\mathcal{X}}(A) + \tilde{\mathcal{X}}(B) - \tilde{\mathcal{X}}(A \cap B)
\end{equation}
である。いま、第三項が0であることに注意すると、求めるものを得る。\\
また、
\begin{equation}
  A \wedge B = A \times B / \{A \vee B\} 
\end{equation}
である、このとき、$A = A \times \{*_B\} \subset A \times B \supset \{*_A\} \times B = B$であり、$A$と$B$の$A \times B$における合併は$A \vee B$と同じ。\\
$A,B$がwell-pointedであることから、$A \vee B \underset{\text{cofibration}}{\subset} A \times B$である。\\
2つ前の定理より、
\begin{align}
  \tilde{\mathcal{X}}(A \wedge B) &= \tilde{\mathcal{X}}(A \times B) / \tilde{\mathcal{X}}(A \vee B)\\
  &= \tilde{\mathcal{X}}(A \times B) - \tilde{\mathcal{X}}(A \vee B)\\
  &= ({\mathcal{X}}(A) {\mathcal{X}}(B) - 1) - ({\mathcal{X}}(A) + {\mathcal{X}}(B) -2)\\
  &= ({\mathcal{X}}(A)-1)(\mathcal{X}(B)-1)\\
  &= (\tilde{\mathcal{X}}(A)(\tilde{\mathcal{X}}(B)))
\end{align}
となる。\\

\textbf{ex}\\
$Top_0$におけるsuspensionを、$A \in Obj(Top_0)$について、
\begin{equation}
  SA = S^1 \wedge A
\end{equation}
と定義する。\\
\begin{equation}
  S^n A = S(S(\cdots S(SA)\cdots))
\end{equation}
と定義する。\\
このとき、
\begin{equation}
  S^n \cong S^n S^0
\end{equation}
である。\\
\begin{align}
  \tilde{\mathcal{X}}(S^n) &= \tilde{\mathcal{X}}(S^n S^0)\\
  &= \tilde{\mathcal{X}}(S^1 \wedge \cdots \wedge S^1 \wedge S^0)\\
  &= (\tilde{\mathcal{X}}(S^1) )^n \tilde{\mathcal{X}}(S^0)\\
  &\because \\
  & \tilde{\mathcal{X}}(S^0) = \mathcal{X}(S^0) - 1 = 2-1 = 1\\
  & \tilde{\mathcal{X}}(S^1) = \mathcal{X}(S^1) - 1 = 0-1 = -1\\
  & \tilde{\mathcal{X}}(S^n) = (-1)^n
\end{align}

\begin{itembox}[l]{\textbf{Def:位相多様体}}
  $M$が位相多様体であるとは、
  \begin{enumerate}
    \item $M$はHausdorffである
    \item 任意の点$p \in M$について、$p$の近傍$U$であって、
    \begin{equation}
      (U,p) \cong (\mathbb{R}^n,0) \label{eq:3}
    \end{equation}
    または、
    \begin{equation}
      (U,p) \cong (H^n_+,0) \label{eq:4}
    \end{equation}
    ただし、
    \begin{equation}
      H^n_+ = \{x \in \mathbb{R}^n | x_n \geq 0\}
    \end{equation}
  \end{enumerate}
  となるものをいう。このとき、(\ref{eq:3})と(\ref{eq:4})は排反事象である。
      
\end{itembox}

\begin{itembox}[l]{\textbf{Def:}}
  位相多様体が閉であるとは、
  \begin{enumerate}
    \item $M$はコンパクト
    \item $\partial M = \varnothing$
    \end{enumerate}
    となるものをいう。
\end{itembox}

$\partial M = \text{$M$の境界点}$とおき、これを$M$の境界という。\\

\textbf{ex}\\
$\mathbb{R}^n, H^n_+$はそれぞれn次元多様体であり、
\begin{equation}
  \partial \mathbb{R}^n = \varnothing, \quad \partial H^n_+ = \mathbb{R}^{n-1}
\end{equation}
である。\\

\textbf{ex}\\
$S^n$は閉多様体\\

\textbf{ex}\\
$A,B$が多様体ならば、$A \times B$も閉多様体である。\\

\section{実験6.2の計算}
今、RCローパスフィルター1段での入力$(v_1,i_1)$と出力$(v_2,i_2)$の関係はF行列を用いて、
\begin{equation}
  \begin{pmatrix}
    v_1\\
    i_1
  \end{pmatrix}
  =
  \begin{pmatrix}
    1+i\omega RC & R\\
    i\omega C & 1
  \end{pmatrix}
  \begin{pmatrix}
    v_2\\
    i_2
  \end{pmatrix}
\end{equation}
と表される。\\
これを3段噛ませたものが、位相$\pi$だけ反転してくれればよい。今、3段噛ませる前後での関係は、F行列を用いて、
\begin{equation}
  \begin{pmatrix}
    v_1\\
    i_1
  \end{pmatrix}
  =
  \begin{pmatrix}
    1+i\omega RC & R\\
    i\omega C & 1
  \end{pmatrix}^3
  \begin{pmatrix}
    v_2\\
    0
  \end{pmatrix}
\end{equation}
と表される。ここで、計算の便宜のために、$t =1+i\omega RC = 1+is$とおくと、
\begin{equation}
  F^3 =
  \begin{pmatrix}
    t^3+2t^2-t & O\\
    O & O
  \end{pmatrix}
\end{equation}
となる。ただし、$O$は適当な数である。したがって、
\begin{equation}
  \begin{pmatrix}
    v_1\\
    i_1
  \end{pmatrix}
  =
  \begin{pmatrix}
    t^3+2t^2-t & O\\
    O & O
  \end{pmatrix}
  \begin{pmatrix}
    v_2\\
    0
  \end{pmatrix}
\end{equation}
となる。電圧について計算することにより、
\begin{equation}
  v_1 = (t^3+2t^2-t)v_2
\end{equation}
となる。ここで、位相差が$\pi$であるためには、
\begin{equation}
  \Im(t^3+2t^2-t) = 0
\end{equation}
である必要がある。
\begin{align}
  \Im(t^3+2t^2-t) &= \Im((1+is)^3+2(1+is)^2-(1+is))\\
  &= -s^3+6s
  &= 0
\end{align}
よって、$s = \sqrt{6}$である。いま、$s = \omega RC$であるから、
\begin{equation}
  \omega = \frac{\sqrt{6}}{RC}
\end{equation}
である。\\

\section{実験6.3の計算}
今、正入力と出力の関係は、
\begin{equation}
  V_{+} = \frac{Z_1}{Z_1+Z_2}V_{\text{out}}
\end{equation}
である。ここで、
\begin{align} 
  Z_1 &= R_1 + \frac{1}{i\omega C_1}\\
  Z_2 &= R_2 + \frac{1}{i\omega C_2}        
\end{align}


\section*{カノニカル分布の裏づけ}

注目する部分系 $S$ と熱浴 $B$ が弱く相互作用しているとする。全系のハミルトニアンは
\begin{equation}
\mathcal{H} = \mathcal{H}_S + \mathcal{H}_B + \mathcal{H}_{\text{int}}.
\end{equation}
ここで、相互作用のオーダーは、$o(V^\frac{2}{3})$なので、無視できるとする。全系の状態は $S$ の状態を表す Hilbert 空間と $B$ の状態を表す Hilbert 空間の直積で表され
\begin{equation}
\mathcal{H}\ket{\Psi_k} = (\mathcal{H}_S + \mathcal{H}_B)\ket{\psi_i}\ket{\phi_j} = E_k \ket{\Psi_k}
\end{equation}
である。ここで \(\ket{\psi_i} \ket{\phi_j}\) は \(\mathcal{H}_S \ket{\psi_i} = E^S_i \ket{\psi_i}\) と \(\mathcal{H}_B \ket{\phi_j} = E^B_j \ket{\phi_j}\) を満たす系 \(S\) と熱浴 \(B\) の固有状態で、\(E_k = E^S_i + E^B_j\), \(\ket{\Psi_k} = \ket{\psi_i} \ket{\phi_j}\) である。全系がミクロカノニカル分布で記述されるとすると、
\begin{equation}
\hat{\rho}_{U, \delta U} = \frac{1}{W(U, \delta U)} \sum_{U-\delta U \leq E_k \leq U} \ket{\Psi_k} \bra{\Psi_k}.
\end{equation}
である。ここで \(W(U,\delta U)\) は \([U-\delta U, U]\) にある固有状態数である。熱浴について対角和をとると
\begin{align}
\hat{\rho}^S &= \text{Tr}_B \hat{\rho}_{U, \delta U} \\
&= \frac{1}{W(U, \delta U)} \sum_{U-\delta U \leq E_k \leq U} \sum_l \bra{\phi_l} \ket{\Psi_k} \bra{\Psi_k} \ket{\phi_l} \\
&= \frac{1}{W(U, \delta U)} \sum_{U-\delta U -E_i^S \leq  E_j^B \leq U-E_i^s} \sum_l \sum_{i} \bra{\phi_l} \ket{\phi_j} \ket{\psi_i} \bra{\psi_i} \bra{\phi_j} \ket{\phi_l} \\
&= \frac{1}{W(U, \delta U)} \sum_i \ket{\psi_i} \bra{\psi_i} \sum_{i} \sum_{U - \delta U -E_i^S \leq E_j^B \leq U - E_i^S} 1 \\
&= \frac{1}{W(U, \delta U)} \sum_i W^B(U - E_i^S, \delta U) \ket{\psi_i} \bra{\psi_i}.
\end{align}
となる。状態数とエントロピーの関係 \(S(E) = k_B \ln W(U, \Delta U)\) から
\begin{align}
W^B(U - E_i^S) &= e^{S^B(U - E_i^S)/k_B}\\ 
&\approx \exp\left[ \frac{1}{k_B} \left( S^B(U) - \frac{d S^B(U)}{dU} E_i^S \right) \right] \\
&= e^{S^B(U)/k_B - \beta E_i^S} \\
&\propto e^{-\beta E_i^S}
\end{align}
となる。ただし、$E_i^S\ll U$ を用いた。したがって、規格化を考慮して、
\begin{align}
  \hat{\rho}^S &\approx \frac{\sum_i e^{-\beta E_i^S} \ket{\psi_i} \bra{\psi_i}}{\sum_i e^{-\beta E_i^S}} \\
  &= \frac{e^{-\beta \mathcal{H}_S}}{\text{Tr} e^{-\beta \mathcal{H}_S}}
\end{align}
となる。ここで、
\begin{equation}
  \beta = \frac{1}{k_B}\dv{S^B}{U} = \frac{1}{k_B T}
\end{equation}
である。ただし、\(T\) は熱浴の温度である。以上より、系 \(S\) はカノニカル分布で記述される。


\end{document}