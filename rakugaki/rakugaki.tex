\documentclass[a4paper,11pt]{jsarticle}

% 数式
\usepackage{amsmath,amsfonts}
\usepackage{amsthm}
\usepackage{bm}
\usepackage{mathtools}
\usepackage{amssymb}

% 表
\usepackage[utf8]{inputenc}
\usepackage{diagbox} % 斜線付きセルを作成するために必要
\usepackage{booktabs} % 表の罫線を美しくするために必要
\usepackage{hhline} % 水平罫線を制御するために必要

% 画像
\usepackage[dvipdfmx]{graphicx}
\usepackage{ascmac}
\usepackage{physics}
\usepackage{float} % 追加

% 図
\usepackage[dvipdfmx]{graphicx}
\usepackage{tikz} %図を描く
\usetikzlibrary{positioning, intersections, calc, arrows.meta,math} %tikzのlibrary

% ハイパーリンク
\usepackage[dvipdfm,
  colorlinks=false,
  bookmarks=true,
  bookmarksnumbered=false,
  pdfborder={0 0 0},
  bookmarkstype=toc]{hyperref}

% 式番号を章ごとにリセット
\numberwithin{equation}{section}

\begin{document}

\title{rakugaki}
\author{大上由人}
\date{\today}
\maketitle

\begin{itembox}[l]{\textbf{Lem:}}
  $ A \underset{\text{cofibration}}{\subset} X$であるならば、$X/A \underset{\text{cofibration}}{\supset}A/A$
\end{itembox}

\begin{itembox}[l]{\textbf{Thm:等価空間のオイラー数}}
  $\phi \neq A \underset{\text{cofibration}}{\subset}X$かつ$\mathcal{X}(A), \mathcal{X}(X), \mathcal{X}(X/A)$のどれか二つがwell-defined

\end{itembox}
\textbf{Prf}\\
$X \cup CA =M_{\iota}/A\times \{0\}$\\
$X(X \cup CA) = X(D) + X(E) -X(D \cap E)$\\
右辺第一項は可縮である、第二項は$E\simeq X$であり、$D\cap E \equiv A \times (\frac{1}{3},\frac{2}{3}) \simeq S$\\
$(X \cup CA)\sqcup_{p} {*} \cong X/A$\\
左辺$\simeq X \cup CA$\\
よって、求めるものを得る。\\

$Top_0$起点つき位相空間の圏(射は連続写像であって、基点を基点に移すもの)\\
$ X \in Obj(Top_0)$について、
\begin{equation}
  \tilde{X}(X) = X(X) -1(X(\{*\}))
\end{equation}
と定義する。\\

\begin{itembox}[l]{\textbf{Cor}}
  $A,X \in Obj(Top_0)$かつ$A \subset X$について(つまり、$X$の基点が$A$の基点に移る)、
  \begin{equation}
    \tilde{X}(X/A) = \tilde{X}(X) - \tilde{X}(A)
  \end{equation}
  ただし、3項のうちどれか二つがwell-definedである。
\end{itembox}
$A,B \in Obj(Top_0)$について、
\begin{equation}
  A \vee B := A \sqcup B / \{A^* \sim B^*\}
\end{equation}
\begin{equation}
  A \wedge B := A \times B / \{A \times \{*_B\} \cup \{*_A\} \times B\}
\end{equation}
と定義する。\\

\begin{itembox}[l]{\textbf{Thm:$\tilde{\mathcal{X}}$の性質}}
  すべて基点つき空間で考える。$A,B \in X \text{s.t.} X = A \cup B$かつ、
  \begin{equation}
    A \underset{\text{cofibration}}{\supset}A \cap B 
  \end{equation}
  かつ
  \begin{equation}
    A \cap B \underset{\text{cofibration}}{\subset} B 
  \end{equation}
  であるとする。このとき、$\mathcal{X}(A), \mathcal{X}(B), \mathcal{X}(X), \mathcal{X}(A \cap B)$のうちどれか三つがwell-definedであるならヴぁ
  \begin{equation}
    \tilde{X}(X) = \tilde{X}(A) + \tilde{X}(B) - \tilde{X}(A \cap B)
  \end{equation}
  が成り立つ。
  

\end{itembox}
\textbf{Prf}\\
仮定より、
\begin{equation}
  A \underset{\text{cofibration}}{\subset} X \underset{\text{cofibration}}{\supset} A
\end{equation}
\begin{equation}
  X/A \cong B/A \cap B \quad X/B \cong A/A \cap B
\end{equation}
先のことから、
\begin{equation}
  \tilde{\mathcal{X}}(X/A) = \tilde{\mathcal{X}}(X) - \tilde{\mathcal{X}}(A)
\end{equation}
\begin{equation}
  \tilde{\mathcal{X}}(X/A) = \tilde{\mathcal{X}}(B/A \cap B) = \tilde{\mathcal{X}}(B) - \tilde{\mathcal{X}}(A \cap B)
\end{equation}
\begin{equation}
  \tilde{\mathcal{X}}(X/B) = \tilde{\mathcal{X}}(X) - \tilde{\mathcal{X}}(B)
\end{equation}
\begin{equation}
  \tilde{\mathcal{X}}(A/A \cap B) = \tilde{\mathcal{X}}(A) - \tilde{\mathcal{X}}(A \cap B)
\end{equation}
よって、
\begin{equation}
  \tilde{\mathcal{X}}(X) - \tilde{\mathcal{X}}(A) = \tilde{\mathcal{X}}(B) - \tilde{\mathcal{X}}(A \cap B) \label{eq:1}
\end{equation}
\begin{equation}
  \tilde{\mathcal{X}}(X) - \tilde{\mathcal{X}}(B) = \tilde{\mathcal{X}}(A) - \tilde{\mathcal{X}}(A \cap B) \label{eq:2}
\end{equation}
であるから、(\ref{eq:1})-(\ref{eq:2})を辺々足して2で割ることで、求めるものを得る。\\

\begin{itembox}[l]{\textbf{Def:退化}}
  $A \in Obj(Top_0)$について、$A$の基点$\{*_A\} \in A $が非退化であるとは、($A$がwell-pointedであるとは)、
  $\{*_A\} \in A$がcofibrationであることをいう。
\end{itembox}

\begin{itembox}[l]{\textbf{Prop:}}
  $A,B$がwell-pointedであるならば、$A \wedge B, A \vee B$もwell-pointedである。
\end{itembox}

\begin{itembox}[l]{\textbf{Cor}}
  $A,B \in Obj(Top_0)$がwell-pointedであるかつ、$\tilde{\mathcal{X}}(A), \tilde{\mathcal{X}}(B)$がwell-definedであるならば、
  \begin{align}
    \tilde{\mathcal{X}}(A \wedge B) &= \tilde{\mathcal{X}}(A) + \tilde{\mathcal{X}}(B) -1\\
    \tilde{\mathcal{X}}(A \vee B) &= \tilde{\mathcal{X}}(A) + \tilde{\mathcal{X}}(B) -1
  \end{align}
  が成り立つ。
\end{itembox}
\textbf{Prf}\\
$A \subset A \vee B \supset B$と思う。
仮定より、$A \cap B \in A \vee B = A \cup B$であり、
\begin{equation}
  A \underset{\text{cofibration}}{\subset} A \cap B(=\{*_A\} = \{*_B\}) \underset{\text{cofibration}}{\subset} B
\end{equation}
\begin{equation}
  A \underset{\text{cofibration}}{\subset} A \cup B(\equiv A \vee B) \underset{\text{cofibration}}{\supset} B
\end{equation}
前の定理より、
\begin{equation}
  \tilde{\mathcal{X}}(A \vee B) = \tilde{\mathcal{X}}(A) + \tilde{\mathcal{X}}(B) - \tilde{\mathcal{X}}(A \cap B)
\end{equation}
である。いま、第三項が0であることに注意すると、求めるものを得る。\\
また、
\begin{equation}
  A \wedge B = A \times B / \{A \vee B\} 
\end{equation}
である、このとき、$A = A \times \{*_B\} \subset A \times B \supset \{*_A\} \times B = B$であり、$A$と$B$の$A \times B$における合併は$A \vee B$と同じ。\\
$A,B$がwell-pointedであることから、$A \vee B \underset{\text{cofibration}}{\subset} A \times B$である。\\
2つ前の定理より、
\begin{align}
  \tilde{\mathcal{X}}(A \wedge B) &= \tilde{\mathcal{X}}(A \times B) / \tilde{\mathcal{X}}(A \vee B)\\
  &= \tilde{\mathcal{X}}(A \times B) - \tilde{\mathcal{X}}(A \vee B)\\
  &= ({\mathcal{X}}(A) {\mathcal{X}}(B) - 1) - ({\mathcal{X}}(A) + {\mathcal{X}}(B) -2)\\
  &= ({\mathcal{X}}(A)-1)(\mathcal{X}(B)-1)\\
  &= (\tilde{\mathcal{X}}(A)(\tilde{\mathcal{X}}(B)))
\end{align}
となる。\\

\textbf{ex}\\
$Top_0$におけるsuspensionを、$A \in Obj(Top_0)$について、
\begin{equation}
  SA = S^1 \wedge A
\end{equation}
と定義する。\\
\begin{equation}
  S^n A = S(S(\cdots S(SA)\cdots))
\end{equation}
と定義する。\\
このとき、
\begin{equation}
  S^n \cong S^n S^0
\end{equation}
である。\\
\begin{align}
  \tilde{\mathcal{X}}(S^n) &= \tilde{\mathcal{X}}(S^n S^0)\\
  &= \tilde{\mathcal{X}}(S^1 \wedge \cdots \wedge S^1 \wedge S^0)\\
  &= (\tilde{\mathcal{X}}(S^1) )^n \tilde{\mathcal{X}}(S^0)\\
  &\because \\
  & \tilde{\mathcal{X}}(S^0) = \mathcal{X}(S^0) - 1 = 2-1 = 1\\
  & \tilde{\mathcal{X}}(S^1) = \mathcal{X}(S^1) - 1 = 0-1 = -1\\
  & \tilde{\mathcal{X}}(S^n) = (-1)^n
\end{align}

\begin{itembox}[l]{\textbf{Def:位相多様体}}
  $M$が位相多様体であるとは、
  \begin{enumerate}
    \item $M$はHausdorffである
    \item 任意の点$p \in M$について、$p$の近傍$U$であって、
    \begin{equation}
      (U,p) \cong (\mathbb{R}^n,0) \label{eq:3}
    \end{equation}
    または、
    \begin{equation}
      (U,p) \cong (H^n_+,0) \label{eq:4}
    \end{equation}
    ただし、
    \begin{equation}
      H^n_+ = \{x \in \mathbb{R}^n | x_n \geq 0\}
    \end{equation}
  \end{enumerate}
  となるものをいう。このとき、(\ref{eq:3})と(\ref{eq:4})は排反事象である。
      
\end{itembox}

\begin{itembox}[l]{\textbf{Def:}}
  位相多様体が閉であるとは、
  \begin{enumerate}
    \item $M$はコンパクト
    \item $\partial M = \varnothing$
    \end{enumerate}
    となるものをいう。
\end{itembox}

$\partial M = \text{$M$の境界点}$とおき、これを$M$の境界という。\\

\textbf{ex}\\
$\mathbb{R}^n, H^n_+$はそれぞれn次元多様体であり、
\begin{equation}
  \partial \mathbb{R}^n = \varnothing, \quad \partial H^n_+ = \mathbb{R}^{n-1}
\end{equation}
である。\\

\textbf{ex}\\
$S^n$は閉多様体\\

\textbf{ex}\\
$A,B$が多様体ならば、$A \times B$も閉多様体である。\\

\section{実験6.2の計算}
今、RCローパスフィルター1段での入力$(v_1,i_1)$と出力$(v_2,i_2)$の関係はF行列を用いて、
\begin{equation}
  \begin{pmatrix}
    v_1\\
    i_1
  \end{pmatrix}
  =
  \begin{pmatrix}
    1+i\omega RC & R\\
    i\omega C & 1
  \end{pmatrix}
  \begin{pmatrix}
    v_2\\
    i_2
  \end{pmatrix}
\end{equation}
と表される。\\
これを3段噛ませたものが、位相$\pi$だけ反転してくれればよい。今、3段噛ませる前後での関係は、F行列を用いて、
\begin{equation}
  \begin{pmatrix}
    v_1\\
    i_1
  \end{pmatrix}
  =
  \begin{pmatrix}
    1+i\omega RC & R\\
    i\omega C & 1
  \end{pmatrix}^3
  \begin{pmatrix}
    v_2\\
    0
  \end{pmatrix}
\end{equation}
と表される。ここで、計算の便宜のために、$t =1+i\omega RC = 1+is$とおくと、
\begin{equation}
  F^3 =
  \begin{pmatrix}
    t^3+2t^2-t & O\\
    O & O
  \end{pmatrix}
\end{equation}
となる。ただし、$O$は適当な数である。したがって、
\begin{equation}
  \begin{pmatrix}
    v_1\\
    i_1
  \end{pmatrix}
  =
  \begin{pmatrix}
    t^3+2t^2-t & O\\
    O & O
  \end{pmatrix}
  \begin{pmatrix}
    v_2\\
    0
  \end{pmatrix}
\end{equation}
となる。電圧について計算することにより、
\begin{equation}
  v_1 = (t^3+2t^2-t)v_2
\end{equation}
となる。ここで、位相差が$\pi$であるためには、
\begin{equation}
  \Im(t^3+2t^2-t) = 0
\end{equation}
である必要がある。
\begin{align}
  \Im(t^3+2t^2-t) &= \Im((1+is)^3+2(1+is)^2-(1+is))\\
  &= -s^3+6s
  &= 0
\end{align}
よって、$s = \sqrt{6}$である。いま、$s = \omega RC$であるから、
\begin{equation}
  \omega = \frac{\sqrt{6}}{RC}
\end{equation}
である。\\

\section{実験6.3の計算}
今、正入力と出力の関係は、
\begin{equation}
  V_{+} = \frac{Z_1}{Z_1+Z_2}V_{\text{out}}
\end{equation}
である。ここで、
\begin{align} 
  Z_1 &= R_1 + \frac{1}{i\omega C_1}\\
  Z_2 &= R_2 + \frac{1}{i\omega C_2}        
\end{align}


\section*{カノニカル分布の裏づけ}

注目する部分系 $S$ と熱浴 $B$ が弱く相互作用しているとする。全系のハミルトニアンは
\begin{equation}
\mathcal{H} = \mathcal{H}_S + \mathcal{H}_B + \mathcal{H}_{\text{int}}.
\end{equation}
ここで、相互作用のオーダーは、$o(V^\frac{2}{3})$なので、無視できるとする。全系の状態は $S$ の状態を表す Hilbert 空間と $B$ の状態を表す Hilbert 空間の直積で表され
\begin{equation}
\mathcal{H}\ket{\Psi_k} = (\mathcal{H}_S + \mathcal{H}_B)\ket{\psi_i}\ket{\phi_j} = E_k \ket{\Psi_k}
\end{equation}
である。ここで \(\ket{\psi_i} \ket{\phi_j}\) は \(\mathcal{H}_S \ket{\psi_i} = E^S_i \ket{\psi_i}\) と \(\mathcal{H}_B \ket{\phi_j} = E^B_j \ket{\phi_j}\) を満たす系 \(S\) と熱浴 \(B\) の固有状態で、\(E_k = E^S_i + E^B_j\), \(\ket{\Psi_k} = \ket{\psi_i} \ket{\phi_j}\) である。全系がミクロカノニカル分布で記述されるとすると、
\begin{equation}
\hat{\rho}_{U, \delta U} = \frac{1}{W(U, \delta U)} \sum_{U-\delta U \leq E_k \leq U} \ket{\Psi_k} \bra{\Psi_k}.
\end{equation}
である。ここで \(W(U,\delta U)\) は \([U-\delta U, U]\) にある固有状態数である。熱浴について対角和をとると
\begin{align}
\hat{\rho}^S &= \text{Tr}_B \hat{\rho}_{U, \delta U} \\
&= \frac{1}{W(U, \delta U)} \sum_{U-\delta U \leq E_k \leq U} \sum_l \bra{\phi_l} \ket{\Psi_k} \bra{\Psi_k} \ket{\phi_l} \\
&= \frac{1}{W(U, \delta U)} \sum_{U-\delta U -E_i^S \leq  E_j^B \leq U-E_i^s} \sum_l \sum_{i} \bra{\phi_l} \ket{\phi_j} \ket{\psi_i} \bra{\psi_i} \bra{\phi_j} \ket{\phi_l} \\
&= \frac{1}{W(U, \delta U)} \sum_i \ket{\psi_i} \bra{\psi_i} \sum_{i} \sum_{U - \delta U -E_i^S \leq E_j^B \leq U - E_i^S} 1 \\
&= \frac{1}{W(U, \delta U)} \sum_i W^B(U - E_i^S, \delta U) \ket{\psi_i} \bra{\psi_i}.
\end{align}
となる。状態数とエントロピーの関係 \(S(E) = k_B \ln W(U, \Delta U)\) から
\begin{align}
W^B(U - E_i^S) &= e^{S^B(U - E_i^S)/k_B}\\ 
&\approx \exp\left[ \frac{1}{k_B} \left( S^B(U) - \frac{d S^B(U)}{dU} E_i^S \right) \right] \\
&= e^{S^B(U)/k_B - \beta E_i^S} \\
&\propto e^{-\beta E_i^S}
\end{align}
となる。ただし、$E_i^S\ll U$ を用いた。したがって、規格化を考慮して、
\begin{align}
  \hat{\rho}^S &\approx \frac{\sum_i e^{-\beta E_i^S} \ket{\psi_i} \bra{\psi_i}}{\sum_i e^{-\beta E_i^S}} \\
  &= \frac{e^{-\beta \mathcal{H}_S}}{\text{Tr} e^{-\beta \mathcal{H}_S}}
\end{align}
となる。ここで、
\begin{equation}
  \beta = \frac{1}{k_B}\dv{S^B}{U} = \frac{1}{k_B T}
\end{equation}
である。ただし、\(T\) は熱浴の温度である。以上より、系 \(S\) はカノニカル分布で記述される。

\section{文献購読}
\subsection{1.13}
ブラックホールの観測的証拠について取り扱う。今回我々は取り扱うブラックホールを大別すると以下の二種類である。
\begin{enumerate}
  \item 恒星質量ブラックホール
  \item 超大質量ブラックホール
\end{enumerate}
恒星質量ブラックホールは、質量が太陽の数倍から数十倍程度のブラックホールであり、超大質量ブラックホールは、質量が太陽の数十万倍から数十億倍程度のブラックホールである。

\subsubsection{恒星質量ブラックホール}
恒星質量ブラックホールの観測的証拠としては、X線連星系が挙げられる。
\begin{itembox}[l]{\textbf{Def:X線連星系}}
  X線連星系とは、恒星質量ブラックホールと恒星が連星系を形成している連星系のことである。

\end{itembox}
このとき、以下の図のように、恒星質量ブラックホールと恒星が互いに引き合って回転運動をする。
%図を挿入する
このとき、恒星質量ブラックホールは、恒星から物質を引き寄せる。この物質は、恒星質量ブラックホールの周りで、遠心力と釣り合うことにより、円盤状に広がる。この円盤を降着円盤という。
降着円盤に入ってきた物体は、円盤内の粘性摩擦により、力学的エネルギーを失いながら、円盤の内側に近づいていく。このとき、物体は円盤内で摩擦熱を発生される。この摩擦熱によって、ガスが熱され、
ガスはX線を放出する。このX線が観測されることによって、恒星質量ブラックホールの存在が確認される。

ところで、このようなブラックホールの質量の測定は、二体の回転運動の周期からわかる。二体の回転運動の方程式は、円運動を仮定すると、
\begin{equation}
  \mu r\omega^2 = \frac{GM_0M_X}{r^2}
\end{equation}
である。ただし、
\begin{align}
  \mu &= \frac{M_0M_X}{M_0+M_X} 
\end{align}
である。これを解いて、
\begin{equation}
  \omega = \sqrt{\frac{G(M_0+M_X)}{r^3}}
\end{equation}
となる。あとは、周期を測定すれば、質量を求めることができる。
具体的な天体としては、X線新星XTE J11118+480が挙げられる。これは、周期が4.08時間で、測度が約700km/sの白色矮星を伴っている。
これは、太陽の少なくとも6倍の質量をもつことに相当し、中性子星がもちうる最大質量を大幅に超えている。

恒星質量ブラックホールが具体的にどのようにしてブラックホールになるかはまだ分かっていない。不確定要素の一つは状態方程式である。状態方程式は、圧力とと密度の関係を
指すわけだが、密度があまりにも大きい($10^15g/cm^3$)ため、物質の性質が分からない。

\subsubsection{超大質量ブラックホール}
銀河の中心には、ブラックホールが存在していると考えられている。超大質量ブラックホールの観測的証拠としては、銀河核の周りを周回する恒星の運動が挙げられる。

\subsection{1.14}
ブラックホールについて理解するうえで、乱流の理論が重要であるが、乱流については未解決の問題が多い。特に、
電離したガスが磁場を通過するような場合について、乱流の理論が未解決である。

乱流は、エネルギーや物質の輸送、角運動量の放出を支配している。乱流過程において、
降着円盤において物体が角運動量を失い、中心へ落ち込む速度を支配している。また、伴星からの物質が降着円盤に入る過程も乱流によって支配されている。
したがって、乱流の理論が重要である。

\subsection{1.15}
宇宙物理学が取り組む課題として、生命の起源の問題がある。

\newpage

\section{電磁気学とは}
電磁気学とは、物理学における4つの力のうちの一つである電磁相互作用を記述する学問である。
とくに、"場"と呼ばれる概念を導入し、電荷はその場から力を受けることとなる。\\

\textbf{近接作用/場}\\
電荷同士の相互作用の仕方として、近接作用というものを考える。これは、電荷同士が直接的に相互作用するのではなく、何かを介して相互作用するという考え方である。例えば、トランポリンに重いボールを一つ置いておく。このとき、
ボールの影響で、トランポリンがゆがむ。そこで、もう一つボールを置くと、もともとあったボールのほうに転がっていくはずである。この現象は、二つのボールが直接相互作用したわけではなく(万有引力はそこまで大きくない)、ボールによって空間がゆがめられ、それによって
力を受けているのである。\\
このような状況が、電磁気学においても成り立つ。例えば、空間に電荷があると、空間がゆがめられる。このゆがみを電場と呼ぶ。電場があるところにもう一つ電荷を置くと、空間のゆがみを介して力を受ける。また、磁気的な力についても同様の考え方ができ、磁気的な力を発生させる空間のゆがみを
磁場という。\\

\textbf{Maxwell方程式}\\
電磁気的な相互作用が電場や磁場を介して行われることを踏まえると、その電場や磁場の満たす普遍的な関係がわかれば、少なくとも理論上は電磁相互作用をすべて記述することができる。
このような関係式は、以下の四本で書くことができる。
\begin{align}
  \nabla \cdot \bm{E} &= \frac{\rho}{\varepsilon_0} \label{eq:1}\\
  \nabla \cdot \bm{B} &= 0 \label{eq:2}\\
  \nabla \times \bm{E} &= -\pdv{\bm{B}}{t} \label{eq:3}\\
  \nabla \times \bm{B} &= \mu_0 \bm{J} + \mu_0 \varepsilon_0 \pdv{\bm{E}}{t} \label{eq:4}
\end{align}
これをMaxwell方程式という。ここで、\(\bm{E}\)は電場、\(\bm{B}\)は磁場、\(\rho\)は電荷密度、\(\bm{J}\)は電流密度、\(\varepsilon_0\)は真空の誘電率、\(\mu_0\)は真空の透磁率である。\\
上の四式で本質的な点は、\(\bm{E}\)と\(\bm{B}\)が互いに関係していることである。とくに、(\ref{eq:3})と(\ref{eq:4})は、電場と磁場が時間変化することによって、互いに変化を引き起こすことを示している。
この意味で、電場と磁場は切り離して考えることができず、電磁場としてまとめて取り扱うべきである。以上より、電場と磁場が一般的に定まった。
このとき、電荷の大きさを$q$とすると、電荷が受ける力は、
\begin{equation}
  \bm{F} = q(\bm{E} + \bm{v} \times \bm{B})
\end{equation}
と書くことができる。また、電場や磁場は物理的実体を持つ。実際、Maxwell方程式を組み合わせることで、電磁波の方程式は、
\begin{align}
  \nabla^2 \bm{E} - \mu_0 \varepsilon_0 \pdv[2]{\bm{E}}{t} &= 0\\
  \nabla^2 \bm{B} - \mu_0 \varepsilon_0 \pdv[2]{\bm{B}}{t} &= 0
\end{align}
となる。\\

\textbf{電磁気学とは}\\
以上を踏まえると、電磁気学とは、近接作用の立場から電荷の受ける力を記述する学問であるかつ、物理的実体を持つ電磁場を記述する学問であると言える。\\

\section{集合位相の証明}
\subsection{定理}
$X/A$がハウスドルフであることと、$\forall p ^in X\setminus A$に対して、$p$と$A$が分離されていることは同値である。ただし、$X$はハウスドルフ空間であり、$A \subset X$である。また、
$X/A$は$A$の点を一つの点につぶした空間である。\\

\textbf{Prf}\\

\section{計算科学期末レポート草稿}
\subsection{KLダイバージェンスを用いた確率分布のフィッティング}
\begin{itembox}[l]{\textbf{Def:KLダイバージェンス}}
  二つの確率分布$P,Q$に対して、KLダイバージェンスは、
  \begin{equation}
    D_{KL}(P||Q) = \sum_{x \in X} P(x) \log \frac{P(x)}{Q(x)}
  \end{equation}
  で定義される。
\end{itembox}
この量は、確率分布間の距離を測る量であるといえる。実際、KLダイバージェンスは非負であり、等号が成り立つのは、$P=Q$のときである。\\

この量を用いて、実験的に得られた確率分布と、理論的に求められる確率分布との違いを評価することができる。具体的には、

\subsection{カノニカル分布の用意の仕方}
カノニカル分布は、エネルギーの期待値を一定に保った時、もっとも一様分布に近い分布とみなすことができる。すなわち、エネルギーの期待値さえ与えれば、最小化問題としてカノニカル分布を求めることができる。\\

\section{a}
である。ここで、熱浴の状態数を、
\begin{equation}
  \Omega_{R} (B) = \exp(V_{R}\sigma \left(\frac{B}{V_{R}},\frac{N_{R}}{V_{R}}\right))
\end{equation}
と書くことができる。
$\tilde{U} = U - E_i$とおくと、
\begin{align}
  p_i &= \frac{\Omega(\tilde{U}) - V_{R}\delta}{\Omega(\tilde{U})} \\
  &= \exp(V_{R}\{\sigma(\tilde{u}-\delta ,\rho) - \sigma(\tilde{u},\rho)\} + o(\delta))\\
  &= \exp(-V_{R}\left(\pdv{\tilde{u}}\sigma(\tilde{u},\rho)\delta + o(\delta^2)\right)+ o(V_R\delta))\ll 1
\end{align}
を満たす。ただし、$\tilde{u} = \frac{U}{V_{R}}$であり、熱浴の体積が十分大きいことを用いている。\\
このとき、
\begin{align}
  p_i &\simeq \frac{\Omega_{R}(U-E_i)}{\sum_{j}^{n} \Omega_{R}(U-E_j)}\\
  &= \frac{\Omega_{R}(U-E_i)}{\Omega_{R}(U)}\left(\sum_{j}^{n} \frac{\Omega_{R}(U-E_j)}{\Omega_{R}(U)}\right)^{-1}
\end{align}
と書ける。ここで、
\begin{align}
  \log \frac{\Omega_{R}(U-E_i)}{\Omega_{R}(U)} &= \log \Omega_{R}(U-E_i) - \log \Omega_{R}(U)\\
  &= -E_{i}\pdv{U}\log \Omega_{R}(U) + \frac{E_{i}^2}{2}\pdv{U}^2 \log \Omega_{R}(U) + \cdots\\
  &= -\frac{E_{i}}{V_{R}}\pdv{u}\{V_R\sigma(u,\rho) + o(V_R)\} + \frac{E_{i}^2}{2V_{R}^2}\pdv{u}^2\{V_R\sigma(u,\rho) + o(V_R)\} + \cdots\quad \\
  &= -E_{i}\pdv{u}\sigma(u,\rho) + \frac{1}{V_{R}}\frac{E_{i}^2}{2}\pdv{u}^2\sigma(u,\rho) +\cdots +\frac{o(V_R)}{V_{R}}\\
  &\simeq -\beta(u,\rho)E_{i} \quad \because \text{$V_{R}$が十分大きい}
\end{align}
となる。ここで、$\beta(u,\rho) = \pdv{u}\sigma(u,\rho)=\pdv{U}\log \Omega_{R}(U)$である。これにより、
\begin{align}
  \frac{\Omega_{R}(U-E_i)}{\Omega_{R}(U)} &= \exp(-\beta(u,\rho)E_{i})
\end{align}
と書くことができ、
\begin{align}
  p_i &= \frac{\exp(-\beta(u,\rho)E_{i})}{Z(\beta)}
\end{align}

\section{超伝導集中講義メモ}
穴埋めの答えを書く。\\
\subsection{London理論}
\noindent
(a)電子密度$n$のうち、数密度$n_s$の電子が超電導に寄与し、残りの電子$(n-n_s)$は超電導に寄与しないと考えるモデル。\\
(b)$0$\\
(c)連続的に増加する\\
(d)0\\
(e)1\\
(f)$ \tau \to \infty$\\
(g)
\begin{align}
  m^{*} \dv{v}{t} = -{e^{*}}\vb{E} - \frac{m^*}{\tau}\vb{v}
\end{align}
このとき、仮定より、第二項を無視することができる。\\
(h)
\begin{align}
  \vb{j}_s = n(-e^{*})\vb{v}
\end{align}
(i)
\begin{align}
  \dv{\vb{j}_s}{t} = \frac{ne^{*2}}{m^*}\vb{E}
\end{align}
(j)
\begin{align}
  \pdv{t}\qty(\nabla \times \Lambda_{L}\vb{j}_s) = -\pdv{\vb{B}}{t}
\end{align}
(k)
\begin{align}
  \pdv{t}\qty(\nabla \times \Lambda_{L}\vb{j}_s + \vb{B}) = 0
\end{align}
(l)
\begin{align}
  -\Lambda_{L}\nabla \times \vb{j}_s
\end{align}
(m)
\begin{align}
  -\frac{\Lambda_{L}}{\mu_0}\nabla \times (\nabla \times \vb{B})
\end{align}
(n)
\begin{align}
  \frac{\Lambda_{L}}{\mu_0}\nabla^2 \vb{B}
\end{align}
(o)
\begin{align}
  \vb{B} = \frac{\Lambda_{L}}{\mu_0}\nabla^2 \vb{B}
\end{align}
より、一次元系なので、
\begin{align}
  \pdv[2]{B}{x} = \frac{\mu_0}{\Lambda_{L}}B
\end{align}
となる。この微分方程式を解けばよい。便宜のために、
\begin{align}
  \lambda^2_L = \frac{\Lambda_{L}}{\sqrt{\mu_0}}
\end{align}
とおくと、一般解は、
\begin{align}
  B(x) = B_1\exp(-\frac{x}{\lambda_L}) + B_2\exp(\frac{x}{\lambda_L})
\end{align}
となる。境界条件より、$x\to \infty$で$B(x) \to 0$であるから、$B_2 = 0$である。
また、$x =0$で$B(x) = B_a$であるから、$B_1 = B_a$である。したがって、
\begin{align}
  B(x) = B_a\exp(-\frac{x}{\lambda_L})
\end{align}
となる。このとき、$x \to \infty$で$B(x) \to 0$であるから、これは、マイスナー効果を説明する。また、このときの$\lambda_L$を侵入長という。\\
%TODO:図を挿入する
(p, q)\\
\begin{align}
  \nabla \times \vb{B} = \mu_0\vb{j}_s  
\end{align}
である。左辺について、
\begin{align}
  \nabla \times \vb{B} = 
  \begin{pmatrix}
    0\\
    -\pdv{B}{x}\\
    0
  \end{pmatrix}
\end{align}
である。すなわち、電流は$y$成分のみを持つ。
\begin{align}
  \mu_0j_{s,y} &= -\pdv{B}{x}\\
  &= \frac{B_a}{\mu_0\lambda_L}\exp(-\frac{x}{\lambda_L})
\end{align}
である。これは、超伝導体の表面付近にのみ電流が流れることを表している。\\
%TODO:図を挿入する
(r)
\begin{align}
  \nabla \times \Lambda_L\vb{j}_s + \nabla \times \vb{A} = 0
\end{align}
\begin{align}
  \nabla \times (\Lambda_L\vb{j}_s + \vb{A}) = 0
\end{align}
となることから、
\begin{align}
  \vb{j_s} = -\frac{1}{\Lambda_L}\vb{A}
\end{align}
となる。(Londonゲージ)\\
(s)
\begin{align}
  -n_se^{*}\vb{v} = -\frac{n_se^{*}}{m^*}\vb{p}^* -\frac{n_se^{*2}}{m^*}\vb{A}
\end{align}
となる。このときの第一項が常磁性電流、第二項が反磁性電流である。とくに、超伝導体内では、ロンドン方程式の条件から、第一項が0になることがわかる。\\
(t)
\begin{align}
  \frac{1}{2m}(p_1 + p_2)(p_1 - p_2) = \frac{1}{m}p(p_1-p_2)
\end{align}
となる。($p_1 \sim p_2 \sim p$)\\
(u)
\begin{align}
  \Delta p \sim \frac{k_B T_{c}}{v_F}
\end{align}
(v)
\begin{align}
  \Delta r \sim \frac{\hbar v_F}{k_B T_{c}}
\end{align}

\subsection{GL理論}
\noindent
(a)時間反転対称性\\
(b)$T \geq T_C$で$M=0$に極小を持つ$F(T,M)$が$T < T_C$で$M\neq 0$に極小をもてばよい。\\
(c)ある温度$T_C$で符号を変える。\\
(d)
\begin{align}
  A(T-T_C)
\end{align}
(e)正の定数\\
(f)$B \geq 0$\\
(g)
\begin{align}
  \pdv{F}{M} &= 2a_2(T)M + 4a_4(T)M^3=0
\end{align}
(h)
\begin{align}
  (A(T-T_C) + 2BM^2)M=0
\end{align}
(i)
\begin{align}
  2a_2(T)M + 4a_4(T)M^3 = H
\end{align}
(j)
\begin{align}
  \pdv[2]{F}{M}^{-1}
\end{align}
(k)
\begin{align}
  \pdv[2]{F}{M} = 2a_2(T) + 12a_4(T)M^2
\end{align}

\subsection{GL理論2}
\noindent
(a)超伝導電子の局所的な密度$n_s(\vb{r})$\\
(b)
\begin{align}
  \alpha' \frac{T-T_{C}}{T_C}
\end{align}
(c)
\begin{align}
  \delta|\psi|^2 &= |\psi + \delta\psi|^2 - |\psi|^2\\
  &= (\psi + \delta\psi)(\psi^* + \delta\psi^*) - \psi\psi^*\\
  &= \psi\psi^* + \psi\delta\psi^* + \psi^*\delta\psi + \delta\psi\delta\psi^* - \psi\psi^*\\
  &\simeq \psi\delta\psi^* + \psi^*\delta\psi 
\end{align}
(d)\\
\begin{align}
  \delta |\psi|^4 &= |\psi + \delta\psi|^4 - |\psi|^4\\
  &= 2|\psi|^2(\psi\delta\psi^* + \psi^*\delta\psi) 
\end{align}
(e)\\
\begin{align}
  \delta f_s &= \alpha(T)\delta|\psi|^2 + \frac{1}{2}\beta(T)\delta|\psi|^4 + \cdots\\
  &= (\alpha(T) + \beta |\psi|^2)(\psi\delta\psi^* + \psi^*\delta\psi)\\
  &= (\alpha(T) + \beta |\psi|^2)\psi\delta\psi^* + (\alpha(T) + \beta |\psi|^2)\psi^*\delta\psi
\end{align}
(f)\\
\begin{align}
  -\frac{1}{2}\mu_0\vb{H_c}^2
\end{align}
(g)\\
\begin{align}
  \xi^2\qty(\dv{f}{x})^2 = \frac{1}{2}(1-f^2)^2
\end{align}
(h)\\
\begin{align}
  \dv{f}{x} = \frac{1-f^2}{\sqrt{2}\xi}
\end{align}
(i)\\
$f \neq \pm 1$より、
\begin{align}
  \int \frac{df}{1-f^2} = \int \frac{dx}{\sqrt{2}\xi}
\end{align}
となる。これを計算すると、
\begin{align}
  \frac{1}{2}\log\left|\frac{1+f}{1-f}\right| = \frac{x}{\sqrt{2}\xi} + C
\end{align}
いま、$x = 0$で$f = 0$であるから、
\begin{align}
  \frac{1+f}{1-f} = \exp(\frac{2x}{\sqrt{2}\xi})
\end{align}
となる。したがって、
\begin{align}
  f = \tanh(\frac{x}{\sqrt{2}\xi})
\end{align}
となる。\\
(j)\\
\begin{align}
  \vb{j}_s(\vb{r}) = -\frac{e^{*2}}{m^*}| \psi(\vb{r})|^2\vb{A}(\vb{r})
\end{align}
(k)\\
\begin{align}
  \nabla \times \vb{j}_s &= -\frac{e^{*2}}{m^*}| \psi(\vb{r})|^2\nabla \times \vb{A}\\
  &= -\frac{e^{*2}}{m^*}| \psi(\vb{r})|^2\vb{B}
\end{align}
(l)\\
\begin{align}
  \qty(\frac{m^*}{\mu_0e^{*2}| \psi(\vb{r})|^2})^{1/2} = \frac{m^* \beta}{\mu_0e^{*2}\alpha'}\qty(1-\frac{T}{T_C})^{-1/2}
\end{align}
(m)\\
\begin{align}
  \kappa &= \frac{(2m^*|\alpha|)^{1/2}}{\hbar} \qty(\frac{\mu_0e^{*2}|\alpha|}{m^*\beta})^{1/2} \lambda^2\\
  &= \sqrt{\frac{2\mu_0e^{*2}|\alpha|}{m^*\beta}}\lambda^2\\
  &= \frac{\sqrt{2}\mu_0e^{*}H_C}{\hbar}\lambda^2\\
  &= \frac{2\pi}{\Phi_0}\sqrt{2}\mu_0e^{*}H_C\lambda^2
\end{align}

\subsection{GL理論3}
(a)\\
\begin{align}
  \frac{1}{\kappa^2}\dv[2]{\psi}{x}+ (1-A_y^2)\psi - \psi^3 = 0
\end{align}
ここで、$\psi$を実数に取る。\\
(b)\\
\begin{align}
  \dv[2]{\vb{A}}{x} = -\psi^2\vb{A}
\end{align}
(c)\\
\begin{align}
  \frac{1}{\kappa^2}\dv[2]{\psi}{x} + \psi - \psi^3 = 0
\end{align}
(d)\\
\begin{align}
  \frac{1}{2}\mu_0 H_C^2 \frac{4\sqrt{2}}{3\kappa} (>0)
\end{align}
(e)境界が少ないほうがエネルギー利得が大きい。\\
(f)境界が多いほうがエネルギー利得が大きい。\\
(g)\\
\begin{align}
  \frac{1}{m^*}(\vb{p}-e^{*}\vb{A}) = \frac{1}{m^*}(-i\hbar\nabla - e^{*}\vb{A})
\end{align}
(h)\\
\begin{align}
  e^{*}\psi^*\vb{v}\psi = \frac{e^{*}}{m^*}(\hbar\nabla\theta - e^{*}\vb{A})
\end{align}
(i)\\
\begin{align}
  \int_{C} \nabla \theta \cdot d\vb{l} = \theta_2 - \theta_1 = 2\pi N \quad N \in \mathbb{Z}
\end{align}
(j)\\
\begin{align}
  \int_{C} \vb{A} \cdot d\vb{l} = \int \nabla \times \vb{A} \cdot d\vb{S} = \Phi
\end{align}
(k)\\
\begin{align}
  \frac{\hbar}{e^{*}}2\pi N = N\Phi_0
\end{align}
(l)\\
\begin{align}
  H_{C2} = \sqrt{2}\kappa H_{C1}
\end{align}




\end{document}