\documentclass[a4paper,11pt]{jsarticle}

% 数式
\usepackage{amsmath,amsfonts}
\usepackage{amsthm}
\usepackage{bm}
\usepackage{mathtools}
\usepackage{amssymb}

% 表
\usepackage[utf8]{inputenc}
\usepackage{diagbox} % 斜線付きセルを作成するために必要
\usepackage{booktabs} % 表の罫線を美しくするために必要
\usepackage{hhline} % 水平罫線を制御するために必要

% 画像
\usepackage[dvipdfmx]{graphicx}
\usepackage{ascmac}
\usepackage{physics}
\usepackage{float} % 追加

% 図
\usepackage[dvipdfmx]{graphicx}
\usepackage{tikz} %図を描く
\usetikzlibrary{positioning, intersections, calc, arrows.meta,math} %tikzのlibrary

% ハイパーリンク
\usepackage[dvipdfm,
  colorlinks=false,
  bookmarks=true,
  bookmarksnumbered=false,
  pdfborder={0 0 0},
  bookmarkstype=toc]{hyperref}

% 式番号を章ごとにリセット
\numberwithin{equation}{section}

\begin{document}

\title{日記}
\author{大上由人}
\maketitle

\section*{2024年11月}
\subsection*{11月4日}
日記をつけることにした。基本的に数式をつかいたくなってしまう気がしたので、\LaTeX を使ってみることにした。\\
\textbf{進捗まわり} \\
久保統計の例題を1問解いた。なんだかんだ解けたが時間がかかった。古典粒子であれば運動エネルギーやポテンシャルエネルギーの形を具体的に決めなくても良いのがよかった。\\
線形代数のpdfを書き足した。概ねベクトル空間についての話が終わった。\\
量子情報のpdfを久々読み返したけど全然覚えてない。ゼミを開きたいけどキャパがまずい。量子熱力学もやりたいので少なくともTPCP写像周りは整理しなくてはならない。\\
準同型定理のうれしさを理解した。不定積分のイメージらしい。urlを貼る。\url{https://qiita.com/hiro949/items/9021c202dcc10792f214}\\
\textbf{どうでもよいこと} \\
体重がまずい。まずいです。本当に。明日からあすけんは付けます$\dots$。\\
新しく買ったズボンが結構よかった。もう一枚ほしいくらい。ワイドパンツっていいね(いいね)。\\

\subsection*{11月6日}
\textbf{進捗まわり} \\
久保統計の例題をもう1問解いた。こっちについても時間はかかったが解けた。\\
多様体ゼミがあった。写像の微分を取り扱った。速度ベクトルの係数を微笑変化と見てあげると、普通の関数との対応が取りやすかった。微分の成分表示は、
\begin{align}
  \begin{pmatrix}
    w_1 \\
    w_2 \\
    \vdots \\
    w_n
  \end{pmatrix}
  =
  \begin{pmatrix}
    \pdv{f_1}{x_1}(p) & \pdv{f_1}{x_2}(p) & \cdots & \pdv{f_1}{x_m}(p) \\
    \pdv{f_2}{x_1}(p) & \pdv{f_2}{x_2}(p) & \cdots & \pdv{f_2}{x_m}(p) \\
    \vdots & \vdots & \ddots & \vdots \\
    \pdv{f_n}{x_1}(p) & \pdv{f_n}{x_2}(p) & \cdots & \pdv{f_n}{x_m}(p)
  \end{pmatrix}
  \begin{pmatrix}
    v_1 \\
    v_2 \\
    \vdots \\
    v_m
  \end{pmatrix}
\end{align}
であった。これについて、とくに$\mathbb{R} \to \mathbb{R}$の場合は、
\begin{align}
  w = \pdv{f}{x}(p) v
\end{align}
と書ける。これは、$f(x)$を$x$で微分したものを$x=p$で評価したものであるが、$v,w$を微小変化と見てあげて、$v\to dx, w\to df$と見てあげると、$df = \pdv{f}{x} dx$となる。これがよく対応が取れていそう。\footnote{$v\to \dv{x}{t}, w\to \dv{f}{t}$と見てあげて、$dt$両辺かけるほうがいいかもしれない。}\\
朔が言っていた、Landau理論で6次までを取り入れると一次相転移が再現されるという話が面白かった。詳しい話は基幹講座統計力学に書いているらしい。具体的には、$\pm m, 0$で自由エネルギーの停留点が出てくるから、秩序層と無秩序層が共存するということらしい。\\
加群ってアーベル群のことなんかい!!!!謎の群かと思っていた。\\
最小多項式を求める方法を知らなかったが、固有多項式を求めて、それを割り切ることで求める方法が基本っぽかった。他にも余因子行列を使うタイプのものもあったが吸収できていない。\\
\textbf{どうでもよいこと} \\
若干疲れていたからか人に強い言葉を使ってしまった。仮に向こうの否だとしてもこちらからは可能な限り最大限の配慮をしていきたい。\\
やきとりおでん然、まあ安かった。飲み放題90min880円は流石に安い。が、一人で行く店としてはもっと美味しい店があるかも。人と行くときに安く済ませたいときはよさそう。\\
隣のおっさんが最後の方無限にカルピスサワーを飲んでいた。おじさんでもサワーを飲むのか。\\

\subsection*{11月7日}
\textbf{進捗まわり} \\
ゆらぐ系のゼミの準備をした。とくに、$n$次ガウス分布についての取り扱いをした。また、ガウシアンノイズの取り扱いについても学んだ。\\
ガウシアンノイズの決め方が、平衡統計力学を真似して、その形をランダム力としてダイナミクスに組み込むというものだった。\\
n次ガウス分布についてはいずれtexにまとめたい。というかランジュバン系についてもまとめたほうが良いのか。\\
\textbf{どうでもよいこと} \\
CoCo壱の20辛を食べた。二度と食べない。\\

\subsection*{11月8日}
\textbf{進捗まわり} \\
ゆらぐ系のゼミと大偏差原理のゼミの発表をした。つかれた。大偏差の方については、とくに大数の法則周りでちょっと気づきがあったのでよかった。
具体的には、$I$が微分不可能な場合について、$a^*= \lambda(0)$となる話について、とくにそれ以外の$k$についても、$I$の傾きが不連続になっている幅の分任意性があるので、$a^*$が一意に定まらないということがある。ただし、$k=0$は絶対に満たすので議論に問題はない。\\
ゆらぐ系についても、とくに長時間平均の下外力を加えることで揺動散逸定理の話につながるということがわかった。\\
\textbf{どうでもよいこと} \\
腰痛がひどい。やばい。\\

\subsection*{11月9日}
\textbf{進捗まわり} \\
toeicの勉強をした。金のフレーズを覚えた。\\
場の理論のレポートと実験レポートが作りかけ。\\
\textbf{どうでもよいこと} \\
ねむい。\\

\subsection*{11月17日}
\textbf{進捗まわり} \\
あまりにテスト勉強に追われていてまずい。採点もまずい。明日である程度終えたいのだが、、、、、、\\
\textbf{どうでもよいこと} \\
基礎に強くならないとなという気持ちがかなりある。線形代数に強くならなくてはならないし、微分積分も強くならなくてはならない。\\

\subsection*{11月25日}
\textbf{進捗まわり} \\
TURのとこの記述を少し付け足した。Langevin系における熱の取り扱いについて復習した。toeicの勉強をした。\\
\textbf{どうでもよいこと} \\
本の読み進め方として、今やっているのが
\begin{itemize}
  \item 一通り読む
  \item コピー用紙に書きだす
  \item 喋る
\end{itemize}
という感じである。わりとうまく回っている感じがするので、この方法でいこうと思う。とくに、ゼミがあると思って準備すると比較的質が上がりやすい。\\
山本解析力学読みたい。熱力学の数理も読みたい。\\
熱力学のうちの物理の部分を取り出しそびれている感じがする。というか自分で体系立ててくみ上げられている気もしない。多分もう一回ぐらい何か読んだ方がよくて、清水も田崎も読みたい。困る。\\

\section*{2024年12月}
\subsection*{12月7日}
\textbf{進捗まわり} \\
非平衡の前回分の資料を若干修正した。次回分の予習を進めた。操作パラメータが時間に依存するパターンの導出がわからなくて困っている。手元にある資料のうちまともに書いているのがゆらぐ系だけなのだけれど、これの導出もよくわからないのだから困った。
白石本の導出からだとうまくいく気が全然しないのでとりあえずゆらぐ系を腰を据えて読むしかなさそう。今日はもう疲れたから読みたくない。\\

toeicの勉強をした。リスニングを始めた。一日5つ分進めれば間に合いそう。\\

\textbf{どうでもよいこと} \\
言葉を使うのが上手な人は、場面ごとに応じて適切に言葉のニュアンスを変えることができる。これになりたい。\\
今上の文章を書くときはもう少し具体的に書きたかったのだけれど適切な表現が浮かばなくて困った。語彙の貧弱さを感じる。\\

\end{document}