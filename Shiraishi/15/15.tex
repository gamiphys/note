\documentclass[a4paper,11pt]{jsarticle}

% 数式
\usepackage{amsmath,amsfonts}
\usepackage{amsthm}
\usepackage{bm}
\usepackage{mathtools}
\usepackage{amssymb}

% 表
\usepackage[utf8]{inputenc}
\usepackage{diagbox} % 斜線付きセルを作成するために必要
\usepackage{booktabs} % 表の罫線を美しくするために必要
\usepackage{hhline} % 水平罫線を制御するために必要

% 画像
\usepackage[dvipdfmx]{graphicx}
\usepackage{ascmac}
\usepackage{physics}
\usepackage{float} % 追加

% 図
\usepackage[dvipdfmx]{graphicx}
\usepackage{tikz} %図を描く
\usetikzlibrary{positioning, intersections, calc, arrows.meta,math} %tikzのlibrary

% ハイパーリンク
\usepackage[dvipdfm,
  colorlinks=false,
  bookmarks=true,
  bookmarksnumbered=false,
  pdfborder={0 0 0},
  bookmarkstype=toc]{hyperref}

% 式番号を章ごとにリセット
\numberwithin{equation}{section}

\begin{document}

\title{15章}
\author{大上由人}
\date{\today}
\maketitle
\setcounter{section}{14}

\section{最大仕事率のもとでのエンジンの効率}
\subsection{}

\begin{itembox}[l]{\textbf{Def. Endoreversible 熱力学}}
  Endoreversible熱力学とは、温度$T$の熱浴に接触した等温過程で、以下の性質を満たすものである。
  \begin{itemize}
    \item 系は常に平衡状態にある。その温度$T'$は熱浴の温度$T$と異なる可能性もある。
    \item 系と熱浴の熱交換は、Fourierの法則に従う。すなわち、
    \begin{equation}
      J_{Q} = \kappa(T-T')
    \end{equation}
    ただし、$J_{Q}$は系に入ってくる向きの熱流、$\kappa$は熱伝導率である。
    \item 系の温度$T'$はその系のエネルギー$E$,体積$V$,粒子数$N$の組$(E,V,N)$にのみ依存する。
  \end{itemize}
\end{itembox}
この枠組みでの熱機関のサイクル過程を考える。この過程において、系は高温熱浴(温度$T_{H}$)と低温熱浴(温度$T_{L}$)との間でエネルギーをやり取りする。
我々は、断熱過程は可逆であるが、等温過程に比べて圧倒的に素早く行われると仮定する。また、等温過程において、高温(低温)熱浴と接しているとき、系の温度は$T'_H$($T'_L$)で一定であると仮定する。このとき、
以下の定理を示すことができる。

\begin{itembox}[l]{\textbf{Thm.Curzon-Ahlborn 効率}}
  すべての物質特性(熱伝導率、エントロピー関数など)が固定されていて、熱機関の二つの温度$T'_H$と$T'_L$が制御可能なパラメータであると仮定する。このとき、
  最大仕事率の元での熱機関の効率は以下の不等式を満たす。
  \begin{equation}
    \eta_{\text{MP}} \leq 1 - \sqrt{\frac{T_L}{T_H}} := \eta_{\text{CA}}
  \end{equation}
  この最右辺の値をCurzon-Ahlborn効率という。\footnote{$\eta_{\text{MP}}$は最大仕事率の元での効率を表す。}
\end{itembox}
\textbf{Prf.}\\
$Q_{H}$を高温熱浴から系に流れ込む熱量、$Q_{L}$を系から低温熱浴に流れ出る熱量とする。また、高温(低温)熱浴と接している時間をそれぞれ$t_{H}$($t_{L}$)とする。また、それぞれの熱伝導率を$\kappa_{H}$($\kappa_{L}$)とする。このとき、
\begin{align}
  Q_{H} &= \kappa_{H}t_{H}(T_{H}-T'_{H})\\
  Q_{L} &= \kappa_{L}t_{L}(T'_{L}-T_{L})
\end{align}
となる。また、サイクル全体でエントロピーが変化しないことから、高温熱浴との相互作用でのエントロピー増大と、低温熱浴との相互作用でのエントロピー減少が等しい。すなわち、
\begin{equation}
  \Delta S = \frac{Q_{H}}{T'_{H}} = \frac{Q_{L}}{T'_{L}}
\end{equation}
となる。このとき、仕事率は、
\begin{align}
  \dot{W} &= \frac{Q_{H} - Q_{L}}{t_{H} + t_{L}}\\
  & x = T_{H} - T'_{H},\quad y = T'_{L} - T_{L} \quad \text{とすると、}\\
  &= \frac{(T_H + T_L - x - y)\kappa_H \kappa_L xy}{\kappa_L T_H y + \kappa_H T_L x + (\kappa_H - \kappa_L) xy}
\end{align}
となる。%TODOもう少し式変形を書く
これを最大化するために、$\dot{W}$を$x$と$y$で偏微分し、それが0になる条件を求めると、かなり面倒な計算の末、
\begin{align}
  \kappa_L T_H y^*(T_H + T_L - x^* - y^*) - (\kappa_L T_H y^* + \kappa_H T_L x^* + (\kappa_H - \kappa_L) x^* y^*) x^* &= 0\\
  \kappa_H T_L x^*(T_H + T_L - x^* - y^*) - (\kappa_L T_H y^* + \kappa_H T_L x^* + (\kappa_H - \kappa_L) x^* y^*) y^* &= 0
\end{align}
となる。両辺割り算して整理することで、
\begin{equation}
  y^* = \sqrt{\frac{\kappa_H T_L}{\kappa_L T_H}}x^*
\end{equation}
となる。したがって、
\begin{align}
  x^* &= \frac{T_H \left(1 - \sqrt{\frac{T_L}{T_H}} \right)}{1 + \sqrt{\frac{\kappa_H}{\kappa_L}}}\\
  y^* &= \frac{T_L \left(\sqrt{\frac{T_H}{T_L}} - 1\right)}{1 + \sqrt{\frac{\kappa_L}{\kappa_H}}}
\end{align}
このもとで、熱効率を求めると、
\begin{align}
  \eta_{\text{MP}} &= \frac{Q_H - Q_L}{Q_H} = 1 - \frac{T_L + y^*}{T_H + x^*} = 1 - \frac{T_L \sqrt{\kappa_H} \sqrt{\frac{T_H}{T_L}} + \sqrt{\kappa_L}}{T_H \sqrt{\kappa_L} \sqrt{\frac{T_L}{T_H}} + \sqrt{\kappa_H}} = 1 - \sqrt{\frac{T_L}{T_H}}
\end{align}
となる。このことと、一般にカルノーサイクルが効率最大のサイクルであることから、目的の不等式が得られる。\qed\\

\subsection{Onsagar行列によるアプローチ}
時間反転対称性を持つような、定常系を考える。\\
\textbf{記号}\\
\begin{itemize}
  \item $X_{1}$: 適当な力学的変数
  \item $J_{1}$: 熱流($J_{2}$)によって駆動される流れ
  \item $X_{2}$:二つの熱浴の逆温度の差
  \begin{align}
    X_{2} = \frac{1}{T_{L}} - \frac{1}{T_{H}}
  \end{align}
  \item $J_{2}$:熱流
\end{itemize}
いま、線形応答の枠組みで考えるため、
\begin{align}
  T_{H} \simeq T_{L} \simeq T
\end{align}
とする。hogehoge%TODOあとで書く


線形応答理論を適応するため、揺動散逸定理のあたりで出てきたOnsagar行列を用いる。この行列は、
\begin{align}
    J_1 &= L_{11} X_1 + L_{12} X_2\\
    J_2 &= L_{21} X_1 + L_{22} X_2
\end{align}
ただし、Onsagarの相反定理により、$L_{12} = L{21}$である。また、このとき、熱力学第二法則から、
\begin{align}
    \dot{S} := J_1 X_1 + J_2 X_2 \geq 0
\end{align}
が得られる。この不等式は、
\begin{align}
  L_{11} X_1^2 + 2L_{12} X_1 X_2 + L_{22} X_2^2 \geq 0
\end{align}
に等しい。これが常に成り立つ条件は、判別式を考えることにより、
\begin{align}
    L_{11} &\geq 0,  \\
    L_{22} &\geq 0,  \\
    L_{11}L_{22} - L_{12}^2 &\geq 0
\end{align}
となる。とくに、三本目の不等式は、
\begin{align}
    q := \frac{L_{12}}{\sqrt{L_{11}L_{22}}}
\end{align}
が
\begin{align}
  -1 \leq q \leq 1
\end{align}
を満たすことを意味する。このもとで、熱効率が以下の不等式を満たすことが知られている。

\begin{itembox}[l]{\textbf{Thm.線形応答のもとでの定常系の効率}}
  上で述べたような系について、Onsagar行列および$X_2$が固定されていて、$X_1$が制御可能なパラメータであると仮定する。このとき、最大仕事率のもとでの熱機関の効率は以下の不等式を満たす。
  \begin{align}
    \eta_{\text{MP}} \leq \frac{\eta_C}{2} 
  \end{align}
  ただし、$\eta_C$はカルノーサイクルの効率である。
\end{itembox}

\textbf{Prf.}\\
いま、仕事率は、
\begin{align}
  \dot{W} &= -X_1 J_1 T =-X_1(L_{11} X_1 + L_{12} X_2) T = \left(-L_{11} \left( X_1 + \frac{L_{12} X_2}{2 L_{11}} \right)^2 + \frac{L_{12}^2 X_2^2}{4 L_{11}} \right) T
\end{align}
と書くことができる。これは$X_1 = -L_{12} X_2 / 2 L_{11}$において最大値をとる。これを熱効率の定義に代入することで、
\begin{align}
  \eta_{\text{MP}} &= -\frac{X_1(L_{11} X_1 + L_{12} X_2) T}{L_{21} X_1 + L_{22} X_2} \\
  &= -\frac{X_1(L_{11}\qty(-\frac{L_{12} X_2}{2 L_{11}}) + L_{12} X_2) T}{L_{21} \qty(-\frac{L_{12} X_2}{2 L_{11}}) + L_{22} X_2} \\
  &= -\frac{X_1\qty(-\frac{L_{12}}{2}+ L_{12})T}{-\frac{L_{12}^2}{2L_{11}} + L_{22}} \\
  &= \frac{L_{12} X_2}{2 L_{11} }\frac{\frac{1}{2}L_{12}T}{-\frac{L_{12}^2}{2L_{11}} + L_{22}} \\
  &= \frac{L_{12} X_2}{2 L_{11} }\frac{\frac{L_{12}}{L_{22}}T}{2-\frac{L_{12}^2}{L_{11}L_{22}} } \\
  &= \frac{1}{2} \frac{L_{12}^2}{L_{11}L_{22}} \frac{T}{2-q^2}X_2\\
  &\simeq \frac{1}{2} \frac{\Delta T}{T} \frac{q^2}{2 - q^2} \\
  &\leq \frac{1}{2} \frac{\Delta T}{T}\\
  &= \frac{\eta_C}{2}
\end{align}
したがって、目的の不等式が得られる。\qed\\
hogehoge

\subsection{速度による展開}
\begin{itembox}[l]{\textbf{Thm.}}
外部から操作されるサイクル型の熱機関を考える。その動作速度はパラメータ $u$ によって特徴付けられ、この $u$ は操作可能なパラメータとする。熱機関が $u \to 0$ の極限で最大効率 $\eta_{\text{max}}$ を達成すると仮定する。また、各電流に伴う単位時間あたりの散逸量は $u$ の増加に伴って増加するものと仮定する。すると、$u$ に関して線形応答の範囲内では、
\begin{align}
    \frac{\eta_{\text{max}}}{2} \leq \eta_{\text{MP}} \leq \frac{1}{2 - \eta_{\text{max}}}
\end{align}
が小さい $u$ の場合に成立する。
\end{itembox}

特に、線形応答の範囲内(すなわち、2つの熱浴間の温度差が小さい場合)では、
\begin{align}
    \eta_{\text{MP}} = \frac{\eta_{\text{max}}}{2}
\end{align}
が成立する。また、Curzon-Ahlborn効率 $\eta_{\text{CA}} := 1 - \sqrt{\frac{T_L}{T_H}}$ は $\eta_{\text{max}} = 1 - \frac{T_L}{T_H}$ を用いて式(15.22)を満たす。

\textbf{Prf.} \ 熱流 $J_Q$ と単位時間あたりに取り出される仕事量 $J_W$ をそれぞれ高温浴からの熱流とする。$J_W$ と $J_Q$ を $u$ で展開する:
\begin{align}
    J_W(u) &= a_1 u - a_2 u^2 + \cdots \\
    J_Q(u) &= b_1 u - b_2 u^2 + \cdots
\end{align}

これらの展開には定数項が存在しない。なぜなら、準静的な極限 ($u \to 0$) では $J_W = J_Q = 0$ だからである。効率が準静的な極限で最大効率 $\eta_{\text{max}}$ に近づくため、以下が成り立つ:
\begin{align}
    \frac{a_1}{b_1} = \eta_{\text{max}}
\end{align}

仕事の取り出しにおける散逸項 $a_2 u^2$ は、冷浴への熱流における対応する散逸項である $b_2 u^2$ と等しいため、$a_2 \geq b_2 \geq 0$ が成立する。

From Eq.~(15.24)、$J_W$は次で最大化される:
\begin{align}
    u^* = \frac{a_1}{2 a_2}
\end{align}

$u = u^*$と設定すると、EMPは次のように表される:
\begin{align}
    \eta_{\text{MP}} := \frac{J_W(u^*)}{J_Q(u^*)} &= \frac{a_1 - a_2 \frac{a_1}{2a_2}}{b_1 \left( 1 - \frac{b_2}{b_1} \frac{a_1}{2a_2} \right)} = \frac{\eta_{\text{max}}}{2} \frac{1}{1 - \eta_{\text{max}} \frac{b_2}{2a_2}}
\end{align}
$\mathcal{O}(u^*)$ まで。この式に $0 \leq \frac{b_2}{a_2} \leq 1$ を代入すると、所望の不等式(15.22)が得られる。 \hfill $\Box$

ここで2つの注意を述べる。まず、もし高温浴からの熱流の損失と冷浴への損失が等しい場合、$a_2 = 2 b_2$ が成立し、この場合のEMPは次のように書ける
\begin{align}
    \eta_{\text{MP}} := \frac{\eta_{\text{max}}}{2} \frac{1}{1 - \frac{\eta_{\text{max}}}{4}} = \frac{\eta_{\text{max}}}{2} - \frac{\eta_{\text{max}}^2}{8} + \mathcal{O}(\eta_{\text{max}}^3)
\end{align}

この式は、左右対称性を持つ普遍的な2次の係数 $1/8$ を回復する。

次に、最大効率がカルノー効率 $\eta_{\text{max}} = \eta_C$ であると仮定していない点に留意する。上述の定理は、一般的に、準静的極限 $u \to 0$ で熱漏れがない場合、EMPが最大効率の半分に等しいことを示している。したがって、有限サイズの熱浴を持つ熱機関にこの定理を適用すると、たとえば、エクセルギーに関してEMPが効率の半分であることが示される。

\end{document}