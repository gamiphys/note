\documentclass[a4paper,11pt]{jsarticle}

% 数式
\usepackage{amsmath,amsfonts}
\usepackage{amsthm}
\usepackage{bm}
\usepackage{mathtools}
\usepackage{amssymb}

% 表
\usepackage[utf8]{inputenc}
\usepackage{diagbox} % 斜線付きセルを作成するために必要
\usepackage{booktabs} % 表の罫線を美しくするために必要
\usepackage{hhline} % 水平罫線を制御するために必要

% 画像
\usepackage[dvipdfmx]{graphicx}
\usepackage{ascmac}
\usepackage{physics}
\usepackage{float} % 追加

% 図
\usepackage[dvipdfmx]{graphicx}
\usepackage{tikz} %図を描く
\usetikzlibrary{positioning, intersections, calc, arrows.meta,math} %tikzのlibrary

% ハイパーリンク
\usepackage[dvipdfm,
  colorlinks=false,
  bookmarks=true,
  bookmarksnumbered=false,
  pdfborder={0 0 0},
  bookmarkstype=toc]{hyperref}

\begin{document}

\title{6章}
\author{大上由人}
\date{\today}
\maketitle
\section{揺動散逸定理}
\subsection{周波数0の場合}
前回のゼミで得た、周波数0の場合の揺動散逸定理の復習をする。\\
\begin{itembox}[l]{\textbf{Thm:揺動散逸定理}}
    IFTを満たす確率的な系で、示量変数$X$をもつ二つの熱浴と接触している。
        平衡状態での$X$のカレントの自己相関関数$\ev{\hat{J}(t)\hat{J}(0)}_0$は、$t$が大きくなるにつれ、急速に0に緩和すると仮定する。\\
        このとき、平衡状態における揺動と、共役変数が変化したときの$X$のカレントの応答は、
        \begin{equation}
        \int_{0}^{\infty} \dd t \ev{\hat{J}(t)\hat{J}(0)}_0 = \pdv{\Delta \Pi_X} \ev{\hat{J}}_0|_{\Delta \Pi_X = 0} =T\pdv{\Delta P_X} \ev{\hat{J}}_0|_{\Delta P_X = 0}
        \end{equation}
        のように書ける。ただし、$\ev{\quad}_0$は平衡状態におけるアンサンブル平均を表し、$\ev{\quad}$は$\Delta \Pi_X$により駆動される非平衡定常状態でのアンサンブル平均を表す。
    \end{itembox}
    \textbf{Prf}\\
    $o \leq t \leq \tau$間のエントロピー生成は、
    \begin{equation}
        \hat{\sigma} = \delta \Pi_X \int_{0}^{\tau} \dd t \hat{J}(t) 
    \end{equation}
    により書くことができる。(証明略)これと、IFTにより、
    \begin{align}
        1=\ev{e^{-\hat{\sigma}}}&=1-\ev{\Delta \Pi_X \int_{0}^{\tau} \dd t \hat{J}(t)} + \frac{1}{2}\ev{\Delta \Pi_X ^2 \int_{0}^{\tau} \dd t' \int_{0}^{\tau} \dd t \hat{J}(t')\hat{J}(t)} + O(\Delta \Pi_X ^3)\\
    \end{align}
    を得る。ここで、
    \begin{equation}
        \ev{\hat{A}}_{\Delta \Pi_X} = \ev{\hat{A}}_0 + \left.\pdv{\Delta \Pi_X} \ev{\hat{A}}\right|_{\Delta \Pi_X = 0} \Delta \Pi_X + O(\Delta \Pi_X ^2)
    \end{equation}
    と展開できるので、これを上の式に代入し、$\Delta \Pi_X$の一次の項と二次の項を比較する。\\
    $\Delta \Pi_X$の一次の項について、
    \begin{align}
        \ev{\int_{0}^{\tau} \dd t \hat{J}(t)}_0 =0
    \end{align}
    が成り立つ。また、二次の項について、
    \begin{equation}
        \label{eq:1}
        \ev{\frac{1}{2}\int_{0}^{\tau} \dd t' \int_{0}^{\tau} \dd t \hat{J}(t')\hat{J}(t)}_0 = \left. \pdv{\Delta \Pi_X} \ev{\int_{0}^{\tau} \dd t \hat{J}(t)}\right|_{\Delta \Pi_X = 0}
    \end{equation}
が成り立つ。(\ref{eq:1})式の両辺に$\frac{1}{\tau}$をかけ、$\tau \to \infty$の極限を取ると、
\begin{align}
    \underset{\tau \to \infty}{\text{lim}} \frac{1}{\tau} \ev{\frac{1}{2}\int_{0}^{\tau} dt' \int_{0}^{\tau} dt \hat{J}(t')\hat{J}(t)}_0 
    &=\underset{\tau \to \infty}{\text{lim}} \frac{1}{\tau} \left.\pdv{\Delta \Pi_X} \ev{\int_{0}^{\tau} dt \hat{J}(t)}\right|_{\Delta \Pi_X = 0}\\
    &=\underset{\tau \to \infty}{\text{lim}} \frac{1}{\tau} \int_{0}^{\infty} dt''(\tau - t'')\ev{\hat{J}(t'')\hat{J}(0)}_0\\
    &\simeq \int_{0}^{\infty} dt \ev{\hat{J}(t)\hat{J}(0)}_0
\end{align}
また、
\begin{align}
    \int_{0}^{\infty} \dd t \ev{\hat{J}(t)\hat{J}(0)}_0 &= \frac{1}{\tau}\left.\pdv{\Delta \Pi_X} \ev{\int_{0}^{\tau} \dd t \hat{J}(t)}\right|_{\Delta \Pi_X = 0}\\
    &= \left.\pdv{\Delta \Pi_X} \ev{\hat{J}(0)}\right|_{\Delta \Pi_X = 0}
\end{align}
が成り立つ(ここで、時間平均がアンサンブル平均に置き換わる、エルゴードの定理を用いている)。\\
$\pdv{\Delta \Pi_X} =T\pdv{\Delta P_X}$であることを用いると、
\begin{equation}
    \int_{0}^{\infty} \dd t \ev{\hat{J}(t)\hat{J}(0)}_0 = T\left.\pdv{\Delta P_X} \ev{\hat{J}(0)}\right|_{\Delta P_X = 0}
\end{equation}
を得る。\hfill \qedsymbol

\subsection{有限周波数の場合}
有限周波数の場合のFDTを扱う。いま、bath1とbath2の示強変数が、
\begin{equation}
    \Pi_X ^2 (t) = \Pi_X ^1 + \epsilon \sin(\omega t)
\end{equation}
となっているとする。このとき、エントロピー生成は以下のように書ける。
\begin{equation}
    \hat{\sigma} = \epsilon \int_{0}^{\tau} \dd t \hat{J}(t) \sin(\omega t)
\end{equation}
ただし、$\Delta \hat{s}$の項は、後で無視するので消去した。\\
($\because$)\\
bath1について、
\begin{align}
    \pdv{S_1}{t} &= \pdv{S_1}{X} \pdv{X}{t}\\
    &= \pdv{S_1}{X} \int \dd V \pdv{x}{t}\\
    &= \pdv{S_1}{X} \int \dd \vb{S} \cdot (\vb{-j})\\
    &= -\pdv{S_1}{X}  \hat{J}(t)\\
\end{align}
となる。両辺積分することで、
\begin{align}
    S_1(\tau) - S_1(0) &= -\int_{0}^{\tau}  \pdv{S_1}{X} (-\hat{J}(t))\dd t\\
    &= -\int_{0}^{\tau}  \Pi_X ^1 \hat{J}(t) \dd t
\end{align}
bath2についても同様にして、
\begin{equation}
    S_2(\tau) - S_2(0) = \int_{0}^{\tau}  \Pi_X ^2 \hat{J}(t) \dd t
\end{equation}
を得る。\\
よって、
\begin{align}
    (S_1(\tau) + S_2(\tau)) - (S_1(0) + S_2(0)) &= \int_{0}^{\tau}  (\Pi_X ^2 - \Pi_X ^1) \hat{J}(t) \dd t\\
    &= \epsilon \int_{0}^{\tau}  \sin(\omega t) \hat{J}(t) \dd t
\end{align}
となる。\hfill \qedsymbol
\\
以下、この式を用いて、有限周波数の場合のFDTを導く。\\

\begin{itembox}[l]{\textbf{Thm:有限周波数の場合のFDT}}
    IFTを満たす確率的な系で、周波数$\omega$による揺動と、共役変数が変化したときの$X$のカレントの応答は、
    \begin{equation}
        \underset{\tau \to \infty}{\text{lim}}\frac{1}{\tau} \ev{\left(\int_{0}^{\tau} \dd t \hat{J}(t)\sin(\omega t)\right)^2}_0 = 2\underset{\tau \to 0}{\text{lim}} \left. \pdv{\epsilon} \ev{\frac{1}{\tau} \int_{0}^{\tau} \dd t \hat{J}(t)\sin(\omega t)}\right|_{\epsilon = 0}
    \end{equation}
    のように書ける。
\end{itembox}
\textbf{Prf}\\
周波数が0のときの証明をまねする。IFTを用いてから、$\epsilon$について展開することで、
\begin{align}
    1=\ev{e^{-\hat{\sigma}}}&=1-\ev{\epsilon \int_{0}^{\tau} \dd t \hat{J}(t) \sin(\epsilon t)} + \frac{1}{2}\ev{\epsilon^2 \int_{0}^{\tau} \dd t' \int_{0}^{\tau} \dd t \hat{J}(t')\hat{J}(t) \sin(\omega t')\sin(\omega t)} + O(\epsilon^3)\\
\end{align}
を得る。ここで、
\begin{equation}
    \ev{\hat{A}}_{\epsilon} = \ev{\hat{A}}_0 + \left. \pdv{\epsilon} \ev{\hat{A}}\right| _{\epsilon = 0} \epsilon + O(\epsilon^2)
\end{equation}
と展開して、上の式に代入すると、$\epsilon$の二次の項を比較することにより、
\begin{align}
    \frac{1}{2}\ev{\int_{0}^{\tau} \dd t' \int_{0}^{\tau} \dd t \hat{J}(t')\hat{J}(t) \sin(\omega t')\sin(\omega t)}_0 &= \left. \pdv{\epsilon} \ev{\int_{0}^{\tau} \dd t \hat{J}(t) \sin(\omega t)}\right|_{\epsilon = 0}
\end{align}
であるから、両辺$\frac{1}{\tau}$をかけ、$\tau \to \infty$の極限を取ると、
\begin{align}
    \underset{\tau \to \infty}{\text{lim}}\frac{1}{\tau} \ev{\left(\int_{0}^{\tau} \dd t \hat{J}(t)\sin(\omega t)\right)^2}_0 &= 2\underset{\tau \to 0}{\text{lim}} \left. \pdv{\epsilon} \ev{\frac{1}{\tau} \int_{0}^{\tau} \dd t \hat{J}(t)\sin(\omega t)}\right|_{\epsilon = 0}
\end{align}
を得る。\hfill \qedsymbol

\subsection{高次における関係}
\begin{itembox}[l]{\textbf{Thm:時間積分されたカレントの高次の関係}}
    磁場中の$\hat{\mathcal{J}}$は、以下の高次における関係を満たす:
    \begin{align}
        \left.\pdv{\Delta \Pi_X} \ev{\hat{\mathcal{J}}^2}^+\right|_{\Delta \Pi_X = 0} = \left.\pdv[2]{\Delta \Pi_X} \ev{\hat{\mathcal{J}}}^+\right|_{\Delta \Pi_X = 0}
    \end{align}

\end{itembox}
\textbf{Prf}\\
FDTを導出する途中の展開をより高次まで行うと、
\begin{align}
    1=\ev{e^{-\hat{\sigma}}}&= 1-\ev{\hat{\mathcal{J}}}^B \Delta \Pi_X + \frac{1}{2}\ev{\hat{\mathcal{J}}^2}^B \Delta \Pi_X ^2- \frac{1}{6}\ev{\hat{\mathcal{J}}^3}^B \Delta \Pi_X ^3 + O(\Delta \Pi_X ^4)\\
    &=1-(\ev{\hat{\mathcal{J}}}^B_0 \Delta \Pi_X + \left.\pdv{\Delta \Pi_X} \ev{\hat{\mathcal{J}}}^B \right|_{\Delta \Pi_X = 0} \Delta \Pi_X ^2 + \frac{1}{2}\left.\pdv[2]{\Delta \Pi_X} \ev{\hat{\mathcal{J}}}^B \right|_{\Delta \Pi_X = 0} \Delta \Pi_X ^3 )\\
    &+ \frac{1}{2}(\ev{\hat{\mathcal{J}}^2}^B_0 \Delta \Pi_X ^2 + \left.\pdv{\Delta \Pi_X} \ev{\hat{\mathcal{J}}^2}^B\right|_{\Delta \Pi_X = 0} \Delta \Pi_X ^3)- \frac{1}{6}(\ev{\hat{\mathcal{J}}^3}^B_0 \Delta \Pi_X ^3) + O(\Delta \Pi_X ^4)
\end{align}
となる。ここで、$\hat{\mathcal{J}}$は、
\begin{equation}
    \hat{\mathcal{J}} = \int_{0}^{\tau} dt \hat{J}(t)
\end{equation}
である。同様の手順を踏むことで、$B$を$-B$としたときの展開は、
\begin{align}
    1=\ev{e^{-\hat{\sigma}}}&= 1-\ev{\hat{\mathcal{J}}}^{-B} \Delta \Pi_X + \frac{1}{2}\ev{\hat{\mathcal{J}}^2}^{-B} \Delta \Pi_X ^2- \frac{1}{6}\ev{\hat{\mathcal{J}}^3}^{-B} \Delta \Pi_X ^3 + O(\Delta \Pi_X ^4)\\
    &=1-\left(\ev{\hat{\mathcal{J}}}^{-B}_0 \Delta \Pi_X + \left.\pdv{\Delta \Pi_X} \ev{\hat{\mathcal{J}}}^{-B}\right|_{\Delta \Pi_X = 0} \Delta \Pi_X ^2 + \frac{1}{2}\left.\pdv[2]{\Delta \Pi_X} \ev{\hat{\mathcal{J}}}^{-B}\right|_{\Delta \Pi_X = 0} \Delta \Pi_X ^3 \right)\\
    &+ \frac{1}{2}(\ev{\hat{\mathcal{J}}^2}^{-B}_0 \Delta \Pi_X ^2 + \left.\pdv{\Delta \Pi_X} \ev{\hat{\mathcal{J}}^2}^{-B}\right|_{\Delta \Pi_X = 0} \Delta \Pi_X ^3)- \frac{1}{6}\left(\ev{\hat{\mathcal{J}}^3}^{-B}_0 \Delta \Pi_X ^3\right) + O(\Delta \Pi_X ^4)
\end{align}
となる。\\
ここで、期待値の時間反転対称性から、
\begin{align}
    \ev{\hat{\mathcal{J}}^3}^B = -\ev{\hat{\mathcal{J}}^3}^{-B}
\end{align}
が成り立つことを用いて、上の二式を両辺足して、$\Delta \Pi_X$の三次の項を比較すると、
\begin{align*}
    0=&-\left.\pdv[2]{\Delta \Pi_X} \ev{\hat{\mathcal{J}}}^B\right|_{\Delta \Pi_X = 0} - \left.\pdv[2]{\Delta \Pi_X} \ev{\hat{\mathcal{J}}}^{-B}\right|_{\Delta \Pi_X = 0}\\
    &+ \left.\pdv{\Delta \Pi_X} \ev{\hat{\mathcal{J}}^2}^B\right|_{\Delta \Pi_X = 0} + \left.\pdv{\Delta \Pi_X} \ev{\hat{\mathcal{J}}^2}^{-B}\right|_{\Delta \Pi_X = 0} -\frac{1}{6}\ev{\hat{\mathcal{J}}^3}^B_0 - \frac{1}{6}\ev{\hat{\mathcal{J}}^3}^{-B}_0\\
\end{align*}
すなわち、
\begin{align}
    \left.\pdv{\Delta \Pi_X} \ev{\hat{\mathcal{J}}^2}^B\right|_{\Delta \Pi_X = 0} + \left.\pdv{\Delta \Pi_X} \ev{\hat{\mathcal{J}}^2}^{-B}\right|_{\Delta \Pi_X = 0} = \left.\pdv[2]{\Delta \Pi_X} \ev{\hat{\mathcal{J}}}^B\right|_{\Delta \Pi_X = 0} + \left.\pdv[2]{\Delta \Pi_X} \ev{\hat{\mathcal{J}}}^{-B}\right|_{\Delta \Pi_X = 0}
\end{align}
を得る。新たに、$\ev{\cdot}^+ = \frac{1}{2}(\ev{\cdot}^B + \ev{\cdot}^{-B})$と定義すると、
\begin{align}
    \left.\pdv{\Delta \Pi_X} \ev{\hat{\mathcal{J}}^2}^+\right|_{\Delta \Pi_X = 0} = \left.\pdv[2]{\Delta \Pi_X} \ev{\hat{\mathcal{J}}}^+\right|_{\Delta \Pi_X = 0}
\end{align}
が成り立つ。\hfill \qedsymbol

また、以下も成り立つ。\\
\begin{itembox}[l]{\textbf{Thm:高次元における関係(2)}}
    IFTを満たす系について、以下が成り立つ:
    \begin{align}
        2\left.\pdv{\Delta \Pi_X} \int_0 ^{\infty} \dd t \ev{\Delta \hat{J}(t) \Delta \hat{J}(0)}^+\right|_{\Delta \Pi_X = 0} = \left.\pdv[2]{\Delta \Pi_X} \ev{\Delta \hat{J}}^+\right|_{\Delta \Pi_X = 0}
    \end{align}
\end{itembox}
\textbf{Prf}\\
$\hat{\mathcal{J}}$を$J$を用いて表すことで、
\begin{align}
    \pdv{\Delta \Pi_X} \ev{\int_0 ^{\tau} \dd t' \int_0 ^{\tau} \dd t \hat{J}(t')\hat{J}(t)}^+|_{\Delta \Pi_X = 0} = \left.\pdv[2]{\Delta \Pi_X} \ev{\int_0 ^{\tau} \dd t \hat{J}(t)}^+\right|_{\Delta \Pi_X = 0}
\end{align}
が成り立つ。ここで、左辺の被積分関数について、$\Delta \hat{J} = \hat{J} - \ev{\hat{J}}$として展開すると、
\begin{align}
    \ev{\int_0 ^{\tau} \dd t' \int_0 ^{\tau} \dd t \hat{J}(t')\hat{J}(t)}^+ &= \ev{\int_0 ^{\tau} \dd t' \int_0 ^{\tau} \dd t (\Delta \hat{J}(t') +{J}(t'))(\Delta \hat{J}(t) +{J}(t))}^+\\
    &= \ev{\int_0 ^{\tau} \dd t' \int_0 ^{\tau} \dd t \Delta \hat{J}(t')\Delta \hat{J}(t)}^+ \\
    &+ 2\ev{\int_0 ^{\tau} \dd t' \int_0 ^{\tau} \dd t \Delta \hat{J}(t'){J}(t)}^+\\
    &+ \ev{\int_0 ^{\tau} \dd t' \int_0 ^{\tau} \dd t {J}(t') {J}(t)}^+
\end{align}
となる。ここで、第二項について、
\begin{align}
    2\ev{\int_0 ^{\tau} \dd t' \int_0 ^{\tau} \dd t \Delta \hat{J}(t') {J}(t)}^+ &= 2\int_0 ^{\tau} \int_0 ^{\tau} \dd t' \dd t J(t) \ev{\Delta \hat{J}(t')}^+\\
    &= 0
\end{align}
である。というのも、$\ev{\Delta \hat{J}(t')}^+ = 0$であるからである。よって、
\begin{align}
    \ev{\int_0 ^{\tau} \dd t' \int_0 ^{\tau} \dd t \hat{J}(t')\hat{J}(t)}^+ &= \ev{\int_0 ^{\tau} \dd t' \int_0 ^{\tau} \dd t (\Delta \hat{J}(t') +{J}(t'))(\Delta \hat{J}(t) +{J}(t))}^+\\
    &= \ev{\int_0 ^{\tau} \dd t' \int_0 ^{\tau} \dd t \Delta \hat{J}(t')\Delta \hat{J}(t)}^+ \\
    &+ \ev{\int_0 ^{\tau} \dd t' \int_0 ^{\tau} \dd t {J}(t'){J}(t)}^+\\
\end{align}
となる。ここで、この最後の式の第二項は、$\Delta \Pi_X=0$で、$\Delta \Pi_X$について微分することで0になる。これは、$(\int_0 ^{\tau} \dd t \ev{\hat{J}(t))^2}$が、$\Delta \Pi_X$についての偶関数であることからわかる。\\
したがって、
\begin{align} 
    \left.\pdv{\Delta \Pi_X} \ev{\int_0 ^{\tau} \dd t' \int_0 ^{\tau} \dd t \Delta \hat{J}(t')\Delta \hat{J}(t)}^+\right|_{\Delta \Pi_X = 0} = \pdv[2]{\Delta \Pi_X} \ev{\int_0 ^{\tau} \dd t \hat{J}(t)}^+|_{\Delta \Pi_X = 0}
\end{align}
が成り立つ。ここで、両辺$\frac{1}{\tau}$をかけ、$\tau \to \infty$の極限を取ると、前回のゼミと同様の変形をすることで、
\begin{align}
    2\left.\pdv{\Delta \Pi_X} \int_0 ^{\infty} \dd t \ev{\Delta \hat{J}(t) \Delta \hat{J}(0)}^+\right|_{\Delta \Pi_X = 0} = \left.\pdv[2]{\Delta \Pi_X} \ev{\Delta \hat{J}}^+\right|_{\Delta \Pi_X = 0}
\end{align}
を得る。\hfill \qedsymbol

\subsection{従来の線形応答理論との違い}
なんかいろいろ書いてる。\\

\section{Onsagerの相反定理}
二つの示量変数$X$と$Y$をもつ系を考える。対応する流れをそれぞれ$J_X$と$J_Y$とする。系が四つのbathと接触しているとする。
そのうち二つが、$\Pi_X$に対応するものとし、残り二つが、$\Pi_Y$に対応するものとする。このとき、一般に、もし$\Pi_X$が$0$でないならば、$\Pi_Y=0$であっても、$J_Y=0$であるとは限らない。\\

\begin{itembox}[l]{\textbf{Def:Onsager行列}}
    Onsager行列$L$は、
    \begin{equation}
        L_{ij} = \left.\pdv{J_i}{\Delta \Pi_j}\right|_{\Delta \Pi_X=\Delta \Pi_Y=0} \quad (i,j = X,Y)
    \end{equation}
    で定義される。
\end{itembox}
このとき、
\begin{align}
    J_X &= L_{XX}\Delta\Pi_X + L_{XY}\Delta\Pi_Y\\
    J_Y &= L_{YX}\Delta\Pi_X + L_{YY}\Delta\Pi_Y
\end{align}
と書ける。\\
\begin{itembox}[l]{\textbf{Thm:Onsagerの相反定理}}
    時間反転対称性を破るような場がないとする。このとき、$\Delta \Pi_Y$に対する$J_X$の応答は、$\Delta \Pi_X$に対する$J_Y$の応答と等しい。すなわち、
    \begin{equation}
        L_{XY} = L_{YX}
    \end{equation}
    が成り立つ。
\end{itembox}
\textbf{Prf}\\
\begin{align}
    \ev{\hat{\mathcal{J}}_X} &= \int \dd \hat{\mathcal{J}}_X \int \dd \hat{\mathcal{J}}_Y \hat{\mathcal{J}}_X P(\hat{\mathcal{J}}_X, \hat{\mathcal{J}}_Y)\\
    &= \int \dd \hat{\mathcal{J}}_X \int \dd \hat{\mathcal{J}}_Y (\hat{\mathcal{J}}_X P(\hat{-\mathcal{J}}_X, \hat{-\mathcal{J}}_Y))exp(\hat{\mathcal{J}}_X \Delta \Pi_X + \hat{\mathcal{J}}_Y \Delta \Pi_Y)\\
    &= -\int \dd \hat{\mathcal{J}}_X \int \dd \hat{\mathcal{J}}_Y \hat{\mathcal{J}}_X P(\hat{\mathcal{J}}_X, \hat{\mathcal{J}}_Y)exp(-\hat{\mathcal{J}}_X \Delta \Pi_X - \hat{\mathcal{J}}_Y \Delta \Pi_Y)\\
    &= -\sum_{k=1}^{\infty} \frac{\ev{\hat{\mathcal{J}}_X(-\hat{\mathcal{J}}_X\Delta \Pi_X-\hat{\mathcal{J}}_Y\Delta \Pi_Y)^{k-1}}}{(k-1)!}
\end{align}
となる。ここで、
\begin{align}
    \pdv{\ev{\hat{\mathcal{J}}_X}}{\Delta \Pi_Y} &= -\pdv{\ev{\hat{\mathcal{J}}_X}}{\Delta \Pi_Y} + \pdv{\Delta \Pi_Y}\sum_{k=2}^{\infty} \frac{\ev{\hat{\mathcal{J}}_X(-\hat{\mathcal{J}}_X\Delta \Pi_X-\hat{\mathcal{J}}_Y\Delta \Pi_Y)^{k-1}}}{(k-1)!}\\
    &= -\pdv{\ev{\hat{\mathcal{J}}_X}}{\Delta \Pi_Y} - \sum_{k=2}^{\infty} \frac{\ev{\hat{\mathcal{J}}_X \hat{\mathcal{J}}_Y(-\hat{\mathcal{J}}_X\Delta \Pi_X-\hat{\mathcal{J}}_Y\Delta \Pi_Y)^{k-2}}}{(k-2)!}
\end{align}
となる。$\Delta \Pi_Y$の0次の項を比較することで、
\begin{align}
    \pdv{\ev{\hat{\mathcal{J}}_X}}{\Delta \Pi_Y} = -\pdv{\ev{\hat{\mathcal{J}}_X}}{\Delta \Pi_Y} + \ev{\hat{\mathcal{J}}_X \hat{\mathcal{J}}_Y}_0
\end{align}
を得る。整理して、
\begin{align}
    2\pdv{\ev{\hat{\mathcal{J}}_X}}{\Delta \Pi_Y} = \ev{\hat{\mathcal{J}}_X \hat{\mathcal{J}}_Y}_0
\end{align}
を得る。同様にして、
\begin{align}
    2\pdv{\ev{\hat{\mathcal{J}}_Y}}{\Delta \Pi_X} = \ev{\hat{\mathcal{J}}_X \hat{\mathcal{J}}_Y}_0
\end{align}
を得る。これらを比較することで、
\begin{align}
    L_{XY} = L_{YX}
\end{align}
を得る。\hfill \qedsymbol

\begin{itembox}[l]{\textbf{Thm:時間反転対称性を破る場があるときのOnsagerの相反定理}}
    例えば、磁場中の系を考えると、Onsagerの相反定理は以下のように拡張される:
    \begin{align}
        L_{XY} (B) = L_{YX} (-B)
    \end{align}
\end{itembox}
\textbf{Prf}\\
磁場が時間反転対称性を破ることを考えると、DFTは、
\begin{align}
    P(\hat{\mathcal{J}}_X, \hat{\mathcal{J}}_Y;B) = P(-\hat{\mathcal{J}}_X, -\hat{\mathcal{J}}_Y;-B)exp(\hat{\mathcal{J}}_X \Delta \Pi_X + \hat{\mathcal{J}}_Y \Delta \Pi_Y)
\end{align}
となる。このとき、
\begin{align}
    \label{eq:a}
    L_{XY} (B) + L_{XY} (-B) = \frac{1}{\tau} \ev{\hat{\mathcal{J}}_X \hat{\mathcal{J}}_Y}_0^{-B}
\end{align}
となる。\\
(\ref{eq:a})の証明\\
上で証明したOnsagerの相反定理の証明を参考にすると、
\begin{align}
    \ev{\hat{\mathcal{J}}_X} ^B &= \int \dd \hat{\mathcal{J}}_X \int \dd \hat{\mathcal{J}}_Y \hat{\mathcal{J}}_X P(\hat{\mathcal{J}}_X, \hat{\mathcal{J}}_Y;B)\\
    &= \int \dd \hat{\mathcal{J}}_X \int \dd \hat{\mathcal{J}}_Y \hat{\mathcal{J}}_X P(-\hat{\mathcal{J}}_X, -\hat{\mathcal{J}}_Y;-B)exp(\hat{\mathcal{J}}_X \Delta \Pi_X + \hat{\mathcal{J}}_Y \Delta \Pi_Y)\\
    &= -\int \dd \hat{\mathcal{J}}_X \int \dd \hat{\mathcal{J}}_Y \hat{\mathcal{J}}_X P(\hat{\mathcal{J}}_X, \hat{\mathcal{J}}_Y;-B)exp(-\hat{\mathcal{J}}_X \Delta \Pi_X - \hat{\mathcal{J}}_Y \Delta \Pi_Y)\\
    &= -\sum_{k=1}^{\infty} \frac{\ev{\hat{\mathcal{J}}_X(-\hat{\mathcal{J}}_X\Delta \Pi_X-\hat{\mathcal{J}}_Y\Delta \Pi_Y)^{k-1}}^{-B}}{(k-1)!}
\end{align}
となる。ここで、
\begin{align}
    \pdv{\ev{\hat{\mathcal{J}}_X}^B}{\Delta \Pi_Y} &= -\pdv{\ev{\hat{\mathcal{J}}_X}^{-B}}{\Delta \Pi_Y} - \sum_{k=2}^{\infty} \frac{\ev{\hat{\mathcal{J}}_X\hat{\mathcal{J}}_Y(-\hat{\mathcal{J}}_X\Delta \Pi_X-\hat{\mathcal{J}}_Y\Delta \Pi_Y)^{k-2}}^{-B}}{(k-2)!}
\end{align}
となる。$\Delta \Pi_Y$の0次の項を比較することで、
\begin{align}
    \pdv{\ev{\hat{\mathcal{J}}_X}^B}{\Delta \Pi_Y} + \pdv{\ev{\hat{\mathcal{J}}_X}^{-B}}{\Delta \Pi_Y} = \ev{\hat{\mathcal{J}}_X \hat{\mathcal{J}}_Y}_0^{-B}
\end{align}
を得る。したがって、
\begin{align}
    L_{XY} (B) + L_{XY} (-B) = \frac{1}{\tau} \ev{\hat{\mathcal{J}}_X \hat{\mathcal{J}}_Y}_0^{-B}
\end{align}
を得る。\\
%TODO:最後\tauで割る作業を確認する。(これ多分Jはアンサンブル平均で、それに\tauかけたものが\mathcal{J})に等しくなることを用いている\\
また、
\begin{align}
    \label{74}
    L_{XY} (-B) + L_{YX} (-B) = \frac{1}{\tau} \ev{\hat{\mathcal{J}}_X \hat{\mathcal{J}}_Y}_0^{-B}
\end{align}
となる。\\
(\ref{74})式の証明\\
磁場が$-B$のときの系について、IFTを用いることにより、
\begin{align}
    1=\ev{e^{-\hat{\sigma}}}^{-B} &= \ev{exp(-\mathcal{J}_X \Delta \Pi_X - \mathcal{J}_Y \Delta \Pi_Y)}^{-B}\\
    &= 1-\ev{\mathcal{J}_X}^{-B} \Delta \Pi_X - \ev{\mathcal{J}_Y}^{-B} \Delta \Pi_Y \\
    &\quad+ \frac{1}{2}\ev{\mathcal{J}_X^2}^{-B} \Delta \Pi_X^2 + \ev{\mathcal{J}_X \mathcal{J}_Y}^{-B} \Delta \Pi_X \Delta \Pi_Y \\
    &\quad+ \frac{1}{2}\ev{\mathcal{J}_Y^2}^{-B} \Delta \Pi_Y^2 + O(\Delta \Pi^3)
\end{align}
を得る。ここで、
\begin{align}
    \ev{\hat{\mathcal{J}}_X}^{-B} = \ev{\hat{\mathcal{J}}_X}_0 + \left.\pdv{\Delta \Pi_X} \ev{\hat{\mathcal{J}}_X}\right|_{\Delta \Pi_X = 0} \Delta \Pi_X +\left.\pdv{\Delta \Pi_Y} \ev{\hat{\mathcal{J}}_X}\right|_{\Delta \Pi_Y = 0} \Delta \Pi_Y + O(\Delta \Pi^2)
\end{align}
と、
\begin{align}
    \ev{\hat{\mathcal{J}}_Y}^{-B} = \ev{\hat{\mathcal{J}}_Y}_0 + \left.\pdv{\Delta \Pi_X} \ev{\hat{\mathcal{J}}_Y}\right|_{\Delta \Pi_X = 0} \Delta \Pi_X +\left.\pdv{\Delta \Pi_Y} \ev{\hat{\mathcal{J}}_Y}\right|_{\Delta \Pi_Y = 0} \Delta \Pi_Y + O(\Delta \Pi^2)
\end{align}
および、
\begin{align}
    \ev{\hat{\mathcal{J}}_X \hat{\mathcal{J}}_Y}^{-B} = \ev{\hat{\mathcal{J}}_X \hat{\mathcal{J}}_Y}^{-B}_0 +O(\Delta \Pi)
\end{align}
であるから、(76)式に代入して、$\Delta \Pi_X \Delta \Pi_Y$の項を比較することで、
\begin{align}
    \left.\pdv{\Delta \Pi_Y} \ev{\hat{\mathcal{J}_X}}^{-B}\right|_{\Delta \Pi_X = 0} + \left.\pdv{\Delta \Pi_X} \ev{\hat{\mathcal{J}_Y}}^{-B}\right|_{\Delta \Pi_Y = 0} = \ev{\hat{\mathcal{J}}_X \hat{\mathcal{J}_Y}}^{-B}_0
\end{align}
を得る。よって、
\begin{align}
    L_{XY} (-B) + L_{YX} (-B) = \frac{1}{\tau}\ev{\hat{\mathcal{J}}_X \hat{\mathcal{J}_Y}}^{-B}_0
\end{align}
を得る。\\


この二式から、
\begin{align}
    L_{XY} (B) = L_{YX} (-B)
\end{align}
を得る。\hfill \qedsymbol
\end{document}