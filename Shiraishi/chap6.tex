\documentclass[a4paper,11pt]{jsarticle}

% 数式
\usepackage{amsmath,amsfonts}
\usepackage{amsthm}
\usepackage{bm}
\usepackage{mathtools}

% 表
\usepackage[utf8]{inputenc}
\usepackage{diagbox} % 斜線付きセルを作成するために必要
\usepackage{booktabs} % 表の罫線を美しくするために必要
\usepackage{hhline} % 水平罫線を制御するために必要

% 画像
\usepackage[dvipdfmx]{graphicx}
\usepackage{ascmac}
\usepackage{physics}
\usepackage{float} % 追加

% 図
\usepackage[dvipdfmx]{graphicx}
\usepackage{tikz} %図を描く
\usetikzlibrary{positioning, intersections, calc, arrows.meta,math} %tikzのlibrary

% ハイパーリンク
\usepackage[dvipdfm,
  colorlinks=false,
  bookmarks=true,
  bookmarksnumbered=false,
  pdfborder={0 0 0},
  bookmarkstype=toc]{hyperref}

\begin{document}

\title{title}
\author{大上由人}
\date{\today}
\maketitle
\section*{6章}
\subsection[6.1]{熱力学第二法則}
\subsubsection[6.1.1]{導出}
\begin{itembox}[l]{\textbf{Thm:熱力学第二法則}}
    任意のIFTを満たす系において、エントロピー生成のアンサンブル平均は非負である:
    \begin{equation}
        \ev{\hat{\sigma}} \geq 0
    \end{equation}

\end{itembox}
\textbf{Prf}\\
Jensenの不等式を用いると、
\begin{align}
    1 &= \ev{\hat{\sigma}}\\
    &= \int d\Gamma P(\Gamma) exp(-\sigma(\Gamma))\\
    &\geq exp(-\int d\Gamma P(\Gamma)\sigma(\Gamma))\\
    &= exp(-\ev{\sigma})
\end{align}
であるから、両辺logをとると
\begin{equation}
    \ev{\sigma} \geq 0
\end{equation}
が成り立つ。\qedsymbol

また、Jarezynski等式をを用いると、非平衡系における最大仕事の原理が得られる。
\begin{itembox}[l]{\textbf{Prop:最大仕事の原理(非平衡)}}
    非平衡系において、
    \begin{equation}
        \ev{W} \leq -\Delta F
    \end{equation}
    が成り立つ。
\end{itembox}
\textbf{Prf}\\
Jarezynski等式より、
\begin{equation}
    exp(-\beta \Delta F) = \ev{exp(\beta W)}
\end{equation}
であるから、右辺にJensenの不等式を適用すると
\begin{align}
    exp(-\beta \Delta F) &= \ev{exp(\beta W)}\\
    &\geq exp(\beta \ev{W})
\end{align}
となり、両辺logをとると
\begin{equation}
    \ev{W} \leq -\Delta F
\end{equation}
が成り立つ。\qedsymbol

ただし、このとき注意しなければならない点は、平衡熱力学とは異なり、仕事が断熱仕事で定義されている点である。(平衡熱力学においては、等温操作について考えていた。)\\
したがって、任意の経路において、上の不等式は等号を満たすことはない。これは、ゆらぎの定理において、確率的にエントロピー生成が負になることと関連している。\\

\subsubsection[6.1.2]{古典極限}
系を十分大きくしたときに、負のエントロピー生成が0に収束することを示す。\\
\begin{itembox}[l]{\textbf{Thm:第二法則に反する確率}}
    系の大きさを大きくしていったとき、エントロピー生成の確率は0に収束する:
    \begin{equation}
        \lim_{V \to \infty} \mathrm{Prob}\left(\frac{\hat{\sigma}}{V} < -\delta\right) = 0
    \end{equation}

\end{itembox}
\textbf{Prf}\\


\end{document}