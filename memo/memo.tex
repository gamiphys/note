\documentclass[a4paper,11pt]{jsarticle}

% 数式
\usepackage{amsmath,amsfonts}
\usepackage{amsthm}
\usepackage{bm}
\usepackage{mathtools}
\usepackage{amssymb}

% 表
\usepackage[utf8]{inputenc}
\usepackage{diagbox} % 斜線付きセルを作成するために必要
\usepackage{booktabs} % 表の罫線を美しくするために必要
\usepackage{hhline} % 水平罫線を制御するために必要

% 画像
\usepackage[dvipdfmx]{graphicx}
\usepackage{ascmac}
\usepackage{physics}
\usepackage{float} % 追加

% 図
\usepackage[dvipdfmx]{graphicx}
\usepackage{tikz} %図を描く
\usetikzlibrary{positioning, intersections, calc, arrows.meta,math} %tikzのlibrary

% ハイパーリンク
\usepackage[dvipdfm,
  colorlinks=false,
  bookmarks=true,
  bookmarksnumbered=false,
  pdfborder={0 0 0},
  bookmarkstype=toc]{hyperref}

% 式番号を章ごとにリセット
\numberwithin{equation}{section}

\begin{document}

\title{memo}
\author{大上由人}
\date{\today}
\maketitle

\section{須藤相対論問題6.10}
\subsection{個数密度}
個数密度を求める。$f(q)$が、運動量空間で考えたときの、粒子の状態の数の密度である(分布関数)ことを踏まえると、個数密度は
\begin{align}
    n = \frac{1}{h^3}g \int f(q)d^3q
\end{align}
と書ける。ここで、$g$はスピンの自由度である。これを極座標系に変換して、角度成分の計算を処理することで、
\begin{align}
    n = \frac{g}{h^3} 4\pi \int_0^\infty f(q)q^2dq
\end{align}
となる。

\end{document}