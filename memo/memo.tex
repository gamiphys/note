\documentclass{article}
\usepackage{amsmath}
\usepackage{amssymb}
\usepackage{bm}

\begin{document}

\section*{問題 1}

平面波を重み $g(\bm{k})$ で重ね合わせた波束
\begin{equation}
    \phi(\bm{x}, t) = \frac{1}{(2\pi)^{3/2}} \iiint g(\bm{k}) e^{i \bm{k} \cdot \bm{x} - i \omega(\bm{k}) t} d^3 k
\end{equation}
を考える。

(1) $t=0$ において、
\begin{equation}
    \phi(\bm{x}, 0) = C e^{i \bm{k}_0 \cdot \bm{x}} \exp \left( - \frac{\bm{x}^2}{2\sigma} \right)
\end{equation}
であるとする。($C$ は複素定数、$\bm{k}_0$, $\sigma$ は実定数。)

この $\phi(\bm{x}, 0)$ を Fourier 変換することで重み $g(\bm{k})$ が
\begin{equation}
    g(\bm{k}) = C \sigma^{3/2} \exp \left( - \frac{\sigma}{2} (\bm{k} - \bm{k}_0)^2 \right)
\end{equation}
と表されることを示せ。

関数 $f(\bm{x})$ の Fourier 変換は
\begin{equation}
    F(\bm{k}) = \frac{1}{(2\pi)^{3/2}} \iiint f(\bm{x}) e^{-i \bm{k} \cdot \bm{x}} d^3 x
\end{equation}
Fourier 逆変換は
\begin{equation}
    f(\bm{x}) = \frac{1}{(2\pi)^{3/2}} \iiint F(\bm{k}) e^{i \bm{k} \cdot \bm{x}} d^3 k
\end{equation}
で定義する。

\subsection*{解答}

与えられた $\phi(\bm{x}, 0)$ を Fourier 変換する。Fourier 変換の定義より、

\begin{align}
    g(\bm{k}) &= \frac{1}{(2\pi)^{3/2}} \iiint \phi(\bm{x}, 0) e^{-i \bm{k} \cdot \bm{x}} d^3 x \\
    &= \frac{1}{(2\pi)^{3/2}} \iiint C e^{i \bm{k}_0 \cdot \bm{x}} \exp \left( - \frac{\bm{x}^2}{2\sigma} \right) e^{-i \bm{k} \cdot \bm{x}} d^3 x \\
    &= \frac{C}{(2\pi)^{3/2}} \iiint \exp \left( i (\bm{k}_0 - \bm{k}) \cdot \bm{x} \right) \exp \left( - \frac{\bm{x}^2}{2\sigma} \right) d^3 x
\end{align}

ここで、指数関数の積をまとめて書き直すと、
\begin{equation}
    g(\bm{k}) = \frac{C}{(2\pi)^{3/2}} \iiint \exp \left( - \frac{\bm{x}^2}{2\sigma} + i (\bm{k}_0 - \bm{k}) \cdot \bm{x} \right) d^3 x
\end{equation}

次に、$\bm{x}$ のガウス積分を評価する。一般的なガウス積分の公式
\begin{equation}
    \iiint \exp \left( -a \bm{x}^2 + \bm{b} \cdot \bm{x} \right) d^3 x = \left( \frac{\pi}{a} \right)^{3/2} \exp \left( \frac{\bm{b}^2}{4a} \right)
\end{equation}
を用いると、ここで $a = \frac{1}{2\sigma}$、$\bm{b} = i (\bm{k}_0 - \bm{k})$ であるため、

\begin{equation}
    g(\bm{k}) = \frac{C}{(2\pi)^{3/2}} \left( 2\pi\sigma \right)^{3/2} \exp \left( -\frac{\sigma}{2} (\bm{k} - \bm{k}_0)^2 \right)
\end{equation}

以上から、
\begin{equation}
    g(\bm{k}) = C \sigma^{3/2} \exp \left( - \frac{\sigma}{2} (\bm{k} - \bm{k}_0)^2 \right)
\end{equation}
が得られる。証明終了。

\end{document}
