\documentclass[a4paper,11pt]{jsarticle}

% 数式
\usepackage{amsmath,amsfonts}
\usepackage{amsthm}
\usepackage{bm}
\usepackage{mathtools}
\usepackage{amssymb}

% 表
\usepackage[utf8]{inputenc}
\usepackage{diagbox} % 斜線付きセルを作成するために必要
\usepackage{booktabs} % 表の罫線を美しくするために必要
\usepackage{hhline} % 水平罫線を制御するために必要

% 画像
\usepackage[dvipdfmx]{graphicx}
\usepackage{ascmac}
\usepackage{physics}
\usepackage{float} % 追加

% 図
\usepackage[dvipdfmx]{graphicx}
\usepackage{tikz} %図を描く
\usetikzlibrary{positioning, intersections, calc, arrows.meta,math} %tikzのlibrary

% ハイパーリンク
\usepackage[dvipdfm,
  colorlinks=false,
  bookmarks=true,
  bookmarksnumbered=false,
  pdfborder={0 0 0},
  bookmarkstype=toc]{hyperref}

% 式番号を章ごとにリセット
\numberwithin{equation}{section}

\begin{document}

\title{メモ}
\author{大上由人}
\date{\today}
\maketitle

\part{数学}
\section{線形代数}

\begin{itembox}[l]{\textbf{Thm:}}
  実$n$次元数ベクトル空間$\mathbb{R}^n$の任意の部分ベクトル空間$W_1, W_2$に対して、
  \begin{align}
    W_1 \subset W_2 \Rightarrow \dim W_1 \leq \dim W_2
  \end{align}
  が成立する。また、
  \begin{align}
    W_1 \subset W_2 \quad \text{and} \quad \dim W_1 = \dim W_2 \quad \Rightarrow \quad W_1 = W_2
  \end{align}
  が成立する。
\end{itembox}
\textbf{Prf}\\
%後で書く。\\

\section{微分積分学}

\section{確率統計}

\section{幾何学}

\section{その他}
\subsection{計算の類}
\textbf{ガウス積分}\\
ガウス積分の変わったやり方を書く。\\
\begin{align}
  \int_{-\infty}^{\infty} x^2 e^{-ax^2} dx
  &=-\int_{-\infty}^{\infty} \dv{a} e^{-ax^2} dx\\
  &=-\dv{a} \int_{-\infty}^{\infty} e^{-ax^2} dx\\
  &=-\dv{a} \sqrt{\frac{\pi}{a}}\\
  &=\frac{\sqrt{\pi}}{2a^{3/2}}
\end{align}
により計算できる。\\

\part{物理}
\section{古典力学}

\section{電磁気学}

\section{熱力学}

\section{統計力学}

\section{非平衡統計力学}

\section{量子力学}

\section{相対論}

\section{場の量子論}

\section{その他}





\end{document}