\documentclass[a4paper,11pt]{jsarticle}

% 数式
\usepackage{amsmath,amsfonts}
\usepackage{amsthm}
\usepackage{bm}
\usepackage{mathtools}
\usepackage{amssymb}

% 表
\usepackage[utf8]{inputenc}
\usepackage{diagbox} % 斜線付きセルを作成するために必要
\usepackage{booktabs} % 表の罫線を美しくするために必要
\usepackage{hhline} % 水平罫線を制御するために必要

% 画像
\usepackage[dvipdfmx]{graphicx}
\usepackage{ascmac}
\usepackage{physics}
\usepackage{float} % 追加

% 図
\usepackage[dvipdfmx]{graphicx}
\usepackage{tikz} %図を描く
\usetikzlibrary{positioning, intersections, calc, arrows.meta,math} %tikzのlibrary

% ハイパーリンク
\usepackage[dvipdfm,
  colorlinks=false,
  bookmarks=true,
  bookmarksnumbered=false,
  pdfborder={0 0 0},
  bookmarkstype=toc]{hyperref}

% 式番号を章ごとにリセット
\numberwithin{equation}{section}

\begin{document}

\section*{4 正規直交標構を使う方法 (orthonormal frame)}

曲面 $\vb{p} = \vb{p}(u, v)$ が与えられたとき、$\mathcal{S}$ の各点で接平面内の正規直交基底 $\mathbf{e}_1, \mathbf{e}_2$ をとる:

\begin{equation}
\mathbf{e}_1 \cdot \mathbf{e}_1 = \mathbf{e}_2 \cdot \mathbf{e}_2 = 1, \quad \mathbf{e}_1 \cdot \mathbf{e}_2 = 0
\end{equation}

\begin{equation}
\mathbf{e}_3 = \mathbf{e}_1 \times \mathbf{e}_2 \quad : \text{単位法線ベクトル}
\end{equation}

$(\mathbf{e}_1, \mathbf{e}_2, \mathbf{e}_3)$ は右手系 $\in SO(3)$

\textbf{Rem.} $\mathbf{e}_1, \mathbf{e}_2$ の与え方の一例としては $\vb{p}_u, \vb{p}_v$ からシュミットの直交化法を使うとよい。

\begin{equation}
\mathbf{e}_1 = \frac{\vb{p}_u}{|\vb{p}_u|}, \quad \mathbf{e}_2 = \frac{\vb{p}_v - (\vb{p}_v \cdot \mathbf{e}_1)\mathbf{e}_1}{|\vb{p}_v - (\vb{p}_v \cdot \mathbf{e}_1)\mathbf{e}_1|}
\end{equation}

\subsection*{基底のとりかえ行列}

\begin{equation}
(\vb{p}_u, \vb{p}_v) = (\mathbf{e}_1, \mathbf{e}_2) A \quad A = 
\begin{pmatrix}
a_1^1 & a_1^2 \\
a_2^1 & a_2^2
\end{pmatrix}
\end{equation}

\begin{equation}
\begin{cases}
\vb{p}_u = a_1^1 \mathbf{e}_1 + a_1^2 \mathbf{e}_2 \\
\vb{p}_v = a_2^1 \mathbf{e}_1 + a_2^2 \mathbf{e}_2
\end{cases}
\end{equation}

\textbf{Rem + 仮定} $\mathbf{e}_1, \mathbf{e}_2$ は $\det A > 0$ となるようにとる。

$\det A < 0$ のときは $\mathbf{e}_1, \mathbf{e}_2$ を入れ替えて改めて $\mathbf{e}_1, \mathbf{e}_2$ とすればよい。このとき $\mathbf{e}_3 = \mathbf{e}$ となる。

\begin{equation}
d\vb{p} = \vb{p}_u du + \vb{p}_v dv = (a_1^1 du + a_1^2 dv) \mathbf{e}_1 + (a_2^1 du + a_2^2 dv) \mathbf{e}_2
\end{equation}

\begin{equation}
d\vb{p} = \theta^1 \mathbf{e}_1 + \theta^2 \mathbf{e}_2
\end{equation}

\begin{equation}
I = d\vb{p} \cdot d\vb{p} = \theta^1 \theta^1 + \theta^2 \theta^2
\end{equation}

\end{document}