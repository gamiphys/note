\documentclass[a4paper,11pt]{jsarticle}

% 数式
\usepackage{amsmath,amsfonts}
\usepackage{amsthm}
\usepackage{bm}
\usepackage{mathtools}
\usepackage{amssymb}

% 表
\usepackage[utf8]{inputenc}
\usepackage{diagbox} % 斜線付きセルを作成するために必要
\usepackage{booktabs} % 表の罫線を美しくするために必要
\usepackage{hhline} % 水平罫線を制御するために必要

% 画像
\usepackage[dvipdfmx]{graphicx}
\usepackage{ascmac}
\usepackage{physics}
\usepackage{float} % 追加

% 図
\usepackage[dvipdfmx]{graphicx}
\usepackage{tikz} %図を描く
\usetikzlibrary{positioning, intersections, calc, arrows.meta,math} %tikzのlibrary

% ハイパーリンク
\usepackage[dvipdfm,
  colorlinks=false,
  bookmarks=true,
  bookmarksnumbered=false,
  pdfborder={0 0 0},
  bookmarkstype=toc]{hyperref}

% 式番号を章ごとにリセット
\numberwithin{equation}{section}

\begin{document}

\title{熱力学的不確定性関係(ゆり京)}
\author{gami\footnote{Twitter :@gami\_physics}}
\date{\today}
\maketitle

非平衡統計力学の枠組みでは、ゆらぎの定理から、様々な不等式が導出される。例えば、熱力学第二法則は、IFT(積分型ゆらぎの定理)にJensenの不等式を適用することで導出される。

しかし、ゆらぎの定理以外から得られる不等式も存在する。その一つが、熱力学的不確定性関係(Thermodynamic Uncertainty Relation,TUR)型の不等式である。
\footnote{"TUR型"と呼んでいるのは、今回紹介する不等式の他にも、似た導出が可能な不等式が存在するためである。}
TURは、カレントのゆらぎとエントロピー生成との間のトレードオフを示す不等式である。以下のノートでは、定常状態におけるTUR及び、KURと呼ばれる不等式について紹介する。



\section{TUR}
\subsection{準備}
\subsubsection{記号}
まず、経路に対して値を返す量(path quantity)の記号を定義する。
\begin{itemize}
    \item  $ \hat{\hat{J}}_{ij} =\sum_{n} (\delta_{\omega_j \to \omega_i}(\omega^{n-1}\to \omega^n) - \delta_{\omega_i \to \omega_j}(\omega^{n-1}\to \omega^n)) \delta(t-t_n)$:$i \to j$なる確率流
    \item $\hat{\hat{\mathcal{J}}}_{ij} = 
    % \sum_{n} (\delta_{\omega_j \to \omega_i}(\omega^{n-1}\to \omega^n) - \delta_{\omega_i \to \omega_j}(\omega^{n-1}\to \omega^n))$
    \int_{0}^{\tau} dt \hat{\hat{J}}_{ij}(t)$:$i \to j$なるカレントの積算
    \item $\hat{\hat{J}}_{d} = \sum_{(i,j)} d_{ij} \hat{\hat{J}}_{ij}$:$i \to j$の物理量のカレントの和
    \item $\hat{\hat{\mathcal{J}}}_{d} = \sum_{(i,j)} d_{ij} \hat{\hat{\mathcal{J}}}_{ij}$
:物理量のカレントの積算
    \item $d_{ij}:$添え字に対する反対称性$d_{ij} = -d_{ji}$を満たす物理量
\end{itemize}


また、ジャンプに対して値を返す量(jump quantity)として、対応する記号を定義する。
\begin{itemize}
    \item ${J}_{ij}(t) = R_{ij}p_j(t) - R_{ji}p_i(t)$
    \item $\qty(\hat{J}_{d}(t))_{ij} = d_{ij} {J}_{ij}(t)$
    \item $J_{d}(t) = \ev{(\hat{J}_{d}(t)_{ij})} = \sum_{(i,j)} d_{ij} {J}_{ij}(t)$
\end{itemize}

このとき、
\begin{align}
    J_{d} = \ev{\hat{J}_{d}} = \ev{\hat{\hat{J}}_{d}}
\end{align}
が言える。\footnote{補足に回す。}また、これを用いることで、
\begin{align}
    \ev{\hat{\hat{\mathcal{J}}}_{d} } = \int_{0}^{\tau} dt \ev{\hat{\hat{J}}_{d}} = \int_{0}^{\tau} dt J_{d}(t)
\end{align}
が成り立つ。
各ジャンプを足し合わせて経路量を作っていると考えると、期待値が一致するのは自然である。\\

\textbf{ex.熱流}\\
このとき、積算熱流は、$d_{ij} = E_i - E_{j}$とすることで、以下のように定義される:
\begin{align}
    \hat{\mathcal{J}}_{d} = \sum_{(i,j)} (E_i - E_j) \hat{\mathcal{J}}_{ij}
\end{align}

\subsubsection{TUR}
先に、今回の目標であるTURを紹介する。
\begin{itembox}[l]{\textbf{Thm.熱力学的不確定性関係}}
    定常Markov過程が局所詳細釣り合い条件を満たすとき、以下の関係が成り立つことが知られている:
    \begin{align}
        \frac{\text{Var}(\hat{\mathcal{J}}_d)}{\ev{\hat{\mathcal{J}}_d^{\text{ss}}}^2} \sigma \geq 2
    \end{align}
    ただし、
    \begin{align}
        \text{Var}(\hat{\mathcal{J}}_d) &= \ev{\qty(\hat{\mathcal{J}}_d -\ev{\hat{\mathcal{J}_d}})}^2
    \end{align}
    である。\footnote{期待値は経路に対してとっている。}
\end{itembox}
この不等式は、以下の3つの性質をまとめて表している。
\begin{itemize}
    \item カレントの分散を抑えるためには、相応のコスト(エントロピー生成)が必要である。
    \item カレントの分散を抑えるためには、カレントの大きさを小さくする必要性がある。
    \item カレントを大きくするためには、相応のコスト(エントロピー生成)が必要である。
\end{itemize}

\noindent
\subsubsection{情報論的不等式}
TURの証明には、情報論的不等式であるCram\'er-Raoの不等式を用いる。まず、Fisher情報量を定義する。
\begin{itembox}[l]{\textbf{Def:Fisher情報量}}
    パラメータ$\theta$をもつ確率分布$P_{\theta}(x)$が与えられたとき、Fisher情報量$F(\theta)$は以下のように定義される:
    \begin{align}
        F(\theta) = -\ev{\pdv[2]{\theta} \log P_{\theta}(x)}_{\theta} =\ev{\qty(\pdv{\theta} \log P_{\theta}(x))^2}_{\theta}
    \end{align}
\end{itembox}
このとき、Fisher情報量とKLダイバージェンスは以下のように関係している。
\begin{align}
    D(P_{\theta + d\theta} \| P_{\theta})
    % &= \int dx \, \pdv{}{\theta'} \left.\left( P_{\theta'}(x) \ln \frac{P_{\theta'}(x)}{P_{\theta}(x)} \right)\right|_{\theta' = \theta} d\theta \notag \\
    % &\quad + \frac{1}{2} \int dx \, \frac{\partial^2}{\partial \theta'^2} \left.\left( P_{\theta'}(x) \ln \frac{P_{\theta'}(x)}{P_{\theta}(x)} \right)\right|_{\theta' = \theta} (d\theta)^2 + O(d\theta^3) \notag \\
    % &= \frac{1}{2} \int dx \, \Bigg[ 2 \frac{1}{P_{\theta}(x)} \left( \pdv{}{\theta'} P_{\theta'}(x) \bigg|_{\theta' = \theta} \right)^2 \notag \\
    % &\quad + P_{\theta}(x) \frac{\partial^2}{\partial \theta'^2} \ln P_{\theta'}(x) \bigg|_{\theta' = \theta} \Bigg] (d\theta)^2 + O(d\theta^3) \notag \\
    % &= \frac{1}{2} \int dx \, P_{\theta}(x) \left( \pdv{}{\theta'} \ln P_{\theta'}(x) \bigg|_{\theta' = \theta} \right)^2 (d\theta)^2 + O(d\theta^3)\\
    &= \frac{1}{2} F(\theta) (d\theta)^2 + O(d\theta^3)
\end{align}
この表式を見ると、$\theta$を変化させたとき、KLダイバージェンスの意味での確率分布の距離がどれほど変化するかを表す量がFisher情報量であることがわかる。\\


$P_{\theta}(x)$をパラメータ$\theta$をもつ確率変数$x$の確率分布であるとする。また、引数$\theta$の関数$f(\theta)$が与えられていたとする。
ここでの目的は、$x$を$g(x)$として測定したとき、$f(\theta)$を推定することである。ここで、$g(x)$が$f(\theta)$の不偏推定量であると仮定する。すなわち、
\begin{align}
    f(\theta) = \ev{g(x)} 
\end{align}
が成り立つとする。
このとき、$g(x)$がどれほど正確な推定量なのかの限界を表すのがCram\'er-Raoの不等式である。\\
\begin{itembox}[l]{\textbf{Thm.一般化Cram\'er-Raoの不等式}}
    $g(x)$がパラメータ$f(\theta)$の不偏推定量であるとき、
    \begin{align}
        \text{Var}_{\theta}(g(x)) \geq \frac{(f'(\theta))^2}{F(\theta)}
    \end{align}
    が成り立つ。
\end{itembox}
KLダイバージェンスの文脈でCram\'er-Raoの不等式を見ると、
$\theta$の変化による確率分布間の距離が大きければ大きいほど、推定量の分散は小さくなるということがわかる。直感的には、確率分布同士が似ているよりも、はっきり異なるほうが推定がしやすい(見分けがつきやすい)
ということである。\\

\noindent
\textbf{Prf.}
\begin{align}
    \text{Var}_{\theta}g(x)F(\theta) &= \qty(\int \dd{x} (g(x) - f(\theta))^2 P_{\theta}(x))\qty(\int \dd{x} \qty(\pdv{\theta} \log P_{\theta}(x))^2 P_{\theta}(x))\\
    &\geq \qty(\int \dd{x} (g(x) - f(\theta)) \pdv{\theta} (\ln P_{\theta}(x)) P_{\theta}(x))^2 \quad \because \text{Cauchy-Schwarz の不等式}\\
    &= \qty(\int \dd{x} g(x) \pdv{\theta} P_{\theta}(x) - \int \dd{x} f(\theta) \pdv{\theta} P_{\theta}(x))^2\\
    &= \qty(\pdv{\theta} \int \dd{x} g(x) P_{\theta}(x) )^2\quad \because \text{第二項の積分は規格化条件より0}\\
    &= \qty(\pdv{\theta} f(\theta))^2 \quad \because \text{不偏推定量の仮定より}\\
\end{align}
\qed



\noindent

\subsection{TUR}
Cram\'er-Raoの不等式を用いて、TURを導出する。\\
\begin{itembox}[l]{\textbf{Thm.熱力学的不確定性関係}}
    定常Markov過程が局所詳細釣り合い条件を満たすとき、以下の関係が成り立つことが知られている:
    \begin{align}
        \frac{\text{Var}(\hat{\mathcal{J}}_d)}{\ev{\hat{\mathcal{J}}_d^{\text{ss}}}^2} \sigma \geq 2
    \end{align}
    ただし、
    \begin{align}
        \text{Var}(\hat{\mathcal{J}}_d) &= \ev{\qty(\hat{\mathcal{J}}_d -\ev{\hat{\mathcal{J}_d}})}^2
    \end{align}
    である。
\end{itembox}
% この不等式は、以下の3つの性質をまとめて表している。
% \begin{itemize}
%     \item カレントの分散を抑えるためには、相応のコスト(エントロピー生成)が必要である。
%     \item カレントの分散を抑えるためには、カレントの大きさを小さくする必要性がある。
%     \item カレントを大きくするためには、相応のコスト(エントロピー生成)が必要である。
% \end{itemize}

\textbf{Prf.}\\
パラメータ$\theta$つきの遷移レートを以下のように定義する。
\begin{align}
    R_{ij}^{\theta} &= R_{ij}e^{\theta Z_{ij}} \quad (i \neq j)\\
    R_{jj}^{\theta} &= -\sum_{i (\neq j)} R_{ij}^{\theta}e^{\theta Z_{ij}}\\
\end{align}
ただし、
\begin{align}
    Z_{ij} = \frac{R_{ij}p_j^\text{ss} - R_{ji}p_i^\text{ss}}{R_{ij}p_j^\text{ss} + R_{ji}p_i^\text{ss}}
\end{align}
である。この遷移レートで特徴づけられるような経路の確率分布を$P_{\theta}(\Gamma)$とする。\\
一般化Cram\'er-Raoの不等式
\begin{align}
    \text{Var}_{\theta}(g(x)) \geq \frac{(f'(\theta))^2}{F(\theta)}
\end{align}
において、
$x = \Gamma,g(\Gamma) = \hat{\mathcal{J}}_{d}, f(\theta) = \ev{g(\Gamma)}^{\text{ss}}_{\theta} = \ev{\hat{\mathcal{J}}_{d}}^{\text{ss}}_{\theta}$としたとき
の$\theta = 0$の場合を使う。すなわち、
\begin{align}
\eval{\left( \frac{\partial \langle {\hat{\mathcal{J}}_{d}} \rangle_\theta^{\text{ss}}}{\partial \theta} \right)^2}_{\theta = 0}
\leq \mathrm{Var}_{0} (\hat{\mathcal{J}}_{d}) F(0) \label{eq:1}
\end{align}
を用いる。\\


\textbf{Fisher情報量について}\\
\begin{align}
    F(0) &= - \int_0^\tau dt 
    \left[
        \sum_{i \neq j} R_{ij} p_j(t) 
        \eval{\frac{\partial^2}{\partial \theta^2} \ln R_{ij}^\theta}_{\theta = 0}
        + \sum_j p_j(t) 
        \eval{\frac{\partial^2}{\partial \theta^2} \ln e^{R_{jj}^\theta}}_{\theta = 0}
    \right] \nonumber \\
    &= \int_0^\tau dt \sum_j p_j(t) 
    \eval{\frac{\partial^2}{\partial \theta^2} 
        \sum_{k (\neq j)} R_{kj} e^{\theta Z_{kj}}}_{\theta = 0} \\
        &\quad \because \text{第一項は微分で消える。第二項について、}R_{jj} = -\sum_{k (\neq j)} R_{kj}  \\
    &= \int_0^\tau dt \sum_{k \neq j} R_{kj} p_j(t) Z_{kj}^2 \quad \because \text{微分を実行} \label{eq:ff}\\
    &= \int_0^\tau dt \, \ev{\hat{Z}^2} 
\end{align}
と計算できる。\footnote{最初の等号は後の方に書いてある。}

また、
\begin{align}
    \ev{\hat{Z}^2}^{\text{ss}}&= \frac{1}{2} \sum_{i,j} (Z_{ij}^2 p_j^\text{ss} R_{ij} + Z_{ji}^2 p_i^\text{ss} R_{ji})\\
    &= \frac{1}{2} \sum_{i,j} Z_{ij}^2(R_{ij} p_j^\text{ss} + R_{ji} p_i^\text{ss})\\
    &= \frac{1}{2} \sum_{i,j} \frac{(R_{ij} p_j^\text{ss} - R_{ji} p_i^\text{ss})^2}{R_{ij} p_j^\text{ss} + R_{ji} p_i^\text{ss}}\\
    &\leq \frac{1}{2}\qty(\frac{1}{2} \sum_{i,j} (R_{ij} p_j^\text{ss} - R_{ji} p_i^\text{ss})\ln \frac{R_{ij} p_j^\text{ss}}{R_{ji} p_i^\text{ss}}) \\
    &= \frac{1}{2} \sum_{i,j} R_{ij} p_j^\text{ss} \ln \frac{R_{ij} p_j^\text{ss}}{R_{ji} p_i^\text{ss}}\\
    &= \frac{\dot{\sigma}}{2} 
\end{align}
が得られる。ただし、不等号の部分については
\begin{align}
    (a-b)\ln \frac{a}{b} \geq  \frac{2(a - b)^2}{a + b}
\end{align}
を用いた。(補足に回す。)\\

\textbf{$f$について}\\
はじめの状態を、定常状態に取る。(定理の仮定より)このとき、$R_{ij} ^{\theta} = R_{ij}(1+\theta Z_{ij}) + O(\theta^2)$であることを用いて、
\begin{align}
    &\sum_{i}R_{ij}^{\theta} p_j^\text{ss} - R_{ji}^{\theta} p_i^\text{ss} \\
    =& \sum_{i}R_{ij} p_j^\text{ss} - R_{ji} p_i^\text{ss} + \theta (R_{ij} p_j^\text{ss} Z_{ij} - R_{ji} p_i^\text{ss} Z_{ji}) + O(\theta^2)\\
    =& \sum_{i}R_{ij} p_j^\text{ss} - R_{ji} p_i^\text{ss} + \theta Z_{ij}(R_{ij} p_j^\text{ss} + R_{ji} p_i^\text{ss}) + O(\theta^2)\\
    =& \sum_{i}R_{ij} p_j^\text{ss} - R_{ji} p_i^\text{ss} + \theta  (R_{ij} p_j^\text{ss} - R_{ji} p_i^\text{ss}) + O(\theta^2)\\
    =& O(\theta^2) \quad \because \text{定常分布}
\end{align}
となる。したがって、$\theta$系の定常状態は、$O(\theta)$までの近似で、$\theta = 0$の定常状態と一致する。このときの定常カレントは、
\begin{align}
   (\hat{J}_{ij})_\theta^{\text{ss}}:= R_{ij}^\theta p_j^\text{ss} - R_{ji}^\theta p_i^\text{ss} = O(\theta^2)
\end{align}
と書くことができる。したがって、
\begin{align}
    &\eval{\pdv{({{{J}}_{ij}})^{\text{ss}}_\theta}{\theta}}_{\theta = 0} \\
    =& \eval{\pdv{\theta} (R_{ij}^\theta p_j^\text{ss} - R_{ji}^\theta p_i^\text{ss})}_{\theta = 0} \\
    =&  R_{ij} Z_{ij} p_j^\text{ss} - R_{ji} Z_{ji} p_i^\text{ss} \\
    =&  Z_{ij} (R_{ij} p_j^\text{ss} + R_{ji} p_i^\text{ss}) \\
    =&  R_{ij}p_j^\text{ss} - R_{ji}p_i^\text{ss} \\
    =& {J}_{ij}^\text{ss}
\end{align}
となる。したがって、
\begin{align}
    \eval{\pdv{({{J}_{ij}})^{\text{ss}}_\theta}{\theta}}_{\theta = 0} = {J}_{ij}^{\text{ss}}
\end{align}
である。両辺$d_{ij}$をかけて和を取ることで、
\begin{align}
    \left.\sum_{(i,j)} \pdv{\theta} ({d_{ij}{J}_{ij}})^{\text{ss}}_\theta \right|_{\theta = 0} = \sum_{(i,j)} d_{ij} {J}_{ij}^{\text{ss}}
\end{align}
が成り立つ。すなわち、
\begin{align}
    \left.\pdv{\theta} \ev{\hat{{J}}_{d}}^{\text{ss}}_\theta\right|_{\theta = 0} = \ev{\hat{{J}}_{d}}^{\text{ss}}_{0}
\end{align}
が成り立つ。両辺時間について積分することにより、
\begin{align}
    \left.\pdv{\theta} \ev{\hat{{\mathcal{J}}_{d}}}^{\text{ss}}_\theta\right|_{\theta = 0} = \ev{\hat{{\mathcal{J}}_{d}}}^{\text{ss}}_{0}
\end{align}
が成り立つ。


以上の関係式を用いて、式(\ref{eq:1})を評価すると、
\begin{align}
    \qty(\ev{\hat{\mathcal{J}}_{d}}^{\text{ss}}_{0})^2\leq \eval{\left( \frac{\partial \langle {\hat{\mathcal{J}}_{d}} \rangle_\theta^{\text{ss}})^2}{\partial \theta} \right)^2}_{\theta = 0}
    \leq \mathrm{Var}_{0} (\hat{\mathcal{J}}_{d}) \int_0^\tau dt \, \langle \hat{Z}^2 \rangle \leq \frac{{\sigma}}{2} \mathrm{Var}_0 (\hat{\mathcal{J}}_{d})
\end{align}
が得られる。したがって、
\begin{align}
    \frac{\mathrm{Var}(\hat{\mathcal{J}}_{d})}{\qty(\ev{\hat{\mathcal{J}}_{d}}^{\text{ss}})^2} \sigma \geq 2
\end{align}
が成り立つ。\qed\\

\newpage
\section{KUR}
TUR型の別の不等式として、KUR(Kinetic Uncertainty Relation)がある。まず、記号を準備する。
\begin{itembox}[l]{\textbf{Def.Activity}}
    状態の組(i,j)に対するactivityは、
    \begin{align}
        \hat{A}_{ij} := \frac{1}{\tau} \sum_{n} \left( 
    \delta_{w_j \to w_i}(w^{n-1} \to w^n) + \delta_{w_i \to w_j}(w^{n-1} \to w^n) 
    \right)
    \end{align}
    で定義される。また、系全体のactivityは、
    \begin{align}
        \hat{A} := \sum_{(i,j)}  \hat{A}_{ij}
    \end{align}
    で定義される。
\end{itembox}
activityは、二つの状態間の遷移の活発さを表す量である。

\textbf{記号}
\begin{itemize}
    \item $\hat{X}$: jump quantity(jumpに対して値を返すような物理量のこと、値$X_{ij}$を返す)
    \item $\hat{\hat{X}} := \sum_{n} X_{ij}(\delta_{\omega_j \to \omega_{i}}(\omega^{n-1} \to \omega^{n})) \delta(t-t_n)$
    : jump quantityに対応するpath quantity\\
    このとき、 $\ev{\ev{\hat{\hat{X}}}}= \ev{\hat{X}}$が成立する。
    \item $\hat{\hat{\mathcal{X}}} := \int_{0}^{\tau} dt \hat{\hat{X}}(t)= \sum_{ij}\sum_{n}X_{ij}(\delta_{\omega_j \to \omega_{i}}(\omega^{n-1} \to \omega^{n}))$: jump quantityの積算
    \item $ A_{ij}(t) := \expval{\hat{A}_{ij}}_t = R_{ji} p_i^{\text{ss}}(t) + R_{ij} p_j^{\text{ss}}(t)$:平均activity
    \item $\mathcal{A} := \int_{0}^{\tau} dt \, A(t) := \int dt \sum_{i \neq j} R_{ij} p_j(t)$: 系の積算activity
\end{itemize}

\begin{align}
    \expval{\hat{1}} = \mathcal{A} := \int_{0}^{\tau} dt \, A(t) := \int dt \sum_{i \neq j} R_{ij} p_j(t).
\end{align}

\begin{itembox}[l]{\textbf{Thm.Kinetic uncertainty relation}}
    jump quantity$X$に対して、$\hat{\mathcal{X}}$のゆらぎは以下のように抑えられる。
    \begin{align}
        \frac{\operatorname{Var} \hat{\mathcal{X}}}{\left(\tau X(\tau)\right)^2} \mathcal{A} \geq 1
    \end{align}
\end{itembox}
要するに、ジャンプ回数が多ければ多いほど、ジャンプ量のゆらぎは小さくなるということである。\\

\textbf{Prf.}\footnote{多分時間が足りないので当日は追わない。}\\
$\theta-$遷移レートを以下のように定義する。
\begin{align}
    R_{ij}^{\theta} &= R_{ij}e^{\theta } \quad (i \neq j)\\
    R_{jj}^{\theta} &= -\sum_{i (\neq j)} R_{ij}^{\theta}e^{\theta }\\
\end{align}
Cram\'er-Raoの不等式を用いる。すなわち、
\begin{align}
    \eval{\left( \frac{\partial \langle {\hat{\mathcal{X}}_{d}} \rangle_\theta}{\partial \theta} \right)^2}_{\theta = 0}
    \leq \mathrm{Var}_{0} (\hat{\mathcal{X}}) F(0) \label{eq:2}
    \end{align}
    を用いる。

\textbf{Fisher情報量について}\\
\begin{align}
    F(0) &= \int_{0}^{\tau} dt \sum_{i \neq j} R_{ij} p_j(t) = \mathcal{A} \quad \because (\ref{eq:ff})\text{で}Z=1\text{とした。}
\end{align}

\textbf{$\ev{\hat{\mathcal{X}}}_{\theta}$について}\\
$\theta$系の時間発展は、
\begin{align}
    \pdv{p_i^{\theta}(t)}{t} &= \sum_{j} R_{ij}^{\theta}p_j^{\theta} (t)\\
    &= e^{\theta} \sum_{j} R_{ij}p_j^{\theta} (t)
\end{align}
となる。この式から、
\begin{align}
    p_i^{\theta}(t) = p_i(e^{\theta}t)
\end{align}
となっていることがわかる。(タイムスケールの変換)。これを用いると、
\begin{align}
    \pdv{\ev{\hat{\mathcal{X}}}_{\theta}}{\theta} &= \pdv{\theta}\int_{0}^{\tau} \dd{t} \sum_{i \neq j} R_{ij}^{\theta} X_{ij}p_j^{\theta}(t)\\
    &= \pdv{\theta}\int_{0}^{\tau} \dd{t} e^{\theta} \sum_{i \neq j} R_{ij} X_{ij}p_j(e^{\theta}t)\\
    &= \pdv{\theta}\int_{0}^{e^{\theta}\tau} \dd{t} \sum_{i \neq j} R_{ij} X_{ij}p_j(t)\\
    &\underset{\theta \to 0}{\longrightarrow} \tau \ev{\hat{X}(\tau)}
\end{align}
となる。\\
以上の結果とCram\'er-Raoの不等式を用いると、
\begin{align}
    (\tau \ev{\hat{X}(\tau)})^2 \leq \text{Var}_0(\hat{\mathcal{X}}) \mathcal{A}
\end{align}
となる。これより、
\begin{align}
    \frac{\text{Var}(\hat{\mathcal{X}})}{\qty(\tau \ev{\hat{X}(\tau)})^2} \mathcal{A} \geq 1
\end{align}
が成り立つことがわかる。\qed

\newpage
\section{補足}
\subsection{物理量の定義}
\subsubsection{物理量の分類}
ゆらぐ系の熱力学においては、基本的に以下の三種類の物理量が考えられる。\\
\begin{itembox}[l]{\textbf{Def.state quantity/jump quantity/path quantity}}
    state quantity $\hat{f}$は、状態 $j$に対して、値 $f_j$をとる物理量である。また、この期待値は、
    \begin{align}
        \ev{\hat{f}}_{\vb{p}} = \sum_{j} f_j p_j
    \end{align}
    で定義される。\\

    jump quantity $\hat{g}$は、状態 $j$から状態 $k$に遷移するとき、値 $g_{j\to k}$をとる物理量である。また、この期待値は、
    \begin{align}
        \ev{\hat{g}}_{\vb{p},R} = \sum_{j\neq k} R_{kj} p_j g_{j\to k}
    \end{align}
    で定義される。\\

    path quantity $\hat{\hat{F}}$は、経路$\Gamma$に対して、値 $F({\Gamma})$をとる物理量である。また、この期待値は、
    \begin{align}
        \ev{\ev{\hat{\hat{F}}}}_{\Gamma} = \int \dd{\Gamma} P(\Gamma) F(\Gamma)
    \end{align}
    で定義される。
\end{itembox}

\subsubsection{物理量の関係}
素朴には、各state/jump quantityを経路にわたって足し合わせることで、対応するpath quantityが定義できそうなものである。以下では、その素朴な定義が正しいことと、更にそれぞれの期待値が一致することを示す。\\

\noindent
\textbf{state quantityとpath quantityの関係}\\
時間に依存するstate quantity $\hat{f}(t)$について、対応するpath quantityを考えることができる。対応するpath quantity $\hat{\hat{f}}(t)$は、経路$\Gamma$に対して、値
\begin{align}
    f(\Gamma,t) = f_{\Gamma(t)} = \sum_{m=0}^{n} f_{j_{m}}(t) \chi[t \in [t_{m},t_{m+1}]]
\end{align}
をとる物理量である。このとき、以下が成り立つ。
\begin{itembox}[l]{\textbf{Prop.state quantityとpath quantityの関係}}
    state quantity $\hat{f}(t)$と対応するpath quantity $\hat{\hat{f}}(t)$について、以下が成り立つ。
    \begin{align}
        \ev{\hat{f}(t)}_{\vb{p}(t)} = \ev{\ev{\hat{\hat{f}}(t)}}_{\Gamma}
    \end{align}
\end{itembox}
\textbf{Prf.}\\
今回は使わないので略。\qed\\

\noindent
\textbf{jump quantityとpath quantityの関係}\\
時間に依存するjump quantity $\hat{g}(t)$について、対応するpath quantityを考えることができる。対応するpath quantity $\hat{\hat{g}}(t)$は、経路$\Gamma$に対して、値
\begin{align}
    g(\Gamma,t) = \sum_{m=1}^{n} g_{j_{m-1} \to j_{m}}(t) \delta(t-t_{m})
\end{align}
をとる物理量である。このとき、以下が成り立つ。
\begin{itembox}[l]{\textbf{Prop.jump quantityとpath quantityの関係}}
    jump quantity $\hat{g}(t)$と対応するpath quantity $\hat{\hat{g}}(t)$について、以下が成り立つ。
    \begin{align}
        \ev{\hat{g}(t)}_{\vb{p}(t),R(t)} = \ev{\ev{\hat{\hat{g}}(t)}}_{\Gamma}
    \end{align}
\end{itembox}
\textbf{Prf.}\\
離散のほうがやりやすいので、離散の場合を考える。右辺は以下のように書ける。
\begin{align}
    \int d\Gamma \, P(\Gamma) \sum_{m=1}^{n} g_{j_{m-1} \to j_{m}}(t) \delta_{t,t_m}
    &= \sum_{j_0, \dots, j_N} \prod_{n=1}^N T^{j^n j^{n-1}} p^0_{j^0} 
    \sum_{m=1}^{n} g_{j_{m-1} \to j_{m}}(t_m) 
\end{align}
一番最後の$\Sigma$について考える。和の$k$番目の項について、
\begin{align}
    (\text{第k項}) &= \sum_{j_0, \dots, j_N} \prod_{n=1}^N T^{j^n j^{n-1}} p^0_{j^0} g_{j_{k-1} \to j_{k}}(t_k) \\
    &(j^{0}\text{から}j^{k-2}\text{まで計算して、}) \nonumber \\
    &= \sum_{j_{k-1}, j_k, \cdots, j_N} \prod_{n=k-1}^N T^{j^n j^{n-1}} p^{k-1}_{j^{k-1}} g_{j_{k-1} \to j_{k}}(t_k) \\
    &(j^{k+1}\text{から}j^{N}\text{まで計算すると、規格化条件より}) \nonumber \\
    &= \sum_{j_{k-1}, j_k} T^{j^k j^{k-1}} p^{k-1}_{j^{k-1}} g_{j_{k-1} \to j_{k}}(t_k)
\end{align}
となる。これを連続極限に持っていくことで、
\begin{align}
    \int d\Gamma \, P(\Gamma) \sum_{m=1}^{n} g_{j_{m-1} \to j_{m}}(t) \delta_{t,t_m}
    &= \sum_{j\neq k} p_{j} R_{kj} g_{j \to k}(t)\\
    &= \ev{\hat{g}(t)}_{\vb{p}(t), R(t)}
\end{align}
となる。\qed

また、path quantityの積分量
\begin{align}
    {\hat{\hat{G}}}= \int_{0}^{\tau} \dd{t} \hat{\hat{g}}(t)
\end{align}
は、pathに対して値
\begin{align}
    G(\Gamma) = \int_{0}^{\tau} \dd{t} g(\Gamma,t) = \sum_{m=1}^{n} g_{j_{m-1} \to j_{m}}(t_{m})
\end{align}
をとる物理量である。このとき、以下が成り立つ。
\begin{align}
    \ev{\ev{\hat{\hat{G}}}}_{\Gamma} = \int \dd{t} \ev{\ev{\hat{\hat{g}}(t)}}_{\Gamma} = \int \dd{t} \ev{\hat{g}(t)}_{\vb{p}(t),R(t)}
\end{align}


\subsection{TURの証明で用いた不等式の証明}
TURの証明で用いた不等式を示す。$1/x$の凸性から、
\begin{align}
    \ln a - \ln b &= \int_b^a \frac{dx}{x} \\
    &\geq (a - b) \frac{1}{\frac{a + b}{2}} \\
    &= \frac{2(a - b)}{a + b}.
\end{align}
が言える。
TODO: 積分の図を書く。



\subsection{$F(0)$の表式}
経路にわたっての期待値を計算するのだが、一旦時間を離散化したほうがわかりやすいので、離散化してマルコフ連鎖として考える。このとき、経路$\Gamma$の実現確率は
\begin{align}
    P_\theta(\Gamma) = \prod_{n=1}^N T^\theta_{w^n w^{n-1}} p^0_{w^0}
\end{align}
と書かれる。このもとで、m-step目における確率分布は、
\begin{align}
    p^m_{w_m} = \sum_{w_0, \dots, w_{m-1}} \prod_{n=1}^m T^{w^n w^{n-1}} p^0_{w^0}
\end{align}
と書かれる。これを下に、Fisher情報量を計算する。
\begin{align}
    \int d\Gamma \, P(\Gamma) \frac{\partial^2}{\partial \theta^2} \ln P_\theta(\Gamma) 
    &= \sum_{w_0, \dots, w_N} \prod_{n=1}^N T^{w^n w^{n-1}} p^0_{w^0} 
    \frac{\partial^2}{\partial \theta^2} 
    \left(
        \sum_{n=1}^N \ln T^\theta_{w^n w^{n-1}} + \ln p^0_{w^0}
    \right) \\
    &= \sum_{w_0, \dots, w_N} \prod_{n=1}^N T^{w^n w^{n-1}} p^0_{w^0} 
    \frac{\partial^2}{\partial \theta^2} 
    \left(
        \sum_{n=1}^N \ln T^\theta_{w^n w^{n-1}}
    \right) \\
\end{align}
一番最後の$\Sigma$について考える。和の$k$番目の項について、
\begin{align}
    (\text{第k項}) &= \sum_{w_0, \dots, w_N} \prod_{n=1}^N T^{w^n w^{n-1}} p^0_{w^0} \frac{\partial^2}{\partial \theta^2} \ln T^\theta_{w^k w^{k-1}} \\
    &(w^{0}\text{から}w^{k-2}\text{まで計算して、}) \nonumber \\
    &= \sum_{w_{k-1}, w_k,\cdots, w_N} \prod_{n=k-1}^N T^{w^n w^{n-1}} p^{k-1}_{w^{k-1}}\frac{\partial^2}{\partial \theta^2} \ln T^\theta_{w^k w^{k-1}} \\
    &(w^{k+1}\text{から}w^{N}\text{まで計算すると、規格化条件より}) \nonumber \\
    &= \sum_{w_{k-1}, w_k} T^{w^k w^{k-1}} p^{k-1}_{w^{k-1}}\frac{\partial^2}{\partial \theta^2} \ln T^\theta_{w^k w^{k-1}} 
\end{align}
となる。したがって、
\begin{align}
    F(\theta) = -\sum_{n=1}^N \sum_{w_{n-1}, w_n} T^{w^n w^{n-1}} p^{n-1}_{w^{n-1}} \pdv[2]{\theta} \ln T^\theta_{w^n w^{n-1}}
\end{align}
となる。これを連続極限に持っていくと、$w^n = w^{n-1}$の場合と$w^n \neq w^{n-1}$の場合に分けて、
\begin{align}
    F(\theta) = -\int \dd{t} \qty[\sum_{i \neq j} R_{ij} p_j(t) \pdv[2]{\theta} \ln R_{ij}^\theta + \sum_j p_j(t) \pdv[2]{\theta} \ln e^{R_{jj}^\theta}]
\end{align}
となる。ただし、第二項については、$T_{ii} = 1-\sum_{j(\neq i)} R_{ij}\Delta t =1 + R_{ii} \Delta t$であることを用いて、
\begin{align}
    (1+R_{ii}\Delta t )\ln (1+R_{ii}\Delta t) &= (1+R_{ii}\Delta t )\ln \exp(R_{ii}\Delta t) + O(\Delta t^2)\\
    &= (1+R_{ii}\Delta t )R_{ii} \Delta t + O(\Delta t^2)\\
    &= R_{ii} \Delta t + O(\Delta t^2)\\
    &= \ln e^{R_{ii}\Delta t} + O(\Delta t^2)
\end{align}
となることを用いている。\\

\begin{thebibliography}{99}
    \bibitem{ref1} Naoto Shiraishi, 『An introduction to stochastic thermodynamics』, Springer, 2023
    \bibitem{ref2} 齊藤圭司, 『ゆらぐ系の熱力学』, SGCライブラリ, 2022
  \end{thebibliography}
  

\end{document}