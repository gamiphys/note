\documentclass[a4paper,11pt]{jsarticle}

% 数式
\usepackage{amsmath,amsfonts}
\usepackage{amsthm}
\usepackage{bm}
\usepackage{mathtools}
\usepackage{amssymb}

% 表
\usepackage[utf8]{inputenc}
\usepackage{diagbox} % 斜線付きセルを作成するために必要
\usepackage{booktabs} % 表の罫線を美しくするために必要
\usepackage{hhline} % 水平罫線を制御するために必要

% 画像
\usepackage[dvipdfmx]{graphicx}
\usepackage{ascmac}
\usepackage{physics}
\usepackage{float} % 追加

% 図
\usepackage[dvipdfmx]{graphicx}
\usepackage{tikz} %図を描く
\usetikzlibrary{positioning, intersections, calc, arrows.meta,math} %tikzのlibrary

% ハイパーリンク
\usepackage[dvipdfm,
  colorlinks=false,
  bookmarks=true,
  bookmarksnumbered=false,
  pdfborder={0 0 0},
  bookmarkstype=toc]{hyperref}

% 式番号を章ごとにリセット
\numberwithin{equation}{section}

\begin{document}

\title{物理量(非平衡)}
\author{大上由人}
\date{\today}
\maketitle

\section{物理量の分類}
ゆらぐ系の熱力学においては、基本的に以下の三種類の物理量が考えられる。\\
\begin{itembox}[l]{\textbf{Def.state quantity/jump quantity/path quantity}}
    state quantity $\hat{f}$は、状態 $j$に対して、値 $f_j$をとる物理量である。また、この期待値は、
    \begin{align}
        \ev{\hat{f}}_{\vb{p}} = \sum_{j} f_j p_j
    \end{align}
    で定義される。\\

    jump quantity $\hat{g}$は、状態 $j$から状態 $k$に遷移するとき、値 $g_{j\to k}$をとる物理量である。また、この期待値は、
    \begin{align}
        \ev{\hat{g}}_{\vb{p},R} = \sum_{j,k} R_{kj} p_j g_{j\to k}
    \end{align}
    で定義される。\\

    path quantity $\hat{\hat{F}}$は、経路$\Gamma$に対して、値 $F({\Gamma})$をとる物理量である。また、この期待値は、
    \begin{align}
        \ev{\hat{\hat{F}}}_{\Gamma} = \int \dd{\Gamma} P(\Gamma) F(\Gamma)
    \end{align}
    で定義される。
\end{itembox}

\section{物理量の関係}
\subsection{state quantityとpath quantityの関係}
時間に依存するstate quantity $\hat{f}(t)$について、対応するpath quantityを考えることができる。対応するpath quantity $\hat{\hat{f}}(t)$は、経路$\Gamma$に対して、値
\begin{align}
    f(\Gamma,t) = f_{\Gamma(t)} = \sum_{m=0}^{n} f_{j_{m}}(t) \chi[t \in [t_{m},t_{m+1}]]
\end{align}
をとる物理量である。このとき、以下が成り立つ。
\begin{itembox}[l]{\textbf{Prop:state quantityとpath quantityの関係}}
    state quantity $\hat{f}(t)$と対応するpath quantity $\hat{\hat{f}}(t)$について、以下が成り立つ。
    \begin{align}
        \ev{\hat{f}(t)}_{\vb{p}(t)} = \ev{\hat{\hat{f}}(t)}_{\Gamma}
    \end{align}
\end{itembox}
\textbf{Prf.}\\



\end{document}