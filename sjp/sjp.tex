\documentclass[a4paper,11pt]{jsarticle}

% 数式
\usepackage{amsmath,amsfonts}
\usepackage{amsthm}
\usepackage{bm}
\usepackage{mathtools}
\usepackage{amssymb}

% 表
\usepackage[utf8]{inputenc}
\usepackage{diagbox} % 斜線付きセルを作成するために必要
\usepackage{booktabs} % 表の罫線を美しくするために必要
\usepackage{hhline} % 水平罫線を制御するために必要

% 画像
\usepackage[dvipdfmx]{graphicx}
\usepackage{ascmac}
\usepackage{physics}
\usepackage{float} % 追加

% 図
\usepackage[dvipdfmx]{graphicx}
\usepackage{tikz} %図を描く
\usetikzlibrary{positioning, intersections, calc, arrows.meta,math} %tikzのlibrary

% ハイパーリンク
\usepackage[dvipdfm,
  colorlinks=false,
  bookmarks=true,
  bookmarksnumbered=false,
  pdfborder={0 0 0},
  bookmarkstype=toc]{hyperref}

% 式番号を章ごとにリセット
\numberwithin{equation}{section}

\begin{document}

\title{物理量(状態量、遷移量、経路量)の関係}
\author{大上由人}
\date{\today}
\maketitle

\section{物理量の分類}
ゆらぐ系の熱力学においては、基本的に以下の三種類の物理量が考えられる。\\
\begin{itembox}[l]{\textbf{Def.state quantity/jump quantity/path quantity}}
    state quantity $\hat{f}$は、状態 $j$に対して、値 $f_j$をとる物理量である。また、この期待値は、
    \begin{align}
        \ev{\hat{f}}_{\vb{p}} = \sum_{j} f_j p_j
    \end{align}
    で定義される。\\

    jump quantity $\hat{g}$は、状態 $j$から状態 $k$に遷移するとき、値 $g_{j\to k}$をとる物理量である。また、この期待値は、
    \begin{align}
        \ev{\hat{g}}_{\vb{p},R} = \sum_{j,k} R_{kj} p_j g_{j\to k}
    \end{align}
    で定義される。\\

    path quantity $\hat{\hat{F}}$は、経路$\Gamma$に対して、値 $F({\Gamma})$をとる物理量である。また、この期待値は、
    \begin{align}
        \ev{\ev{\hat{\hat{F}}}}_{\Gamma} = \int \dd{\Gamma} P(\Gamma) F(\Gamma)
    \end{align}
    で定義される。
\end{itembox}

\section{物理量の関係}
\subsection{state quantityとpath quantityの関係}
時間に依存するstate quantity $\hat{f}(t)$について、対応するpath quantityを考えることができる。対応するpath quantity $\hat{\hat{f}}(t)$は、経路$\Gamma$に対して、値
\begin{align}
    f(\Gamma,t) = f_{\Gamma(t)} = \sum_{m=0}^{n} f_{j_{m}}(t) \chi[t \in [t_{m},t_{m+1}]]
\end{align}
をとる物理量である。このとき、以下が成り立つ。
\begin{itembox}[l]{\textbf{Prop.state quantityとpath quantityの関係}}
    state quantity $\hat{f}(t)$と対応するpath quantity $\hat{\hat{f}}(t)$について、以下が成り立つ。
    \begin{align}
        \ev{\hat{f}(t)}_{\vb{p}(t)} = \ev{\ev{\hat{\hat{f}}(t)}_{\Gamma}}
    \end{align}
\end{itembox}
\textbf{Prf.}\\

\subsection{jump quantityとpath quantityの関係}
時間に依存するjump quantity $\hat{g}(t)$について、対応するpath quantityを考えることができる。対応するpath quantity $\hat{\hat{g}}(t)$は、経路$\Gamma$に対して、値
\begin{align}
    g(\Gamma,t) = \sum_{m=1}^{n} g_{j_{m-1} \to j_{m}}(t) \delta(t-t_{m})
\end{align}
をとる物理量である。このとき、以下が成り立つ。
\begin{itembox}[l]{\textbf{Prop.jump quantityとpath quantityの関係}}
    jump quantity $\hat{g}(t)$と対応するpath quantity $\hat{\hat{g}}(t)$について、以下が成り立つ。
    \begin{align}
        \ev{\hat{g}(t)}_{\vb{p}(t),R(t)} = \ev{\ev{\hat{\hat{g}}(t)}}_{\Gamma}
    \end{align}
\end{itembox}
\textbf{Prf.}\\
離散のほうがやりやすいので、離散の場合を考える。右辺は以下のように書ける。
\begin{align}
    \int d\Gamma \, P(\Gamma) \sum_{m=1}^{n} g_{j_{m-1} \to j_{m}}(t) \delta_{t,t_m}
    &= \sum_{j_0, \dots, j_N} \prod_{n=1}^N T^{j^n j^{n-1}} p^0_{j^0} 
    \sum_{m=1}^{n} g_{j_{m-1} \to j_{m}}(t_m) 
\end{align}
一番最後の$\Sigma$について考える。和の$k$番目の項について、
\begin{align}
    (\text{第k項}) &= \sum_{j_0, \dots, j_N} \prod_{n=1}^N T^{j^n j^{n-1}} p^0_{j^0} g_{j_{m-1} \to j_{m}}(t_m) \\
    &(j^{0}\text{から}j^{k-2}\text{まで計算して、}) \nonumber \\
    &= \sum_{j_{k-1}, j_k, \cdots, j_N} \prod_{n=k-1}^N T^{j^n j^{n-1}} p^{k-1}_{j^{k-1}} g_{j_{m-1} \to j_{m}}(t_m) \\
    &(j^{k+1}\text{から}j^{N}\text{まで計算すると、規格化条件より}) \nonumber \\
    &= \sum_{j_{k-1}, j_k} T^{j^k j^{k-1}} p^{k-1}_{j^{k-1}} g_{j_{m-1} \to j_{m}}(t_m)
\end{align}
となる。これを連続極限に持っていくことで、
\begin{align}
    \int d\Gamma \, P(\Gamma) \sum_{m=1}^{n} g_{j_{m-1} \to j_{m}}(t) \delta_{t,t_m}
    &= \sum_{j\neq k} p_{j} R_{kj} g_{j \to k}(t)\\
    &= \ev{\hat{g}(t)}_{\vb{p}(t), R(t)}
\end{align}
となる。\qed

また、path quantityの積分量
\begin{align}
    {\hat{\hat{G}}}= \int_{0}^{\tau} \dd{t} \hat{\hat{g}}(t)
\end{align}
は、pathに対して値
\begin{align}
    G(\Gamma) = \int_{0}^{\tau} \dd{t} g(\Gamma,t) = \sum_{m=1}^{n} g_{j_{m-1} \to j_{m}}(t_{m})
\end{align}
をとる物理量である。このとき、以下が成り立つ。
\begin{align}
    \ev{\ev{\hat{\hat{G}}}}_{\Gamma} = \int \dd{t} \ev{\ev{\hat{\hat{g}}(t)}}_{\Gamma} = \int \dd{t} \ev{\hat{g}(t)}_{\vb{p}(t),R(t)}
\end{align}




\end{document}