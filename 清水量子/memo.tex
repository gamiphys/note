\documentclass[a4paper,11pt]{jsarticle}


% 数式
\usepackage{amsmath,amsfonts}
\usepackage{bm}
\usepackage{mathtools}

%表
\usepackage[utf8]{inputenc}
\usepackage{diagbox} % 斜線付きセルを作成するために必要
\usepackage{booktabs} % 表の罫線を美しくするために必要
\usepackage{hhline} % 水平罫線を制御するために必要

% 画像
\usepackage[dvipdfmx]{graphicx}
\usepackage{ascmac}
\usepackage{physics}
\usepackage{float} % 追加

\begin{document}
\section*{3章}
共有結合結晶や、そのsp3混成軌道、また、強束縛近似によるハミルトニアンの対角化、バンド図が主な話題となる。\\
\subsection*{sp3混成軌道}
sp3混成軌道について話せそうなこと挙げると、まず結晶の配置からp軌道の向きを考え、波動関数の重ね合わせを考える。\\
波動関数の重ね合わせからsp3のエネルギーの概略を説明することになる。\\
\subsection*{強束縛近似}
強束縛近似の概要をまず述べる(一次元鎖について考えればよさそう)。すなわち、ハミルトニアンのうち、どの項を捨て、どの項を用いるかを説明する。\\
その後、対角化によりバンドが得られることを確認する。\\
ダイヤモンドのほうを対角化するのはダイジェストでいいと思う。\\
\section*{4章}
自由電子近似による金属結合の仕組みや、パウリ排他律の効き方について話す。
\subsection*{自由電子近似}
自由電子近似の妥当性をまず話してから、近似の概要を喋る。
\subsection*{相関係数周りの話}
話せたら話したいなあ
\end{document}