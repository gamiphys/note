\documentclass[a4paper,11pt]{jsarticle}


% 数式
\usepackage{amsmath,amsfonts}
\usepackage{bm}
\usepackage{mathtools}

%表
\usepackage[utf8]{inputenc}
\usepackage{diagbox} % 斜線付きセルを作成するために必要
\usepackage{booktabs} % 表の罫線を美しくするために必要
\usepackage{hhline} % 水平罫線を制御するために必要

% 画像
\usepackage[dvipdfmx]{graphicx}
\usepackage{ascmac}
\usepackage{physics}
\usepackage{float} % 追加

% 図
\usepackage[dvipdfmx]{graphicx}
\usepackage{tikz} %図を描く
\usetikzlibrary{positioning, intersections, calc, arrows.meta,math} %tikzのlibrary

\begin{document}

\title{一次元束縛状態の性質}
\author{大上由人}
\date{\today}
\maketitle
\section{一次元束縛状態には縮退が存在しない}
一次元束縛状態には縮退が存在しないことを示す。\\
同じ固有値に属する、線形独立な二つの固有関数$\psi_1(x)$と$\psi_2(x)$が存在するとする。\\
このとき、ロンスキー行列式は
\begin{align}
    W(x) = \begin{vmatrix}
        \psi_1(x) & \psi_2(x) \\
        \psi_1'(x) & \psi_2'(x)
    \end{vmatrix}
\end{align}
である。\\
このとき、この一回微分を計算すると、
\begin{align}
    W'(x) &= \psi_1(x)\psi_2''(x) - \psi_1''(x)\psi_2(x) \\
    &= \psi_1(x)\frac{2m}{\hbar^2}(V(x) - E)\psi_2(x) - \frac{2m}{\hbar^2}(V(x) - E)\psi_1(x)\psi_2(x) \\
    &= 0
\end{align}
となる。ただし、ここで固有関数が縮退していることを用いていることに注意せよ。\\
したがって、$W(x)$は一定である。\\
また、$W(x)$は束縛条件を満たすため、$W(x) = 0$である。\\
ゆえに、
\begin{align}
    \psi_1(x)\psi_2'(x) = \psi_1'(x)\psi_2(x)
\end{align}
となる。\\
この両辺を積分すると、
\begin{align}
    \int \frac{\psi_1'(x)}{\psi_1(x)}dx = \int \frac{\psi_2'(x)}{\psi_2(x)}dx
\end{align}
となる。\\
したがって、
\begin{align}
    \log|\psi_1(x)| = \log|\psi_2(x)| + C
\end{align}
となる。\\
したがって、
\begin{align}
    \psi_1(x) = \pm e^C\psi_2(x)
\end{align}
となり、$\psi_1(x)$と$\psi_2(x)$は比例関係にあることがわかる。\\
したがって、$\psi_1(x)$と$\psi_2(x)$は線形独立であることに矛盾する。\\
したがって、一次元束縛状態には縮退が存在しないことが示された。

\end{document}