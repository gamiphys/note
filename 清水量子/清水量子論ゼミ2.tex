\documentclass[a4paper,11pt]{jsarticle}


% 数式
\usepackage{amsmath,amsfonts}
\usepackage{bm}
\usepackage{mathtools}

%表
\usepackage[utf8]{inputenc}
\usepackage{diagbox} % 斜線付きセルを作成するために必要
\usepackage{booktabs} % 表の罫線を美しくするために必要
\usepackage{hhline} % 水平罫線を制御するために必要

% 画像
\usepackage[dvipdfmx]{graphicx}
\usepackage{ascmac}
\usepackage{physics}
\usepackage{float} % 追加

\begin{document}

\title{清水量子論ゼミ二章\footnote{文字をきれいに書くのが苦手なので\LaTeX を用いることにした。}}
\author{大上由人}
\date{\today}
\maketitle

\section{基本的枠組み}

\subsection{古典論の基本的枠組み}
古典論における仮定は以下の4つにまとめられる。

\begin{itembox}[l]{\textbf{古典論の基本仮定と枠組み}}
  \begin{enumerate}
 \item すべての物理量は、どの瞬間にも、各々一つずつ定まった値を持っている。\\
 \item  測定とは、その時刻における物理量の値を知ることである。\\
 \item ある時刻における物理状態とは、その時刻における物理量の値の一覧表(例えば、相空間上の点(q,p)を定めるようなことを考えると良い)のことである。\\
 \item 時間発展とは、物理量の値が時々刻々と変化することである。
  \end{enumerate}
\end{itembox}
以上の仮定の特徴は、"測定に関係なく"物理量は各瞬間で定まった値を持っているということである。\\
仮定をもとに、"初期時刻$t_0$における物理量の値が与えられると、後の時刻tを計算できる"ことを目標に定式化された。(微分方程式を用いた。)\\
\textbf{例}\\
一次元調和振動子の運動を考える。このときハミルトニアンは\\
\begin{align}
H(q,p)=\frac{p^2}{2m}+\frac{1}{2}m\omega ^2 q^2
\end{align}
となる。\\
ある時刻における物理状態は、位相空間上の点$(q,p)$により指定される(枠組み3)。
このとき、時間発展は正準方程式
\begin{align}
\dv{q}{t}=\pdv{H(q,p)}{p}\\
\dv{p}{t}=-\pdv{H(q,p)}{q}
\end{align}
により記述される。(枠組み4)\\
時刻tに、例えば位置を測定すると、その時刻のq(t)を知ることができる。(枠組み2)
\bigskip

\subsection{量子論の基本的枠組み}
一つの物理状態に、$\psi$という名前を付ける。量子論において、物理量$A$の測定値$a$は同じ$\psi$に対しても、ばらつく。\footnote{ミクロスケールになると、観測行為(例えば、粒子の位置を観測するために光を照射するようなことを考えると良い。)によって注目している系を乱してしまう。ミクロな系をしたがって、ミクロ系を扱う量子力学においては、完全に物理量を一意に定めることができない。しかし、確率分布までなら一意に定まることが実験的にわかっている。たとえば、教科書の1.2節や、シュテルンゲルラッハの実験では、スピン上向きと下向きがそれぞれ確率$\frac{1}{2}$であることが示されている。}しかし、"同じ物理状態"を用意して、繰り返し測定すると、その確率分布$\{P(a)\}$は一意に定まる。\\

\textbf{例}シュテルンゲルラッハの実験\\
スピン\footnote{スピンとは、粒子の持つ固有の角運動量のことで、古典的な開店などでは理解できない内部自由度である。}の向きがランダムな原子が炉に入っていて、炉の小さな穴から原子が放出されることを考える。放出された電子は、z方向の磁場から力を受ける。このときの力の大きさと向きは、各電子のスピンの向きに依存する。
このとき、スピンの向きは本来ランダムなはず(炉の中でランダムだから)で、スクリーンには連続的な分布が見られそうなものだが、実際にはスクリーンの二か所に電子が局在する。これは、スピン上向きと下向きがそれぞれ確率$\frac{1}{2}$であることを示している。\\

このとき、別の状態を測れば確率分布は変わるし、同じ状態でも物理量が違ったら、確率分布は変わる。よって、$\{P(a)\}$は$\psi$と$A$の関数である。\\

ここで、量子論における"同じ状態"を定義する。\\

\begin{itembox}[l]{\textbf {def 同じ状態/異なる状態}}
2つの状態$\psi ,\psi '$について、どんな物理量の測定値の確率分布も一致すれば、$\psi ,\psi '$は同じ状態である。\\
また、測定値の確率分布が異なるような物理量が一つでもあれば、$\psi$ と$\psi '$は異なる状態である。\\
\end{itembox}
例えば、$\psi$と$\psi '$で位置と運動量の測定をしたとき、$\{P_x(\psi)\}=\{P_x(\psi ')\}$かつ$\{P_p(\psi)\}=\{P_p(\psi ')\}$であるならば、二つの状態は同じだが、$\{P_x(\psi)\}=\{P_x(\psi ')\}$かつ$\{P_p(\psi)\}\neq \{P_p(\psi ')\}$であるならば、二つの状態は異なる。\\
また、実際に測定可能な量(物理量)のことを、可観測量と呼ぶ。\\
以上を踏まえて、量子論の基本的な枠組みは以下のようになる。\\

\begin{itembox}[l]{\textbf{量子論の基本的枠組み}}
  \begin{enumerate}
    \item すべての物理量が各瞬間に定まった値を持つことが一般にはない。すなわち、各々の物理量は、ひとつの数値をとるような変数ではない。
    \item 物理量Aの測定とは、観測者が測定値を一つ得る行為である。得られた測定値aの値は、同じ物理的状態について測定しても、一般には測定の度にばらつく。しかし、確率分布$\{P(a)\}$一意に定まる。
\item 物理的状態とは、各Aに対してそれを測定したときの確率分布$\{P(a)\}$を与えるものであり、物理量ではない別のもの$\psi$で表す。すなわち$\psi$は
\begin{align}
\psi : A \mapsto \{P(a)\}
\end{align}
なる写像である。\\
また、物理的状態の違いとは、この写像の違いである。
\item 系が時間発展するとは、測定を行った時刻によって、異なる$\{P(a)\}$が得られるということである。(このとき、この時間発展の変化を、$\psi$によるものとしても、$A$によるものとしても、その両方としても良い。\footnote{それぞれシュレディンガー描像、ハイゼンベルク描像と呼ばれる。最後のものは特に名前はないし、何に使えるのか僕は知らない。})
  \end{enumerate}
\end{itembox}

\textbf{例}\\
明日元気があれば作る。

\newpage

\subsection{自由度}
物理においては、基本変数を用いて理論を構成する。
ここで、基本変数とは、「系のすべての物理量を構成できるような変数」のことである。(例えば、古典力学においては、(q,p)を用いてハミルトニアンを書き下し、他の物理量を表す。\footnote{一般に、量子力学において、基本変数は可観測量とは限らない。たとえば、調和振動子の運動を昇降演算子を用いて表すことができるが、昇降演算子には$q$と$p$が含まれていて、量子論の枠組み(1)より、昇降演算子は測定不可能な量である
。})\\
必要な基本変数の組の数を自由度と呼ぶ。(例えば、三次元自由粒子の運動は、$(q_i,p_i)(i=1,2,3)$の三組によって記述され、自由度は3)\\
有限の基本変数で書ける系を有限自由系といい、そうでない形を無限自由系と呼ぶ。\\

\subsection{閉じた系}
考える系が、他の系と一切相互作用しない場合、その系を閉じた系と呼び、そうでない系を開いた系と呼ぶ。\footnote{今回のゼミでは閉じた系しか扱わないと思う。}\\
\subsection{純粋状態/混合状態}
純粋状態とは、原理的に許される最大限のところまで状態を指定しつくした状態のことで、そうでない状態のことを混合状態と呼ぶ。\\
これを定式化すると、$\psi _j$により混合された量子状態は
\begin{align}
P_{\psi}(a)=\Sigma \lambda _j P_{\psi _j}(a)\quad (0<\lambda _j <1,\Sigma_j \lambda _j=1)
\end{align}
と書くことができ、とくに純粋状態では(混合状態でいうところj番目の状態のみの純粋状態は)
\begin{align}
P_{\psi}(a)= P_{\psi _j}
\end{align}
となり、$\lambda_j =1$に対応する。\\

\noindent{\textbf{純粋状態と混合状態の例}}\\
1電子についてのスピンについて考える。\\
いつか気が向いたら書く。
\subsection{純粋状態/混合状態}
上の例からわかるように、混合状態の純粋状態への分解は一意的でない。

\end{document}