\documentclass[a4paper,11pt]{jsarticle}


% 数式
\usepackage{amsmath,amsfonts}
\usepackage{bm}
\usepackage{mathtools}

%表
\usepackage[utf8]{inputenc}
\usepackage{diagbox} % 斜線付きセルを作成するために必要
\usepackage{booktabs} % 表の罫線を美しくするために必要
\usepackage{hhline} % 水平罫線を制御するために必要

% 画像
\usepackage[dvipdfmx]{graphicx}
\usepackage{ascmac}
\usepackage{physics}
\usepackage{float} % 追加

% 図
\usepackage[dvipdfmx]{graphicx}
\usepackage{tikz} %図を描く
\usetikzlibrary{positioning, intersections, calc, arrows.meta,math} %tikzのlibrary

\begin{document}

\title{純粋状態と混合状態}
\author{大上由人}
\date{\today}
\maketitle
\section[1]{密度演算子}
これまで、純粋系の状態は、状態ケット$\ket*{\psi}$によりすべて記述されるとしてきた。
以下、混合状態も含めてすべて記述する量として、密度演算子を導入する。\\

\begin{itembox}[l]{\textbf {密度演算子}}
   混合状態も含めて、系の状態は、密度演算子
    \begin{align}
        \hat{\rho} = \sum_{i} w_i\ketbra*{\psi_i}
    \end{align}
    により記述される。ただし、$w_i$は、状態iが選ばれる古典的確率(すなわち、状態の重ね合わせでないような確率\footnote{くじ引きとか、さいころとかで決めてもいい})である。
    \end{itembox}
イメージとしては、くじ引きにより、確率$w_i$で状態iというものが選ばれる、といった様子となる。\\

\begin{itembox}[l]{\textbf{純粋状態/混合状態}}
    密度演算子の、iが一つだけに限られる状態を純粋状態といい、そうでない系を混合状態という。
\end{itembox}
\textbf{注意}\\
このとき、$\ket*{\psi}$は重ね合わせの状態にあっても良い。\\

状態ケットでないのに(演算子なのに)状態を表すとは何事かと思うかもしれないが、我々の量子論でやりたいことは、測定値の確率分布を作ることであった。
以下、確率分布が確かに定まることを示す。

\textbf{純粋状態について}\\
純粋状態の密度演算子は、
\begin{align}
    \hat{\rho} = \ketbra{\psi}
\end{align}
により記述される。このとき、両辺を固有状態$\ket*{a}$を用いて変形すると\\
\begin{align}
    \bra{a}\hat{\rho}\ket*{a} =\braket{a}{\psi}\braket{\psi}{a}
    =|\braket{a}{\psi}|^2
    =\mathrm{P_{\psi}(a)}
\end{align}

となり、見慣れたボルンの確率振幅となる。\\

\textbf{混合状態について}\\
混合状態の密度演算子は、
\begin{align}
    \hat{\rho} = \sum_{i} w_i\ketbra*{\psi_i}
\end{align}
により記述される。このとき、両辺を固有状態$\ket*{a}$を用いて変形すると\\
\begin{align}
    \bra{a}\hat{\rho}\ket*{a}&
    =\sum_i w_i |\braket{a}{\psi_i}|^2
    =\sum_i w_i \mathrm{P_{\psi_i}(a)}
\end{align}
となる。\\

\section[2]{清水量子論との整合性}
清水量子論における、純粋状態および混合状態の定義を再掲する。\\
\begin{itembox}[l]{\textbf{純粋状態と混合状態(清水)}}
純粋状態とは、原理的に許される最大限のところまで状態を指定しつくした状態のことで、そうでない状態のことを混合状態と呼ぶ。\\
これを定式化すると、$\psi _i$により混合された量子状態は
\begin{align}
P_{\psi}(a)=\Sigma w _i P_{\psi _i}(a)\quad (0<w_i <1,\sum w_i=1)
\end{align}
と書くことができ、とくに純粋状態では(混合状態でいうところi番目の状態のみの純粋状態は)
\begin{align}
P_{\psi}(a)= P_{\psi _i}
\end{align}
となり、$w_i =1$に対応する。\\
\end{itembox}
これと、section1でつくった例を見てみると、密度演算子を用いた純粋状態/混合状態の定義と一致する。\\

\section[3]{具体例(思考実験)}
有名な量子論の思考実験として、シュレディンガーの猫というものがある。これを用いて混合状態を説明することを試みる。\\
混合状態は以下のようである:\\
いま、目の前にボタンがある。ボタンを押すと、三つの箱A,B,Cのうち一つが$\frac{1}{3}$で選ばれる。($w_A=w_B=w_C=\frac{1}{3}$)
箱が選ばれたあとは、いつものシュレディンガーの猫と同じで、猫の生死が重ね合わせの状態にある。\\
このときの、ボタンを押す前の状態が混合状態である。

\end{document}