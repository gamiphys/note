\documentclass[a4paper,11pt]{jsarticle}


% 数式
\usepackage{amsmath,amsfonts}
\usepackage{bm}
\usepackage{mathtools}

%表
\usepackage[utf8]{inputenc}
\usepackage{diagbox} % 斜線付きセルを作成するために必要
\usepackage{booktabs} % 表の罫線を美しくするために必要
\usepackage{hhline} % 水平罫線を制御するために必要

% 画像
\usepackage[dvipdfmx]{graphicx}
\usepackage{ascmac}
\usepackage{physics}
\usepackage{float} % 追加

\begin{document}

\title{清水量子論ゼミ三章\footnote{文字をきれいに書くのが苦手なので\LaTeX を用いることにしたが、タイピングも普通に面倒なのでかなり省略しながら書いていく}}
\author{大上由人}
\date{\today}
\maketitle

\section{閉じた有限自由度系の純粋状態の量子論}


\subsection{基本的な考え方}
量子論を組み立てるうえで、$\{P(a)\}$を具体的に求めることを目標として構成していく。というのも、量子力学において予言できるのは$\{P(a)\}$のみだからだ。\\
演算子形式の量子論では、「複素ヒルベルト空間」と呼ばれる空間を考え、「状態」や「物理量」を、複素ヒルベルト空間の元や、演算子(ベクトルをほかのベクトルに移す写像\footnote{一般には、関数を他の関数に変換する作用素を演算子という})に対応させる。\\
そのようにすれば、確率分布$\{P(a)\}$を計算するように定式化できる。

\subsection{複素ヒルベルト空間}
まず、複素ベクトル空間の公理を確認する。\\
\textbf{複素ベクトル空間の公理:}

\begin{enumerate}
    \item \textbf{加法:}
    \begin{itemize}
        \item 任意の \( u, v \) について、\( u + v \) も複素ベクトルである。
        \item \( u + v = v + u \) (可換法則)
        \item \( (u + v) + w = u + (v + w) \) (結合法則)
        \item \( u + 0 = u \) (零ベクトルの存在)
        \item 任意の \( u \) に対して、\( u + (-u) = 0 \) (逆ベクトルの存在)
    \end{itemize}
    
    \item \textbf{スカラー倍:}
    \begin{itemize}
        \item 任意の \( \alpha \) と \( u \) について、\( \alpha u \) も複素ベクトルである。
        \item \( 1u = u \) (単位スカラーの存在)
        \item \( \alpha(\beta u) = (\alpha \beta) u \) (スカラーの結合法則)
        \item \( (\alpha + \beta)u = \alpha u + \beta u \) (分配法則)
        \item \( \alpha(u + v) = \alpha u + \alpha v \) (分配法則)
    \end{itemize}
\end{enumerate}
であった。この複素ベクトル空間に内積を備えた、内積空間を考える。
\begin{itembox}[l]{複素ヒルベルト空間}
\textbf{$\vb C$}上の完備な内積空間$\mathcal{H}$を、複素ヒルベルト空間という。\footnote
{
内積空間$\mathcal{H}$が完備であるとは、$\mathcal{H}$内のどんなベクトル列$\ket{\psi_1},\ket{\psi_2}\cdots$についても、もしそれが
\begin{align}
\lim_{n,m \to \infty} \|\ket{\psi_n}-\ket{\psi_m}\|=0
\end{align}
を満たすベクトル列であれば、必ずその収束先
\begin{align}
\lim_{n,m \to \infty} \|\ket{\psi_n}-\ket{\psi}\|=0
\end{align}
が$\mathcal H$内に存在することである。
}
\end{itembox}
\[
\begin{bmatrix}
x_{1} \\
x_{2} \\
\vdots \\
x_{n}
\end{bmatrix}
\]
\noindent{\textbf{僕が勝手にしていた勘違いとその訂正}}\\
なぜかヒルベルト空間自体に無限次元が要請されていると思っていたが、そうではない。\\

\noindent{\textbf{複素ヒルベルト空間の元に関する注意}}\\
一般に、複素ヒルベルト空間は、縦ベクトルの集合に限るわけではなく、同じ内積を与える範囲なら、任意のヒルベルト空間を用いてよい。\\
\textbf{ex)}
適当な$\ket{A}$と$\ket{B}$を定めて、
\begin{align}
\braket{A}{B}=\delta_{AB}
\end{align}
としても(行列で定義しなくても)、行列で行うのと同じような演算:
\begin{align}
\braket{\psi_1}{\psi_2}&=\braket{aA+bB}{aA+bB}\\
&=a^{*}a\braket{A}{A}+a^{*}b\braket{A}{B}+ab^{*}\braket{B}{A}+ab^{*}\braket{B}{B}
=a^{*}a+b^{*}b
\end{align}
ができる。
\bigskip

\subsection{量子状態}
\textbf{射線}\\
$\ket{\psi}$と同じ長さを持ち、平行なケットの集合は
\begin{align}
\{e^{i\theta}\ket{\psi}|\theta \in\vb{R}\}
\end{align}
となり、射線と呼ぶ。このとき、
\begin{itembox}[l]{\textbf {要請1}}
量子系の純粋状態は、ある複素ヒルベルト空間$\mathcal H$の、規格化された射線で表される。
\end{itembox}
ここで、一般には異なる量子系は異なるヒルベルト空間で表す必要があるが、たまたま同じ複素ヒルベルト空間を使える場合もある。
\newpage

\subsection{演算子とその固有値・固有ベクトル}
ベクトルを別のベクトルに移す写像を考える。その写像を$\hat{A}$と書き、$\ket{\psi}$が$\hat{A}$で移った先を$\hat{A}\ket{\psi}$とする。

\begin{itembox}[l]{\textbf{数学的定義}}
  \begin{enumerate}
    \item すべての物理量が各瞬間に定まった値を持つことが一般にはない。すなわち、各々の物理量は、ひとつの数値をとるような変数ではない。
    \item 物理量Aの測定とは、観測者が測定値を一つ得る行為である。得られた測定値aの値は、同じ物理的状態について測定しても、一般には測定の度にばらつく。しかし、確率分布$\{P(a)\}$は$A$と$\psi$辛い地位に定まる。
\item 物理的状態とは、各Aに対してそれを測定したときの確率分布$\{P(a)\}$を与えるものであり、物理量ではない別のもの$\psi$で表す。すなわち$\psi$は
\begin{align}
\psi : A \mapsto \{P(a)\}
\end{align}
なる写像である。\\
また、物理的状態の違いとは、この写像の違いである。
\item 系が時間発展するとは、測定を行った時刻によって、異なる$\{P(a)\}$が得られるということである。(このとき、この時間発展の変化を、$\phi$によるものとしても、$A$によるものとしても、その両方としても良い。\footnote{それぞれシュレディンガー描像、ハイゼンベルク描像と呼ばれる。})
  \end{enumerate}
\end{itembox}

\begin{itembox}[l]{\textbf {要請2}}
可観測量は、$\mathcal H$上の自己共役演算子によって表される。\footnote{ただし、任意の自己共役演算子が物理量になるわけではない。}
\end{itembox}
\textbf{例}

\subsection{自己共役演算子の固有値}
\begin{itembox}[l]{\textbf {定理3.1}}
自己共役演算子の固有値は、すべて実数である。
\end{itembox}
\textbf{Prf.}
\begin{align}
\hat{A}\ket{a} =a\ket{a}
\end{align}
の両辺に$\bra{a}$を作用させて
\begin{align}
\bra{a}\hat{A}\ket{a} =a\braket{a}
\end{align}
また、
\begin{align}
\bra{a}\hat{A} =\bra{a}a^{*}
\end{align}
の両辺に$\ket{a}$を作用させて
\begin{align}
\bra{a}\hat{A}\ket{a} =a^{*}\braket{a}
\end{align}
$\braket{a}{a}\neq 0$より、
\begin{align}
a=a^{*}
\end{align}
hogehoge
\newpage

\begin{itembox}[l]{\textbf {定理3.2}}
自己共役演算子の相異なる固有値に属する固有ベクトルは、直交し、線形独立である。\\
したがって、相異なる固有値の数は$\mathrm{dim}\mathcal{H}$が有限であれば、離散スペクトルしか現れない。
\end{itembox}
\subsection{正規直交基底と波動関数(離散固有値の場合)}
一つの自己共役演算子固有ベクトルを、縮退しているものまで含めてもれなく集めてくると、それらは完全系を成す。
\begin{itembox}[l]{\textbf {定理3.3}}
自己共役演算子の固有ベクトルの全体は、完全系を成す。
\end{itembox}
したがって、hogehoge
\\
この例のように\footnote{書いてない。}
縮退がある場合、一つの固有値aに属する$m_a$この固有ベクトル$\ket{a,l}(l=1,2,\cdots,m_a)$は、互いに直交している保証はない。\\
しかし、シュミットの直行化法により、これらの線形結合をとり、互いに直交するように選びなおすことができる。\\
\textbf{シュミットの直行化法}\\
hogehoge

\subsection{自由度}
物理においては、基本変数を用いて理論を構成する。
ここで、基本変数とは、「系のすべての物理量を構成できるような変数」のことである。(例えば、古典力学においては、(q,p)を用いてハミルトニアンを書き下し、他の物理量を表す。\footnote{一般に、量子力学において、基本変数は可観測量とは限らない。たとえば、調和振動子の運動を昇降演算子を用いて表すことができるが、昇降演算子には$q$と$p$が含まれていて、量子論の枠組み(1)より、昇降演算子は測定不可能な量である
。})\\
必要な基本変数の組の数を自由度と呼ぶ。(例えば、三次元自由粒子の運動は、$(q_i,p_i)(i=1,2,3)$の三組によって記述され、自由度は3)\\
有限の基本変数で書ける系を有限自由系といい、そうでない形を無限自由系と呼ぶ。\\

\subsection{閉じた系}
考える系が、他の系と一切相互作用しない場合、その系を閉じた系と呼び、そうでない系を開いた系と呼ぶ。\\
\subsection{純粋状態/混合状態}
純粋状態とは、原理的に許される最大限のところまで状態を指定しつくした状態のことで、そうでない状態のことを混合状態と呼ぶ。\\
これを定式化すると、$\psi _j$により混合された量子状態は
\begin{align}
P_{\psi}(a)=\Sigma \lambda _j P_{\psi _j}(a)\quad (0<\lambda _j <1,\Sigma_j \lambda _j=1)
\end{align}
と書くことができ、とくに純粋状態ではjがひとつのみになる。\\
\noindent{\textbf{純粋状態と混合状態の例}}\\
1電子についてのスピンについて考える。\\
いつか気が向いたら書く。
\subsection{純粋状態/混合状態}
上の例からわかるように、混合状態の純粋状態への分解は一意的でない。
\\


\end{document}